%\documentclass[a4paper]{article}
%%\documentclass{tuftebook}
\usepackage{amsthm}
\usepackage[many]{tcolorbox}

\makeatletter

\makeatletter

\def\renewtheorem#1{%
	\expandafter\let\csname#1\endcsname\relax
	\expandafter\let\csname c@#1\endcsname\relax
	\gdef\renewtheorem@envname{#1}
	\renewtheorem@secpar
}
\def\renewtheorem@secpar{\@ifnextchar[{\renewtheorem@numberedlike}{\renewtheorem@nonumberedlike}}
\def\renewtheorem@numberedlike[#1]#2{\newtheorem{\renewtheorem@envname}[#1]{#2}}
\def\renewtheorem@nonumberedlike#1{
	\def\renewtheorem@caption{#1}
	\edef\renewtheorem@nowithin{\noexpand\newtheorem{\renewtheorem@envname}{\renewtheorem@caption}}
	\renewtheorem@thirdpar
}
\def\renewtheorem@thirdpar{\@ifnextchar[{\renewtheorem@within}{\renewtheorem@nowithin}}
\def\renewtheorem@within[#1]{\renewtheorem@nowithin[#1]}

\makeatother

%%%%%%%%%%%%%%%%%%%%
% New environments %
%%%%%%%%%%%%%%%%%%%%

\makeatother
%\mdfsetup{skipabove=1em,skipbelow=0em}

\tcbuselibrary{skins}

% Color definitions

\definecolor{proofcolor}{RGB}{0,0,0}

% Dark orange and Dark Red rgb
\definecolor{theorembordercolor}{RGB}{151, 63, 5}
\definecolor{theorembackgroundcolor}{RGB}{248, 241, 234}

\definecolor{examplebordercolor}{RGB}{0, 110, 184}
\definecolor{examplebackgroundcolor}{RGB}{240, 244, 250}

\definecolor{definitionbordercolor}{RGB}{0, 150, 85}
\definecolor{definitionbackgroundcolor}{RGB}{239, 247, 243}

\definecolor{propertybordercolor}{RGB}{128, 0, 128}
\definecolor{propertybackgroundcolor}{RGB}{255, 240, 255}

\definecolor{formulabordercolor}{RGB}{0, 0, 0}
\definecolor{formulabackgroundcolor}{RGB}{230, 229, 245}

\newtheoremstyle{theorem}
{0pt}{0pt}{\normalfont}{0pt}
{}{\;}{0.25em}
{{\sffamily\bfseries\color{theorembordercolor}\thmname{#1}~\thmnumber{\textup{#2}}.}
	\thmnote{\normalfont\color{black}~(#3)}}

\newtheoremstyle{definition}
{0pt}{0pt}{\normalfont}{0pt}
{}{\;}{0.25em}
{{\sffamily\bfseries\color{definitionbordercolor}\thmname{#1}~\thmnumber{\textup{#2}}.}
	\thmnote{\normalfont\color{black}~(#3)}}

\newtheoremstyle{example}
{0pt}{0pt}{\normalfont}{0pt}
{}{\;}{0.25em}
{{\sffamily\bfseries\color{examplebordercolor}\thmname{#1}.}
	\thmnote{\normalfont\color{black}~(#3)}}

\newtheoremstyle{property}
{0pt}{0pt}{\normalfont}{0pt}
{}{\;}{0.25em}
{{\sffamily\bfseries\color{propertybordercolor}\thmname{#1}~\thmnumber{\textup{#2}}.}
	\thmnote{\normalfont\color{black}~(#3)}}

\newtheoremstyle{formula}
{0pt}{0pt}{\normalfont}{0pt}
{}{\;}{0.25em}
{{\sffamily\bfseries\color{formulabordercolor}\thmname{#1}~\thmnumber{\textup{#2}}.}
	\thmnote{\normalfont\color{black}~(#3)}}

%%%%%%%%%%%%%%%%%%%%%%%%
% Theorem Environments %
%%%%%%%%%%%%%%%%%%%%%%%%

\theoremstyle{theorem}

\newtheorem{theorem}{Theorem}
\newtheorem{postulate}{Postulate}
\newtheorem{conjecture}{Conjecture}
\newtheorem{corollary}{Corollary}
\newtheorem{lemma}{Lemma}
\newtheorem{conclusion}{Conclusion}

\tcolorboxenvironment{theorem}{
	enhanced jigsaw, pad at break*=1mm, breakable,
	left=4mm, right=4mm, top=1mm, bottom=1mm,
	colback=theorembackgroundcolor, boxrule=0pt, frame hidden,
	borderline west={0.5mm}{0mm}{theorembordercolor}, arc=.5mm
}
\tcolorboxenvironment{postulate}{
	enhanced jigsaw, pad at break*=1mm, breakable,
	left=4mm, right=4mm, top=1mm, bottom=1mm,
	colback=theorembackgroundcolor, boxrule=0pt, frame hidden,
	borderline west={0.5mm}{0mm}{theorembordercolor}, arc=.5mm
}
\tcolorboxenvironment{conjecture}{
	enhanced jigsaw, pad at break*=1mm, breakable,
	left=4mm, right=4mm, top=1mm, bottom=1mm,
	colback=theorembackgroundcolor, boxrule=0pt, frame hidden,
	borderline west={0.5mm}{0mm}{theorembordercolor}, arc=.5mm
}
\tcolorboxenvironment{corollary}{
	enhanced jigsaw, pad at break*=1mm, breakable,
	left=4mm, right=4mm, top=1mm, bottom=1mm,
	colback=theorembackgroundcolor, boxrule=0pt, frame hidden,
	borderline west={0.5mm}{0mm}{theorembordercolor}, arc=.5mm
}
\tcolorboxenvironment{lemma}{
	enhanced jigsaw, pad at break*=1mm, breakable,
	left=4mm, right=4mm, top=1mm, bottom=1mm,
	colback=theorembackgroundcolor, boxrule=0pt, frame hidden,
	borderline west={0.5mm}{0mm}{theorembordercolor}, arc=.5mm
}
\tcolorboxenvironment{conclusion}{
	enhanced jigsaw, pad at break*=1mm, breakable,
	left=4mm, right=4mm, top=1mm, bottom=1mm,
	colback=theorembackgroundcolor, boxrule=0pt, frame hidden,
	borderline west={0.5mm}{0mm}{theorembordercolor}, arc=.5mm
}

%%%%%%%%%%%%%%%%%%%%%%%%%%%
% Definition Environments %
%%%%%%%%%%%%%%%%%%%%%%%%%%%

\theoremstyle{definition}
\newtheorem{definition}{Definition}
\newtheorem{review}{Review}

\tcolorboxenvironment{definition}{
	enhanced jigsaw, pad at break*=1mm, breakable,
	left=4mm, right=4mm, top=1mm, bottom=1mm,
	colback=definitionbackgroundcolor, boxrule=0pt, frame hidden,
	borderline west={0.5mm}{0mm}{definitionbordercolor}, arc=.5mm
}
\tcolorboxenvironment{review}{
	enhanced jigsaw, pad at break*=1mm, breakable,
	left=4mm, right=4mm, top=1mm, bottom=1mm,
	colback=definitionbackgroundcolor, boxrule=0pt, frame hidden,
	borderline west={0.5mm}{0mm}{definitionbordercolor}, arc=.5mm
}


%%%%%%%%%%%%%%%%%%%%%%%%
% Example Environments %
%%%%%%%%%%%%%%%%%%%%%%%%

\theoremstyle{example}
\newtheorem*{example}{Example}
\newtheorem*{remark}{Remark}
\newtheorem*{note}{Note}

\tcolorboxenvironment{example}{
	enhanced jigsaw, pad at break*=1mm, breakable,
	left=4mm, right=4mm, top=1mm, bottom=1mm,
	colback=examplebackgroundcolor, boxrule=0pt, frame hidden,
	borderline west={0.5mm}{0mm}{examplebordercolor}, arc=.5mm
}
\tcolorboxenvironment{remark}{
	enhanced jigsaw, pad at break*=1mm, breakable,
	left=4mm, right=4mm, top=1mm, bottom=1mm,
	colback=white, boxrule=0pt, frame hidden,
	borderline west={0.5mm}{0mm}{examplebordercolor}, arc=.5mm
}
\tcolorboxenvironment{note}{
	enhanced jigsaw, pad at break*=1mm, breakable,
	left=4mm, right=4mm, top=1mm, bottom=1mm,
	colback=white, boxrule=0pt, frame hidden,
	borderline west={0.5mm}{0mm}{examplebordercolor}, arc=.5mm
}


%%%%%%%%%%%%%%%%%%%%%%%%%
% Property Environments %
%%%%%%%%%%%%%%%%%%%%%%%%%

\theoremstyle{property}
\newtheorem{property}{Property}
\newtheorem{proposition}{Proposition}

\tcolorboxenvironment{property}{
	enhanced jigsaw, pad at break*=1mm, breakable,
	left=4mm, right=4mm, top=1mm, bottom=1mm,
	colback=propertybackgroundcolor, boxrule=0pt, frame hidden,
	borderline west={0.5mm}{0mm}{propertybordercolor}, arc=.5mm
}
\tcolorboxenvironment{proposition}{
	enhanced jigsaw, pad at break*=1mm, breakable,
	left=4mm, right=4mm, top=1mm, bottom=1mm,
	colback=propertybackgroundcolor, boxrule=0pt, frame hidden,
	borderline west={0.5mm}{0mm}{propertybordercolor}, arc=.5mm
}

%%%%%%%%%%%%
% Formula %
%%%%%%%%%%%%

\theoremstyle{formula}
\newtheorem{formula}{Formula}

\tcolorboxenvironment{formula}{
	enhanced jigsaw, pad at break*=1mm, breakable,
	left=4mm, right=4mm, top=1mm, bottom=1mm,
	colback=formulabackgroundcolor, boxrule=0pt, frame hidden,
	borderline west={0.5mm}{0mm}{formulabordercolor}, arc=.5mm
}

%%%%%%%%%
% Proof %
%%%%%%%%%

% These patches must be placed after \tcolorboxenvironment !
\AddToHook{env/theorem/after}{\colorlet{proofcolor}{theorembordercolor}}
\AddToHook{env/postulate/after}{\colorlet{proofcolor}{theorembordercolor}}
\AddToHook{env/conjecture/after}{\colorlet{proofcolor}{theorembordercolor}}
\AddToHook{env/corollary/after}{\colorlet{proofcolor}{theorembordercolor}}
\AddToHook{env/lemma/after}{\colorlet{proofcolor}{theorembordercolor}}
\AddToHook{env/conclusion/after}{\colorlet{proofcolor}{theorembordercolor}}

\AddToHook{env/definition/after}{\colorlet{proofcolor}{definitionbordercolor}}
\AddToHook{env/review/after}{\colorlet{proofcolor}{definitionbordercolor}}

\AddToHook{env/example/after}{\colorlet{proofcolor}{examplebordercolor}}
\AddToHook{env/remark/after}{\colorlet{proofcolor}{examplebordercolor}}
\AddToHook{env/note/after}{\colorlet{proofcolor}{examplebordercolor}}

\AddToHook{env/property/after}{\colorlet{proofcolor}{propertybordercolor}}
\AddToHook{env/proposition/after}{\colorlet{proofcolor}{propertybordercolor}}

\AddToHook{env/formula/after}{\colorlet{proofcolor}{formulabordercolor}}

\renewcommand{\qedsymbol}{Q.E.D.}
\let\qedsymbolMyOriginal\qedsymbol
\renewcommand{\qedsymbol}{
	\color{proofcolor}\qedsymbolMyOriginal
}

\newtheoremstyle{proof}
{0pt}{0pt}{\normalfont}{0pt}
{}{\;}{0.25em}
{{\sffamily\bfseries\color{proofcolor}\thmname{#1}.}
	\thmnote{\normalfont\color{black}~(\textit{#3})}}

\theoremstyle{proof}
\renewtheorem{proof}{Proof}

\tcolorboxenvironment{proof}{
	enhanced jigsaw, pad at break*=1mm, breakable,
	left=4mm, right=4mm, top=1mm, bottom=1mm,
	colback=white, boxrule=0pt, frame hidden,
	borderline west={0.5mm}{0mm}{proofcolor}, arc=.5mm
}

\newenvironment{info}{\begin{tcolorbox}[
		arc=0mm,
		colback=white,
		colframe=gray,
		title=Info,
		fonttitle=\sffamily,
		breakable
		]}{\end{tcolorbox}}
\newenvironment{terminology}{\begin{tcolorbox}[
		arc=0mm,
		colback=white,
		colframe=green!60!black,
		title=Terminology,
		fonttitle=\sffamily,
		breakable
		]}{\end{tcolorbox}}
\newenvironment{warning}{\begin{tcolorbox}[
		arc=0mm,
		colback=white,
		colframe=red,
		title=Warning,
		fonttitle=\sffamily,
		breakable
		]}{\end{tcolorbox}}
\newenvironment{caution}{\begin{tcolorbox}[
		arc=0mm,
		colback=white,
		colframe=yellow,
		title=Caution,
		fonttitle=\sffamily,
		breakable
		]}{\end{tcolorbox}}

%\date{}
%\title{\flushleft \textbf{Compactness}}
%\begin{document}
%	\maketitle
	
	\begin{definition}[Compact]
		Let $(X,d)$ be a metric space and let $K \subseteq X$. $K$ is said to be compact if every open cover of $K$ has a finite subcover. That is, if $\{O_\alpha\}_{\alpha \in \Lambda}$ is any open cover of $K$, then
		$$\exists \alpha_1, ..., \alpha_n \text{ such that } K \subseteq O_{\alpha_1} \cup ... \cup O_{\alpha_n} $$
	\end{definition}
	
	\begin{example}
		Let $(X,d)$ be a metric space and let $E \subseteq X$. \\
		If $E$ is finite, then $E$ is compact.
	\end{example}
	\begin{proof}
	The reason is as follows:\\
	Let $\{O_\alpha\}_{\alpha \in \Lambda}$ be any open cover of $E$. Our goal is to show that this open cover has a finite subcover. \\
	If $E= \emptyset$, there is nothing to prove. \\
	If $E \not = \emptyset$, denote the elements of $E$ by $x_1,...x_n:$ $$E=\{x_1,...,x_n\}$$.
	We have:
	\begin{align*}
		x_1 \in E \subseteq \bigcup \limits_{\alpha \in \Lambda} O_\alpha &\implies \exists \alpha_1 \in \Lambda \text { such that } x_1 \in O_{\alpha_1} \\
		x_2 \in E \subseteq \bigcup \limits_{\alpha \in \Lambda} O_\alpha &\implies \exists \alpha_2\in \Lambda \text { such that } x_2 \in O_{\alpha_2} \\
		\vdots \\
		x_n \in E \subseteq \bigcup \limits_{\alpha \in \Lambda} O_\alpha &\implies \exists \alpha_n\in \Lambda \text { such that } x_n \in O_{\alpha_n} 
	\end{align*}
	Hence,
	$$E = {x_1, ..., x_n} \subseteq O_{\alpha_1} \cup ... \cup O_{\alpha_n}$$
	So, $O_{\alpha_1},...,O_{\alpha_n}$ is a finite subcover of $E$. \qed \\
	\end{proof}
	
	\begin{example}
		Consider $( \mathbb{R} , | |)$ and let $E=\{ \frac{1}{n} : n \in \mathbb{N} \} \cup \{0\}$. \\
		Prove that $E$ is compact. (In general, if $a_n \rightarrow a$ in $\mathbb{R}$ then $F=\{a_n : n \in \mathbb{N} \} \cup \{a\}$ is compact.)
	\end{example}
	\begin{proof}
		\label{prf:proof_3}
		Let $\{O_\alpha\}_{alpha\in \Lambda}$ be any open cover of $E$. Our goal is to show that this open cover has a finite subcover.
		\begin{align*}
			\begin{rcases}
				0 \in E \\
				E \subseteq \bigcup \limits_{\alpha \in \Lambda} O_\alpha
			\end{rcases}
			&\implies 0\in \bigcup \limits_{\alpha \in \Lambda} O_\alpha \implies \exists \alpha_0 \in \Lambda \text{ such that } 0 \in O_{\alpha_0}  \tag{$I$} \\
			\begin{rcases}
				0 \in O_{\alpha_0} \\
				O_{\alpha_0} \text{ is open }
			\end{rcases}
			&\implies \exists \epsilon > 0 \text{ such that } (-\epsilon, \epsilon) \subseteq O_{\alpha_0}
		\end{align*}
		By the archimedean property of $\mathbb{R}$,
		$$ \exists m \in \mathbb{N} \text{ such that } \frac{1}{n} < \epsilon$$
		so
		$$\forall n \geq m ~~~ \frac{1}{n} < \epsilon.$$
		Hence
		\begin{equation}
			\forall n \geq m ~~~ \frac{1}{n} \in (-\epsilon, \epsilon) \subseteq O_{\alpha_0} \tag{$II$}
		\end{equation}
		Notice that $E=\{0, \frac{1}{1}, \frac{1}{2},\frac{1}{3},...,\frac{1}{m-1}, \frac{1}{m},\frac{1}{m+1},\frac{1}{m+2},... \} \text{ for } m\in\mathbb{N}.$ All that remains is to find a subcover for the elements $\frac{1}{1}, ..., \frac{1}{m-1}:$ \\
		\begin{align*}
			1 \in E &\implies \exists \alpha_1 \in \Lambda \text{ such that } 1 \in O_{\alpha_1} \\
			\frac{1}{2} \in E &\implies \exists \alpha_2 \in \Lambda \text { such that } \frac{1}{2} \in O_{\alpha_2} \\
			\vdots \\
			\frac{1}{m-1} \in E &\implies \exists \alpha_{m-1} \in \Lambda \text{ such that } \frac{1}{m-1} \in O_{\alpha_{m-1}} \tag{$III$}
		\end{align*}
		By ($I$), ($II$), and ($III$), we have $$E \subseteq O_{\alpha_0} \cup ... \cup O_{\alpha_{m-1}}$$
		Thus, $\{O_\alpha\}_{\alpha\in \Lambda}$ has a finite subcover. Therefore $E$ is compact. \qed
	\end{proof}
	
	\begin{remark}
		If $X$ itself is compact, we say $(X,d)$ is a compact metric space.
		If $\ocover{O}$ is any collection of open sets such that $X= \unioncollect{O},$ then
		$$\exists \alpha_{1}, ..., \alpha_n \in \Lambda \st X = O_{\alpha_1} \cup ... \cup O_{\alpha_n}.$$
	\end{remark}
	
	\begin{theorem}
		Compact subsets of metric spaces are closed.
	\end{theorem}
	\begin{proof}
		\routineMS and let $K\subseteq X$ be compact. We want to show that $K$ is closed. It is enough to show that $K^c$ is open. To this end, we need to show that every point of $K^c$ is an interior point. \\
		Let $a\in K^c$. Our goal is to show that $$\exists \epsilon > 0 \st \nbhd{\epsilon}{a} \subseteq K^c.$$
		That is, we want to show that $$\exists \epsilon > 0 \st \nbhd{\epsilon}{a} \cap K = \emptyset.$$
		We have
		\begin{align*}
			a \in K^c &\implies a \not \in K \\
			&\implies \forall x \in K ~~ d(x,a) > 0.
		\end{align*}
		For all $x\in K$, let $$\epsilon_x = \frac{1}{4}d(x,a).$$
		Clearlly, $$\forall x \in K ~~ \nbhd{\epsilon_x}{x} \cap \nbhd{\epsilon_x}{a} = \emptyset.$$
		Notice that $$\{\nbhd{\epsilon_x}{x}\}_{x\in K} \text { is an open cover of } K.$$
		Since $K$ is compact, there is a finite subcover $$\exists x_1,..., x_n \in K \st K \subseteq \nbhd{\epsilon_{x_1}}{x_1} \cup ... \cup \nbhd{\epsilon_{x_n}}{x_n}$$
		and of course
		$$\begin{cases}
			\nbhd{\epsilon_{x_1}}{x_1}\cap \nbhd{\epsilon_{x_n}}{a} = \emptyset \\
			\vdots \\
			\nbhd{\epsilon_{x_n}}{x_n}\cap \nbhd{\epsilon_{x_n}}{a} = \emptyset
		\end{cases}$$
		Let $\epsilon=\min\{\epsilon_{x_1}, ..., \epsilon_{x_n}\}$. Clearly,
		$$\nbhd{\epsilon}{a} \subseteq \nbhd{\epsilon_{x_i}}{a} ~~ \forall 1 \leq i \leq n.$$
		Hence
		$$\begin{cases}
			\nbhd{\epsilon_{x_1}}{x_1} \cap \nbhd{\epsilon}{a} = \emptyset \\
			\vdots \\
			\nbhd{\epsilon_{x_n}}{x_n} \cap \nbhd{\epsilon}{a} = \emptyset
		\end{cases}$$
		Therefore
		$$
		\nbhd{\epsilon}{a} \cap [ \nbhd{\epsilon_{x_1}}{x_1}\cup ... \cup \nbhd{\epsilon_{x_n}}{x_n}] =\emptyset.
		$$
		So,
		$$
			\nbhd{\epsilon}{a} \cap K = \emptyset.
		$$
		\qed
	\end{proof}
	
	\begin{note}
			So, it has been shown that compact $\implies$ closed and bounded $\checkmark$. However, it is not necessarily the case that closed and bounded $\implies$ compact.
	\end{note}
	
	\begin{theorem}
		\routineMS and let \routineCompact. Let $E\subseteq K$ be closed. Then $E$ is compact.
	\end{theorem}
	
	\begin{proof}
		Let $\ocover{O}$ be an open cover of $E$. Our goal is to show that this cover has a finite subcover. Not that $$E \text{ is closed} \implies E^c \text{ is open}.$$
		We have $$E \subseteq K \subseteq X = E\cup E^c \subseteq \left(\unioncollect{O}\right) \cup E^c$$
		Therefore, $E^c$ together with $\ocover{O}$ is an open cover for the compact set $K$. Since $K$ is compact, this open cover has a finite subcover:
		$$\exists \alpha_1,...,\alpha_n \in \Lambda \st K\subseteq O_{\alpha_1} \cup ... \cup O_{\alpha_n} \cup E^c.$$
		Considering $E\subseteq K,$ we can write
		$$E \subseteq O_{\alpha_1}\cup ... \cup O_{\alpha_n} \cup E^c.$$
		However, $E\cap E^c = \emptyset$, so
		$$E \subseteq O_{\alpha_1} \cup... \cup O_{\alpha_n}.$$
		So, $O_{\alpha_1} ,... , O_{\alpha_n}$ can be considered as the finite subcover that we were looking for. \qed
	\end{proof}
	
	\begin{corollary}
		If $F$ is closed and $K$ is compact, then $F\cap K$ is compact. ($F\cap K$ is a closed subset of the compact set $K$)
	\end{corollary}
	
	Consider $X=\R$ and $Y=[0, \infty)$ ($Y$ is a subspace of $X$). Then
	$$[0, \epsilon) \text{ is open in } Y \text{ because } [0,\epsilon) = (-\epsilon, \epsilon)\cap Y.$$
	
	\begin{theorem}
		\routineMS and let $K\subseteq Y \subseteq X$ with $Y \not = \emptyset$. $K$ is compact relative to $X$ if and only if $K$ is compact relative to $Y$.
	\end{theorem}
	
	\begin{proof}
		$(\Leftarrow)$ Suppose $K$ is compact relative to $Y$. We want to show $K$ is compact relative $X$. Let $\ocover{O}$ be a collection of open sets in $X$ that covers $K$. Our goal is to show that this cover has a finite subcover. Note that
		$$K=K\cap Y \subseteq \left( \unioncollect{O} \right)\cap Y = \bigcup \limits_{\alpha \in \Lambda} \left(O_\alpha \cap Y \right).$$
		By Theorem 2.30, for each $\alpha \in \Lambda$, $O_\alpha \cap Y$ is an open set in the metric space $(Y,d^Y).$ So, $\{O_\alpha \cap Y\}_{\alpha \in \Lambda}$ is a collection of open sets in $(Y,d^Y)$ that covers $K$. Since $K$ is compact relative to $Y$, there exists a finite subcover:
		\begin{align*}
			\exists \alpha_1,...,\alpha_n \in \Lambda \st K &\subseteq (O_{\alpha_1}\cap Y)\cup...\cup(O_{\alpha_n}\cap Y) \\
			&\subseteq(O_{\alpha_1}\cup ... \cup O_{\alpha_n})\cap Y \\
			&\subseteq O_{\alpha_1}\cup ... \cup O_{\alpha_n}\\
			&\implies K \subseteq O_{\alpha_1}\cup ... \cup O_{\alpha_n} \text{(we have a finite subcover)}
		\end{align*}
		
		$(\Rightarrow)$ Now suppose $K$ is compact relative to $X$. We want to show $K$ is compact relative to $Y$. Let $\ocover{G}$ be a collection of open sets in $(Y, d^Y)$ that covers $K$. Our goal is to show that this cover has a finite subcover. It follows from Theorem 2.30 that
		$$\forall \alpha \in \Lambda ~~ \exists O_{\alpha_{\text{open}}} \subseteq X \st G_\alpha = O_\alpha \cap Y.$$
		We have
		$$K \subseteq \unioncollect{G} = \bigcup \limits_{\alpha \in \Lambda} \left(O_\alpha \cap Y\right) = \left(\unioncollect{O}\right) \cap Y \subseteq \unioncollect{O}.$$
		So, $\ocover{O}$ is an open cover for $K$ in the metric space $(X,d)$. Since $K$ is compact, $$\exists \alpha_1,...,\alpha_n \in \Lambda \st K \subseteq O_{\alpha_1}\cup... \cup O_{\alpha_n}.$$
		Therefore, $$K=K\cap Y \subseteq (O_{\alpha_1}\cup ... \cup O_{\alpha_n})\cap y = (O_{\alpha_1}\cap Y)\cup ... \cup (O_{\alpha_n}\cap Y) = G_{\alpha_1}\cup...\cup G_{\alpha_n}.$$
		(We have found the finite subcover we were looking for) \qed
	\end{proof}
	
	Consider $X=\R$ and $Y=(0,\infty).$ \\
	$(0, 2]$ is closed and bounded in $Y$, but it is not closed and bounded in $\R$.
	$$(0,2]=[-2,2]\cap Y$$
	
	\begin{theorem}
		If $E$ is an infinite subset of a compact set $K$, then $E$ has a limit point in $K$. $E'\cap K \not = \emptyset$.
	\end{theorem}
	
	\begin{proof}
		Assume foolishly that $E'\cap K = \emptyset$; for every point you select in $K$, that point will not be a limit point of $E$. That is,
		$$
		\begin{cases}
			\forall a \in E &a\not \in E' \\
			\forall b \in K \backslash E &b\not \in E'
		\end{cases}
		$$
		Therefore,
		$$
		\begin{cases}
			\forall a \in E ~\exists \epsilon_a > 0 \st \nbhd{\epsilon_a}{a}\cap (E \backslash \{a\})=\emptyset \\
			\forall b \in K \backslash E ~\exists \delta_b > 0 \st \nbhd{\delta_b}{b}\cap (E \backslash \{b\})=\emptyset \\
		\end{cases}
		$$
		Thus
		$$
		\begin{cases}
			\forall a \in E ~\exists \epsilon_a > 0 \st \nbhd{\epsilon_a}{a}\cap E = \{a\} \\
			\forall b \in K \backslash E ~\exists \delta_b > 0 \st \nbhd{\epsilon_b}{b}\cap E = \emptyset
		\end{cases}
		$$
		Clearly, $K \subseteq \left( \bigcup \limits_{a\in E} \nbhd{\epsilon_a}{a}
		\right) \cup \left( \bigcup \limits_{b\in K\backslash E} \nbhd{\delta_b}{b}\right).$ Since $K$ is compact,
		
		$$\exists a_1,...,a_n \in E, b_1,...,b_n \in K \backslash E \st \\ E \subseteq K \subseteq \left(\nbhd{\epsilon_{a_1}}{a_1}\cup...\cup \nbhd{\epsilon_{a_n}}{a_n}\right)\cup \left(\nbhd{\delta_{b_1}}{b_1}\cup...\cup \nbhd{\delta_{b_n}}{b_n}\right) $$
		
		Since for all $b\in K \backslash E$, $\nbhd{\delta_b}{b}\cap E = \emptyset,$ we can conclude that
		$$E \subseteq \left( \nbhd{\epsilon_{a_1}}{a_1} \cup...\cup \nbhd{\epsilon_{a_n}}{a_n} \right)$$
		Hence, 
		
		\begin{align*}
			E &= E \cap \left[ N_{\epsilon_{a_1}}a_1 \cup ... \cup N_{\epsilon_{a_n}}a_n \right] \\
				&= \left [E\cap \nbhd{\epsilon_{a_1}}{a_1} \right] \cup ... \cup \left[ E \cap \nbhd{\epsilon_{a_n}}{a_n}\right] \\
				&= \{a_1\} \cup ... \cup \{a_n\} \\
				&= \{a_1,...,a_n\}.
		\end{align*}
		
		This contradicts the assumption that E is infinite. \qed
	\end{proof}
	
	\begin{remark}
		\begin{enumerate}
			\item $K$ is compact
			\item Every infinite subset of $K$ has a limit point in $K$
			\item Every sequence in $K$ has a subsequence that converges to a point in $K$
		\end{enumerate}
	\end{remark}
	$\overset{A_1}{[1, \infty]}, \overset{A_2}{[2, \infty]}, \overset{A_3}{[3,\infty]}, \overset{A_4}{[4,\infty]},...$
	\begin{align*}A_2\cap A_3 \cap A_4 &= [4, \infty) = A_4 \\
	A_1 \cap A_3 \cap A_4&= A_4 \\
	\bigcap \limits_{n=1}^\infty A_n = \emptyset
	\end{align*}
	
	\begin{theorem}
		\routineMS, and let $\{K_\alpha\}_{\alpha \in \Lambda}$ be a collection of compact sets. Every finite intersection is nonempty.
	\end{theorem}
	\begin{proof}
		Assume for contradiction that $\bigcap \limits_{\alpha \in \Lambda} K_\alpha = \emptyset.$ Let $\alpha_0 \in \Lambda$. We have
		$$K_{\alpha_0} \cap \left(\bigcap \limits_{\alpha \not = \alpha_0} K_alpha \right) = \emptyset$$
		So,
		$$k_{alpha_0}  \subseteq \left(\bigcup \limits_{\alpha \in \Lambda, \alpha \not = \alpha_0} K_\alpha \right)^c
		\implies K_{\alpha_0} \subseteq \bigcup \limits_{a\alpha \in Lambda, \alpha \not = \alpha_0} K_\alpha ^c$$
		So, ${\ocover{K^c}} _{,\alpha \not = \alpha_0}$ is an open cover of $K_{\alpha_0}$. Since $K_{\alpha_0}$ is compact,
		$$\exists \alpha_1, ..., \alpha_n \st K_{\alpha_0} \subseteq K_{\alpha_1}^c \cap ... \cap K_{\alpha_n}^c \subseteq \left( \bigcap \limits_{i = 1}^n K_{\alpha_i} \right)^c$$
		So,
		$$K_{\alpha_0} \cap \left(\bigcap \limits_{i=1}^n K_{\alpha_i} \right) = \emptyset.$$
		This contradicts the assumption that every finite intersection is nonempty. \qed
	\end{proof}
%\end{document}