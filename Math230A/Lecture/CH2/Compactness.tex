\documentclass[a4paper]{article}
%\documentclass{tuftebook}
\usepackage{amsthm}
\usepackage[many]{tcolorbox}

\makeatletter

\makeatletter

\def\renewtheorem#1{%
	\expandafter\let\csname#1\endcsname\relax
	\expandafter\let\csname c@#1\endcsname\relax
	\gdef\renewtheorem@envname{#1}
	\renewtheorem@secpar
}
\def\renewtheorem@secpar{\@ifnextchar[{\renewtheorem@numberedlike}{\renewtheorem@nonumberedlike}}
\def\renewtheorem@numberedlike[#1]#2{\newtheorem{\renewtheorem@envname}[#1]{#2}}
\def\renewtheorem@nonumberedlike#1{
	\def\renewtheorem@caption{#1}
	\edef\renewtheorem@nowithin{\noexpand\newtheorem{\renewtheorem@envname}{\renewtheorem@caption}}
	\renewtheorem@thirdpar
}
\def\renewtheorem@thirdpar{\@ifnextchar[{\renewtheorem@within}{\renewtheorem@nowithin}}
\def\renewtheorem@within[#1]{\renewtheorem@nowithin[#1]}

\makeatother

%%%%%%%%%%%%%%%%%%%%
% New environments %
%%%%%%%%%%%%%%%%%%%%

\makeatother
%\mdfsetup{skipabove=1em,skipbelow=0em}

\tcbuselibrary{skins}

% Color definitions

\definecolor{proofcolor}{RGB}{0,0,0}

% Dark orange and Dark Red rgb
\definecolor{theorembordercolor}{RGB}{151, 63, 5}
\definecolor{theorembackgroundcolor}{RGB}{248, 241, 234}

\definecolor{examplebordercolor}{RGB}{0, 110, 184}
\definecolor{examplebackgroundcolor}{RGB}{240, 244, 250}

\definecolor{definitionbordercolor}{RGB}{0, 150, 85}
\definecolor{definitionbackgroundcolor}{RGB}{239, 247, 243}

\definecolor{propertybordercolor}{RGB}{128, 0, 128}
\definecolor{propertybackgroundcolor}{RGB}{255, 240, 255}

\definecolor{formulabordercolor}{RGB}{0, 0, 0}
\definecolor{formulabackgroundcolor}{RGB}{230, 229, 245}

\newtheoremstyle{theorem}
{0pt}{0pt}{\normalfont}{0pt}
{}{\;}{0.25em}
{{\sffamily\bfseries\color{theorembordercolor}\thmname{#1}~\thmnumber{\textup{#2}}.}
	\thmnote{\normalfont\color{black}~(#3)}}

\newtheoremstyle{definition}
{0pt}{0pt}{\normalfont}{0pt}
{}{\;}{0.25em}
{{\sffamily\bfseries\color{definitionbordercolor}\thmname{#1}~\thmnumber{\textup{#2}}.}
	\thmnote{\normalfont\color{black}~(#3)}}

\newtheoremstyle{example}
{0pt}{0pt}{\normalfont}{0pt}
{}{\;}{0.25em}
{{\sffamily\bfseries\color{examplebordercolor}\thmname{#1}.}
	\thmnote{\normalfont\color{black}~(#3)}}

\newtheoremstyle{property}
{0pt}{0pt}{\normalfont}{0pt}
{}{\;}{0.25em}
{{\sffamily\bfseries\color{propertybordercolor}\thmname{#1}~\thmnumber{\textup{#2}}.}
	\thmnote{\normalfont\color{black}~(#3)}}

\newtheoremstyle{formula}
{0pt}{0pt}{\normalfont}{0pt}
{}{\;}{0.25em}
{{\sffamily\bfseries\color{formulabordercolor}\thmname{#1}~\thmnumber{\textup{#2}}.}
	\thmnote{\normalfont\color{black}~(#3)}}

%%%%%%%%%%%%%%%%%%%%%%%%
% Theorem Environments %
%%%%%%%%%%%%%%%%%%%%%%%%

\theoremstyle{theorem}

\newtheorem{theorem}{Theorem}
\newtheorem{postulate}{Postulate}
\newtheorem{conjecture}{Conjecture}
\newtheorem{corollary}{Corollary}
\newtheorem{lemma}{Lemma}
\newtheorem{conclusion}{Conclusion}

\tcolorboxenvironment{theorem}{
	enhanced jigsaw, pad at break*=1mm, breakable,
	left=4mm, right=4mm, top=1mm, bottom=1mm,
	colback=theorembackgroundcolor, boxrule=0pt, frame hidden,
	borderline west={0.5mm}{0mm}{theorembordercolor}, arc=.5mm
}
\tcolorboxenvironment{postulate}{
	enhanced jigsaw, pad at break*=1mm, breakable,
	left=4mm, right=4mm, top=1mm, bottom=1mm,
	colback=theorembackgroundcolor, boxrule=0pt, frame hidden,
	borderline west={0.5mm}{0mm}{theorembordercolor}, arc=.5mm
}
\tcolorboxenvironment{conjecture}{
	enhanced jigsaw, pad at break*=1mm, breakable,
	left=4mm, right=4mm, top=1mm, bottom=1mm,
	colback=theorembackgroundcolor, boxrule=0pt, frame hidden,
	borderline west={0.5mm}{0mm}{theorembordercolor}, arc=.5mm
}
\tcolorboxenvironment{corollary}{
	enhanced jigsaw, pad at break*=1mm, breakable,
	left=4mm, right=4mm, top=1mm, bottom=1mm,
	colback=theorembackgroundcolor, boxrule=0pt, frame hidden,
	borderline west={0.5mm}{0mm}{theorembordercolor}, arc=.5mm
}
\tcolorboxenvironment{lemma}{
	enhanced jigsaw, pad at break*=1mm, breakable,
	left=4mm, right=4mm, top=1mm, bottom=1mm,
	colback=theorembackgroundcolor, boxrule=0pt, frame hidden,
	borderline west={0.5mm}{0mm}{theorembordercolor}, arc=.5mm
}
\tcolorboxenvironment{conclusion}{
	enhanced jigsaw, pad at break*=1mm, breakable,
	left=4mm, right=4mm, top=1mm, bottom=1mm,
	colback=theorembackgroundcolor, boxrule=0pt, frame hidden,
	borderline west={0.5mm}{0mm}{theorembordercolor}, arc=.5mm
}

%%%%%%%%%%%%%%%%%%%%%%%%%%%
% Definition Environments %
%%%%%%%%%%%%%%%%%%%%%%%%%%%

\theoremstyle{definition}
\newtheorem{definition}{Definition}
\newtheorem{review}{Review}

\tcolorboxenvironment{definition}{
	enhanced jigsaw, pad at break*=1mm, breakable,
	left=4mm, right=4mm, top=1mm, bottom=1mm,
	colback=definitionbackgroundcolor, boxrule=0pt, frame hidden,
	borderline west={0.5mm}{0mm}{definitionbordercolor}, arc=.5mm
}
\tcolorboxenvironment{review}{
	enhanced jigsaw, pad at break*=1mm, breakable,
	left=4mm, right=4mm, top=1mm, bottom=1mm,
	colback=definitionbackgroundcolor, boxrule=0pt, frame hidden,
	borderline west={0.5mm}{0mm}{definitionbordercolor}, arc=.5mm
}


%%%%%%%%%%%%%%%%%%%%%%%%
% Example Environments %
%%%%%%%%%%%%%%%%%%%%%%%%

\theoremstyle{example}
\newtheorem*{example}{Example}
\newtheorem*{remark}{Remark}
\newtheorem*{note}{Note}

\tcolorboxenvironment{example}{
	enhanced jigsaw, pad at break*=1mm, breakable,
	left=4mm, right=4mm, top=1mm, bottom=1mm,
	colback=examplebackgroundcolor, boxrule=0pt, frame hidden,
	borderline west={0.5mm}{0mm}{examplebordercolor}, arc=.5mm
}
\tcolorboxenvironment{remark}{
	enhanced jigsaw, pad at break*=1mm, breakable,
	left=4mm, right=4mm, top=1mm, bottom=1mm,
	colback=white, boxrule=0pt, frame hidden,
	borderline west={0.5mm}{0mm}{examplebordercolor}, arc=.5mm
}
\tcolorboxenvironment{note}{
	enhanced jigsaw, pad at break*=1mm, breakable,
	left=4mm, right=4mm, top=1mm, bottom=1mm,
	colback=white, boxrule=0pt, frame hidden,
	borderline west={0.5mm}{0mm}{examplebordercolor}, arc=.5mm
}


%%%%%%%%%%%%%%%%%%%%%%%%%
% Property Environments %
%%%%%%%%%%%%%%%%%%%%%%%%%

\theoremstyle{property}
\newtheorem{property}{Property}
\newtheorem{proposition}{Proposition}

\tcolorboxenvironment{property}{
	enhanced jigsaw, pad at break*=1mm, breakable,
	left=4mm, right=4mm, top=1mm, bottom=1mm,
	colback=propertybackgroundcolor, boxrule=0pt, frame hidden,
	borderline west={0.5mm}{0mm}{propertybordercolor}, arc=.5mm
}
\tcolorboxenvironment{proposition}{
	enhanced jigsaw, pad at break*=1mm, breakable,
	left=4mm, right=4mm, top=1mm, bottom=1mm,
	colback=propertybackgroundcolor, boxrule=0pt, frame hidden,
	borderline west={0.5mm}{0mm}{propertybordercolor}, arc=.5mm
}

%%%%%%%%%%%%
% Formula %
%%%%%%%%%%%%

\theoremstyle{formula}
\newtheorem{formula}{Formula}

\tcolorboxenvironment{formula}{
	enhanced jigsaw, pad at break*=1mm, breakable,
	left=4mm, right=4mm, top=1mm, bottom=1mm,
	colback=formulabackgroundcolor, boxrule=0pt, frame hidden,
	borderline west={0.5mm}{0mm}{formulabordercolor}, arc=.5mm
}

%%%%%%%%%
% Proof %
%%%%%%%%%

% These patches must be placed after \tcolorboxenvironment !
\AddToHook{env/theorem/after}{\colorlet{proofcolor}{theorembordercolor}}
\AddToHook{env/postulate/after}{\colorlet{proofcolor}{theorembordercolor}}
\AddToHook{env/conjecture/after}{\colorlet{proofcolor}{theorembordercolor}}
\AddToHook{env/corollary/after}{\colorlet{proofcolor}{theorembordercolor}}
\AddToHook{env/lemma/after}{\colorlet{proofcolor}{theorembordercolor}}
\AddToHook{env/conclusion/after}{\colorlet{proofcolor}{theorembordercolor}}

\AddToHook{env/definition/after}{\colorlet{proofcolor}{definitionbordercolor}}
\AddToHook{env/review/after}{\colorlet{proofcolor}{definitionbordercolor}}

\AddToHook{env/example/after}{\colorlet{proofcolor}{examplebordercolor}}
\AddToHook{env/remark/after}{\colorlet{proofcolor}{examplebordercolor}}
\AddToHook{env/note/after}{\colorlet{proofcolor}{examplebordercolor}}

\AddToHook{env/property/after}{\colorlet{proofcolor}{propertybordercolor}}
\AddToHook{env/proposition/after}{\colorlet{proofcolor}{propertybordercolor}}

\AddToHook{env/formula/after}{\colorlet{proofcolor}{formulabordercolor}}

\renewcommand{\qedsymbol}{Q.E.D.}
\let\qedsymbolMyOriginal\qedsymbol
\renewcommand{\qedsymbol}{
	\color{proofcolor}\qedsymbolMyOriginal
}

\newtheoremstyle{proof}
{0pt}{0pt}{\normalfont}{0pt}
{}{\;}{0.25em}
{{\sffamily\bfseries\color{proofcolor}\thmname{#1}.}
	\thmnote{\normalfont\color{black}~(\textit{#3})}}

\theoremstyle{proof}
\renewtheorem{proof}{Proof}

\tcolorboxenvironment{proof}{
	enhanced jigsaw, pad at break*=1mm, breakable,
	left=4mm, right=4mm, top=1mm, bottom=1mm,
	colback=white, boxrule=0pt, frame hidden,
	borderline west={0.5mm}{0mm}{proofcolor}, arc=.5mm
}

\newenvironment{info}{\begin{tcolorbox}[
		arc=0mm,
		colback=white,
		colframe=gray,
		title=Info,
		fonttitle=\sffamily,
		breakable
		]}{\end{tcolorbox}}
\newenvironment{terminology}{\begin{tcolorbox}[
		arc=0mm,
		colback=white,
		colframe=green!60!black,
		title=Terminology,
		fonttitle=\sffamily,
		breakable
		]}{\end{tcolorbox}}
\newenvironment{warning}{\begin{tcolorbox}[
		arc=0mm,
		colback=white,
		colframe=red,
		title=Warning,
		fonttitle=\sffamily,
		breakable
		]}{\end{tcolorbox}}
\newenvironment{caution}{\begin{tcolorbox}[
		arc=0mm,
		colback=white,
		colframe=yellow,
		title=Caution,
		fonttitle=\sffamily,
		breakable
		]}{\end{tcolorbox}}

\date{}
\title{\flushleft \textbf{Compactness}}
\begin{document}
	\maketitle
	
	\begin{definition}[Compact]
		Let $(X,d)$ be a metric space and let $K \subseteq X$. $K$ is said to be compact if every open cover of $K$ has a finite subcover. That is, if $\{O_\alpha\}_{\alpha \in \Lambda}$ is any open cover of $K$, then
		$$\exists \alpha_1, ..., \alpha_n \text{ such that } K \subseteq O_{\alpha_1} \cup ... \cup O_{\alpha_n} $$
	\end{definition}
	
	\begin{example}
		Let $(X,d)$ be a metric space and let $E \subseteq X$. \\
		If $E$ is finite, then $E$ is compact.
	\end{example}
	\begin{proof}
	The reason is as follows:\\
	Let $\{O_\alpha\}_{\alpha \in \Lambda}$ be any open cover of $E$. Our goal is to show that this open cover has a finite subcover. \\
	If $E= \emptyset$, there is nothing to prove. \\
	If $E \not = \emptyset$, denote the elements of $E$ by $x_1,...x_n:$ $$E=\{x_1,...,x_n\}$$.
	We have:
	\begin{align*}
		x_1 \in E \subseteq \bigcup \limits_{\alpha \in \Lambda} O_\alpha &\implies \exists \alpha_1 \in \Lambda \text { such that } x_1 \in O_{\alpha_1} \\
		x_2 \in E \subseteq \bigcup \limits_{\alpha \in \Lambda} O_\alpha &\implies \exists \alpha_2\in \Lambda \text { such that } x_2 \in O_{\alpha_2} \\
		\vdots \\
		x_n \in E \subseteq \bigcup \limits_{\alpha \in \Lambda} O_\alpha &\implies \exists \alpha_n\in \Lambda \text { such that } x_n \in O_{\alpha_n} 
	\end{align*}
	Hence,
	$$E = {x_1, ..., x_n} \subseteq O_{\alpha_1} \cup ... \cup O_{\alpha_n}$$
	So, $O_{\alpha_1},...,O_{\alpha_n}$ is a finite subcover of $E$. \qed \\
	\end{proof}
	
	\begin{example}
		Consider $( \mathbb{R} , | |)$ and let $E=\{ \frac{1}{n} : n \in \mathbb{N} \} \cup \{0\}$. \\
		Prove that $E$ is compact. (In general, if $a_n \rightarrow a$ in $\mathbb{R}$ then $F=\{a_n : n \in \mathbb{N} \} \cup \{a\}$ is compact.)
	\end{example}
	\begin{proof}
		\label{prf:proof_3}
		Let $\{O_\alpha\}_{alpha\in \Lambda}$ be any open cover of $E$. Our goal is to show that this open cover has a finite subcover.
		\begin{align*}
			\begin{rcases}
				0 \in E \\
				E \subseteq \bigcup \limits_{\alpha \in \Lambda} O_\alpha
			\end{rcases}
			&\implies 0\in \bigcup \limits_{\alpha \in \Lambda} O_\alpha \implies \exists \alpha_0 \in \Lambda \text{ such that } 0 \in O_{\alpha_0}  \tag{$I$} \\
			\begin{rcases}
				0 \in O_{\alpha_0} \\
				O_{\alpha_0} \text{ is open }
			\end{rcases}
			&\implies \exists \epsilon > 0 \text{ such that } (-\epsilon, \epsilon) \subseteq O_{\alpha_0}
		\end{align*}
		By the archimedean property of $\mathbb{R}$,
		$$ \exists m \in \mathbb{N} \text{ such that } \frac{1}{n} < \epsilon$$
		so
		$$\forall n \geq m ~~~ \frac{1}{n} < \epsilon.$$
		Hence
		\begin{equation}
			\forall n \geq m ~~~ \frac{1}{n} \in (-\epsilon, \epsilon) \subseteq O_{\alpha_0} \tag{$II$}
		\end{equation}
		Notice that $E=\{0, \frac{1}{1}, \frac{1}{2},\frac{1}{3},...,\frac{1}{m-1}, \frac{1}{m},\frac{1}{m+1},\frac{1}{m+2},... \} \text{ for } m\in\mathbb{N}.$ All that remains is to find a subcover for the elements $\frac{1}{1}, ..., \frac{1}{m-1}:$ \\
		\begin{align*}
			1 \in E &\implies \exists \alpha_1 \in \Lambda \text{ such that } 1 \in O_{\alpha_1} \\
			\frac{1}{2} \in E &\implies \exists \alpha_2 \in \Lambda \text { such that } \frac{1}{2} \in O_{\alpha_2} \\
			\vdots \\
			\frac{1}{m-1} \in E &\implies \exists \alpha_{m-1} \in \Lambda \text{ such that } \frac{1}{m-1} \in O_{\alpha_{m-1}} \tag{$III$}
		\end{align*}
		By ($I$), ($II$), and ($III$), we have $$E \subseteq O_{\alpha_0} \cup ... \cup O_{\alpha_{m-1}}$$
		Thus, $\{O_\alpha\}_{\alpha\in \Lambda}$ has a finite subcover. Therefore $E$ is compact. \qed
	\end{proof}
	
	\begin{remark}
		If $X$ itself is compact, we say $(X,d)$ is a compact metric space.
		If $\ocover{O}$ is any collection of open sets such that $X= \unioncollect{O},$ then
		$$\exists \alpha_{1}, ..., \alpha_n \in \Lambda \st X = O_{\alpha_1} \cup ... \cup O_{\alpha_n}.$$
	\end{remark}
	
	\begin{theorem}
		Compact subsets of metric spaces are closed.
	\end{theorem}
	\begin{proof}
		\routineMS and let $K\subseteq X$ be compact. We want to show that $K$ is closed. It is enough to show that $K^c$ is open. To this end, we need to show that every point of $K^c$ is an interior point. \\
		Let $a\in K^c$. Our goal is to show that $$\exists \epsilon > 0 \st \nbhd{\epsilon}{a} \subseteq K^c.$$
		That is, we want to show that $$\exists \epsilon > 0 \st \nbhd{\epsilon}{a} \cap K = \emptyset.$$
		We have
		\begin{align*}
			a \in K^c &\implies a \not \in K \\
			&\implies \forall x \in K ~~ d(x,a) > 0.
		\end{align*}
		For all $x\in K$, let $$\epsilon_x = \frac{1}{4}d(x,a).$$
		Clearlly, $$\forall x \in K ~~ \nbhd{\epsilon_x}{x} \cap \nbhd{\epsilon_x}{a} = \emptyset.$$
		Notice that $$\{\nbhd{\epsilon_x}{x}\}_{x\in K} \text { is an open cover of } K.$$
		Since $K$ is compact, there is a finite subcover $$\exists x_1,..., x_n \in K \st K \subseteq \nbhd{\epsilon_{x_1}}{x_1} \cup ... \cup \nbhd{\epsilon_{x_n}}{x_n}$$
		and of course
		$$\begin{cases}
			\nbhd{\epsilon_{x_1}}{x_1}\cap \nbhd{\epsilon_{x_n}}{a} = \emptyset \\
			\vdots \\
			\nbhd{\epsilon_{x_n}}{x_n}\cap \nbhd{\epsilon_{x_n}}{a} = \emptyset
		\end{cases}$$
		Let $\epsilon=\min\{\epsilon_{x_1}, ..., \epsilon_{x_n}\}$. Clearly,
		$$\nbhd{\epsilon}{a} \subseteq \nbhd{\epsilon_{x_i}}{a} ~~ \forall 1 \leq i \leq n.$$
		Hence
		$$\begin{cases}
			\nbhd{\epsilon_{x_1}}{x_1} \cap \nbhd{\epsilon}{a} = \emptyset \\
			\vdots \\
			\nbhd{\epsilon_{x_n}}{x_n} \cap \nbhd{\epsilon}{a} = \emptyset
		\end{cases}$$
		Therefore
		$$
		\nbhd{\epsilon}{a} \cap [ \nbhd{\epsilon_{x_1}}{x_1}\cup ... \cup \nbhd{\epsilon_{x_n}}{x_n}] =\emptyset.
		$$
		So,
		$$
			\nbhd{\epsilon}{a} \cap K = \emptyset.
		$$
		\qed
	\end{proof}
	
	\begin{note}
			So, it has been shown that compact $\implies$ closed and bounded $\checkmark$. However, it is not necessarily the case that closed and bounded $\implies$ compact.
	\end{note}

\end{document}