\begin{definition} [Closed Set] \leavevmode \\
    Let $(X, d)$ be a metric space and let $E \subseteq X$.
    $E$ is said to be closed if it contains all of its limit points.
    $$E \text{ is closed } \iff E' \subseteq E.$$
\end{definition}

\begin{example}
    $(\mathbb{R}, d), ~~d(x,y) = |x-y|, ~~E = {1,2,3}$
    \begin{enumerate}[$(1)$]
        \item What is $E'$? \\
        Claim: $E' = \emptyset$.
        Let $p \in \mathbb{R}$. Our goal is to show that $p \not \in E'$. That is, we want to show $$\exists \epsilon > 0 \text{ such that } N_\epsilon (p) \cap (E \backslash \{p\}) = \emptyset.$$We may consider the following cases:
        \begin{description} 
            \item[Case 1: $p < 1$] \leavevmode \\ $\epsilon = \frac{1-p}{2}$ works.
            \item[Case 2: $p > 3$] \leavevmode \\ $\epsilon = \frac{p-3}{2}$ works.
            \item[Case 3: $p \in \{1, 2, 3\}$] \leavevmode \\ $epsilon = \frac{1}{4}$ works.
            \item[Case 4: $1 < p < 2$] \leavevmode \\ $\epsilon = \frac{1}{2}\text{min}\{p-1, 2-p\}$ works.
            \item[Case 5: $2 < p < 3$] \leavevmode \\ $\epsilon = \frac{1}{2} \text{min} \{p-2, 3-p\}$ works.
        \end{description}
        \item Since $E' = \emptyset$, we have $E' \subseteq E$ and so $E$ is closed.
    \end{enumerate}
\end{example}

\begin{definition} [Interior Point] \leavevmode \\
    Let $(X, d)$ be a metric space and let $E\subseteq X$.
    \begin{enumerate}[$(i)$]
        \item A point $p\in E$ is said to be an interior point of $E$ if
        $$\exists \delta > 0 \text{ such that } N_\delta (p) \subseteq E.$$
        \item The collection of all interior points of $E$ is called the interior of $E$ and is denoted by $\overset{\circ}{E}$ or $E^{\circ}$ or $int(E).$
    \end{enumerate}
\end{definition}

\begin{note} \leavevmode \\
    \begin{enumerate}[$(*)$]
        \item By definition we know that $\overset{\circ}{E} \subseteq E.$
        \item $p \in \overset{\circ}{E} \iff \exists N_\delta (p) \text{ such that } N_\delta(p) \subseteq E.$
    \end{enumerate}
\end{note}

\begin{example}
    $(\mathbb{R}, d), ~~ d(x,y) = |x - y|, ~~ E = (1,3].$ What is $\overset{\circ}{E}$?
\end{example}
\begin{proof}
    Claim: $\overset{\circ}{E} = (1,3).$ We need to show:
    \begin{enumerate}[$(1)$]
        \item Let $p \in (1,3)$, then $p \in \overset{\circ}{E}$
        \item If $p = 3,$ then $p \not \in \overset{\circ}{E}$
    \end{enumerate}
    \begin{description}
        \item[Proof of 1: ]  It is enough to show that $\exists \delta > 0$ such that $N_\delta (p) \subseteq E$. Clearly, $\delta = \frac{1}{2} \text{min}\{p-1, 3-p\}$ does the job.
        \item[Proof of 2: ] It is enough to show that $\forall \epsilon > 0 ~N_\epsilon (3) \not \subseteq E.$ That is, we want to show that $$\forall \epsilon > 0 ~~(3 - \epsilon, 3 + \epsilon) \cap E^c \not = \emptyset.$$
        Clearly, for all $\epsilon > 0$,
        $$\begin{cases} 3 + \frac{\epsilon}{2} \in (3 - \epsilon, 3+ \epsilon) \\ 3 + \frac{\epsilon}{2} \in E^c \end{cases} \implies (3-\epsilon, 3 + \epsilon)\cap E^c \not = \emptyset$$
    \end{description}
    \qed
\end{proof}

\begin{example}
    $(\mathbb{R}, d), ~~d(x,y) = |x-y|, ~~ E = \{1,2,3\}.$ Find $\overset{\circ}{E}.$
\end{example}

\begin{proof}
    Claim: $\overset{\circ}{E} = \emptyset$.
    The reason is as follows: \leavevmode \\
    Let $p \in \{1,2,3\}. ~\forall \epsilon > 0 ~N_\epsilon (p) = (p- \epsilon, p + \epsilon)\not \subseteq E.$ Hence $p \not \in \overset{\circ}{E}.$ We proved that if $p \in E$, then $p \not \in \overset{\circ}{E}.$
    \qed
\end{proof}

\begin{definition} [Open Set] \leavevmode \\
    Let $(X, d)$ be a metric space and let $E\subseteq X$.
    $E$ is said to be open if every point of $E$ is an interior point of $E$. That is,
    $$E \text{ is open } \iff E\subseteq \overset{\circ}{E}.$$
\end{definition}

\begin{note}
    We know that, for any set, $\overset{\circ}{E} \subseteq E$, so
    $$E \text{ is open } \iff E = \overset{\circ}{E}.$$
\end{note}

\begin{example}
    $(\mathbb{R}, d), ~~ d(x,y) = |x - y|, ~~E=\{1,2,3\}$. Is $E$ open?
\end{example}

\begin{proof}
    $\overset{\circ}{E} = \emptyset,$ so $\overset{\circ}{E} \not = E,$ so $E$ is NOT open.
    \qed
\end{proof}

\begin{example}
    $(\mathbb{R}, d), ~~ d(x,y) = |x - y|, ~~E = (1,4).$ Prove that $E$ is open.
\end{example}

\begin{proof}
    It is enough to show that every point $p \in E$ is an interior point. Let $p \in E,$ we need to show that $$\exists \delta > 0 \text { such that } N_\delta (p) \subseteq E.$$Clearly, $\delta = \frac{1}{2}\text{min}\{p-1, 4-p\}$ works.
    \qed
\end{proof}

\begin{definition} [Bounded Set]
    Let $(X,d)$ be a metric space and let $E\subseteq X$.
    $E$ is said to be bounded if
    $$\exists q\in X ~\exists \epsilon > 0 \text{ such that } E \subseteq N_\epsilon (q)$$
\end{definition}

\begin{example}
    $(\mathbb{R}, d), ~~d(x,y) = |x- y|, ~~E= [0, \infty).$ Is $E$ bounded?
\end{example}

\begin{proof}
    No, because
    $$\forall q \in \mathbb{R} ~\forall \epsilon > 0 ~~ [0, \infty) \not \subseteq (q - \epsilon, q + \epsilon).$$
    \qed
\end{proof}

\begin{example}
    $(\mathbb{R}, d), ~~d(x,y) = \begin{cases}1 & x\not = y \\ 0 &x=y\end{cases}, ~~E = [0, \infty)$. Is $E$ bounded?
\end{example}

\begin{proof}
    Yes. For example,
    $$E \subseteq N_{10}(0) = \mathbb{R}.$$
    \qed
\end{proof}

\begin{definition}[Closure] \leavevmode \\
    Let $(X, d)$ be a metric space and let $E\subseteq X.$
    The closure of $E$, denoted by $\overline{E}$, is defined as follows:
    $$\overline{E} = E \cup \overset{\circ}{E}.$$
\end{definition}

\begin{example}
    $(\mathbb{R}, d), ~~d(x,y) = |x-y|.$ What is $\overline{\mathbb{Q}}$?
    $$\overline{\mathbb{Q}} = \mathbb{Q} \cup \mathbb{Q'} = \mathbb{Q} \cup \mathbb{R} = \mathbb{R}$$
\end{example}

\begin{definition} [Dense] \leavevmode \\
    Let $(X, d)$ be a metric space and let $E \subseteq X.$
    We say that $E$ is dense in $X$ if
    $$\overline{E} = X.$$
    (That is, every point of $X$ is either in $E$ or is a limit point of $E$)
\end{definition}

\begin{example}
     $\overline{\mathbb{Q}} = \mathbb{R},$ so $\mathbb{Q}$ is dense in $\mathbb{R}.$
\end{example}