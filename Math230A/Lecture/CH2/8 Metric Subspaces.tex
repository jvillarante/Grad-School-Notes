\begin{remark}
    Suppose $(X,d)$ is a metric space and $Y$ is a nonempty subset of $X$. Clearly, $Y \times Y \subseteq X \times X.$ It is easy to show that $d |_{Y\times Y}$ satisfies all the defining properties of a metric space on the set $Y$. So, $\left(Y, d |_{Y\times Y}\right)$ is a metric space. This metric space is sometimes referred to as a metric subspace of $(X,d).$
\end{remark}

\begin{definition} \leavevmode \\
    Let $(X,d)$ be a metric space and let $Y$ be a nonempty subset of $X$. $\left(Y, d|_{Y\times Y}\right)$ is called a metric subspace of $(X,d)$.
    \begin{notation}
        Sometimes we write $\overset{Y}{d}$ instead of $d|_{Y\times Y}$.
        \begin{enumerate} [$*)$]
            \item $\forall y_1, y_2\in Y,$ $~\overset{Y}{d}(y_1, y_2) = d(y_1, y_2)$
            \item $\overset{X}{d} = d$
        \end{enumerate}
    \end{notation}
\end{definition}

\begin{remark}
    Consider the metric subspace $\left(Y, d|_{Y\times Y}\right)$ of $(X,d)$. Let $p\in Y, ~\epsilon > 0$.
    \begin{align*}
        \nbhds{\epsilon}{Y}{p} &= \{y \in Y : \overset{Y}{d}(y,p) < \epsilon\} \\
        &= \nbhd{\epsilon}{p}\cap Y
    \end{align*}
\end{remark}

\begin{example}
    $\left(\R, |~~|\right), ~Y = [0, \infty).$ \leavevmode \\
    $\nbhd{\epsilon}{0} = \{p \in \R : d(p,0) < \epsilon\} = (-\epsilon, \epsilon).$
    \begin{align*}
        \nbhds{\epsilon}{Y}{0} = \{p \in [0, \infty) : d(p, 0) < \epsilon\} &= \nbhd{\epsilon}{0}\cap Y \\
        &= (-\epsilon, \epsilon) \cap [0, \infty) \\
        &= [0, \epsilon)
    \end{align*}
\end{example}

\begin{remark}
    Let $(X,d)$ be a metric space nd let $Y \subseteq X.$ Suppose $E \subseteq Y.$
    $$\text{$E$ is open relative to $Y$} \iff \forall p \in E ~\exists \epsilon > 0 \st \nbhds{\epsilon}{Y}{p} \subseteq E.$$
    \begin{align*}
        \nbhds{\epsilon}{Y}{p} &= \{y \in Y : \overset{Y}{d}(y, p) < \epsilon\} \\
        &= \{y \in Y : d(y, p) < \epsilon\} \\
        &= \nbhd{\epsilon}{p} \cap Y.
    \end{align*}
\end{remark}

\begin{theorem} \leavevmode \\
    \label{thm2.30}
    \routineMS and let $E\subseteq Y \subseteq X$ given $Y \not = \emptyset$. Then
    $$\text{$E$ is open relative to $Y \iff \exists G_{\text{open}} \subseteq X \st E = G \cap Y$}$$
\end{theorem}

\begin{proof}
    $(\implies)$ Suppose $E$ is open relative to $Y$. We want to show there exists an open set $G_{\text{open}}\subseteq X \st E = G \cap Y$.
    \begin{align*}
        \text{$E$ is open relative to $Y$} &\implies \forall a \in E ~\exists \epsilon_a > 0 \st \nbhds{\epsilon_a}{Y}{a}\subseteq E \\
        &\implies \forall a \in E ~\exists \epsilon_a > 0 \st \nbhd{\epsilon_a}{a}\cap Y \subseteq E.
    \end{align*}
    Let $G=\bigcup \limits_{a\in E} \nbhd{\epsilon_a}{a}.$ Clearly, $G$ is open in $X$ because
    \begin{enumerate}[$(i)$]
        \item $\forall a \in E ~\nbhd{\epsilon_a}{a}$ is a neighborhood and so it is open in $X$
        \item A union of open sets is open
    \end{enumerate}
    In what follows, we'll prove that $E=G\cap Y:$
    \begin{enumerate}[$(I)$]
        \item $G\cap Y=\left(\bigcup \limits_{a\in E} \nbhd{\epsilon_a}{a}\right) \cap Y = \bigcup \limits_{a\in E} \left(\nbhd{\epsilon_a}{a} \cap Y\right) \subseteq \bigcup \limits_{a\in E} E = E$
        \item Suppose $b\in E$. We have 
        \begin{align*}
            &b\in \nbhd{\epsilon_b}{b} \implies b \in G \implies G \cap Y \\
            &b\in E \implies b \in Y
        \end{align*}
        Hence, $E\subseteq G \cap Y.$
    \end{enumerate}
    $(I),(II) \implies E = G \cap Y$.
    \\ \\
    $(\impliedby)$ Suppose there exists $G_{\text{open}} \subseteq X \st E = G \cap Y$. Our goal is to show that 
    $$\forall a \in E ~\exists \epsilon > 0 \st \nbhds{\epsilon}{Y}{a} \subseteq E.$$
    Let $a\in E$ be given. Our goal is to find $\epsilon > 0$ such that
    $$\nbhd{\epsilon}{a}\cap Y \subseteq E.$$
    We have
    $$a\in E = G \cap Y \implies a \in G \implies \exists \epsilon > 0 \st \nbhd{\epsilon}{a} \subseteq G.$$
    Therefore,
    $$\nbhd{\epsilon}{a}\cap Y \subseteq G \cap Y = E.$$
    \qed 
\end{proof}

\begin{theorem} \leavevmode \\
    \label{thm2.28}
    \begin{enumerate}[$(i)$]
        \item $E\subseteq \R$ is bounded above $\implies \sup E \in \overline{E}$
        \item $E\subseteq \R$ is bounded below $\implies \inf E \in \overline{E}$
    \end{enumerate}
\end{theorem}

\begin{proof}
    Here we will prove $(i)$. The proof of $(ii)$ is completely analogous. \\
    $E$ is bounded above $\implies \sup E$ exists and is a real number. Let $\alpha = \sup E$. Our goal is to show that $\alpha \in \overline{E}$. That is, we want to show that 
    $$\forall \epsilon > 0 ~\nbhd{\epsilon}{\alpha} \cap E \not = \emptyset.$$
    Let $\epsilon > 0$ be given.
    \begin{align*}
        \alpha = \sup E &\implies \exists x \in E \st \alpha - \epsilon < x \\
        &\implies \exists x \in E \st \alpha - \epsilon < x \leq \alpha < \alpha + \epsilon
    \end{align*}
    Hence, $\alpha - \epsilon < x < \alpha + \epsilon.$ That is, $x \in \nbhd{\epsilon}{\alpha}.$ Therefore,
    $$\nbhd{\epsilon}{\alpha} \cap E \not = \emptyset$$
    \qed
\end{proof}

\begin{definition} [Open Cover] \leavevmode \\
    \routineMS and let $E\subseteq X.$ A collection of sets $\ocover{O}$ is said to be an open cover of the set $E$ if 
    \begin{enumerate}[$(i)$]
        \item $\forall \alpha \in \Lambda ~~O_{\alpha}$ is open in $X$
        \item $E\subseteq \bigcup \limits_{\alpha \in \Lambda} O_{\alpha}$
    \end{enumerate}
\end{definition}

\begin{example}
    $(\R, |~~|), ~E = [0, \infty)$. \leavevmode \\
    The collection $\left\{E_n\right\}_{n\in \N}$ defined by $\forall n \in \N ~~E_n = \left(-\frac{1}{n}, n\right)$ is an open cover of $E$. The reason is as follows:
    \begin{enumerate} [$(i)$]
        \item In HW4, we proved that every open interval is an open set. So each $E_n$ is an open set.
        \item $E\subseteq \bigcup \limits_{n=1}^\infty E_n.$ Indeed, let $a\in E.$
    \end{enumerate}
    If $a = 0$, then $a \in E_n ~\forall n$. So, $a \in \bigcup \limits_{n=1}^\infty E_n.$ \\
    If $a \not = 0$, then $\frac{1}{a} > 0$ and so by the archimedean property of $\R$
    $$\exists n \in \N \st \frac{1}{n} < \frac{1}{a}$$
    So $a < n.$ Clearly $-\frac{1}{n} < 0 < a > n,$ hence $a \in E_n = \left(-\frac{1}{n}, n\right).$ Consequently
    $$a \in \bigcup \limits_{m=1}^\infty E_m$$
\end{example}