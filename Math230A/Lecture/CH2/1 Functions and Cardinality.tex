\begin{example}
    Show that there is no rational number whose square is 2.
\end{example}

\begin{theorem} \leavevmode \\
    There is a unique positive real number satisfying $\alpha ^2 = 2$ (this is written as $\sqrt{2}$).
\end{theorem}

\begin{info}
    Idea for our proof:
    \begin{enumerate}[1.]
        \item Uniqueness: suppose there are two of them $\alpha_{1}$ and $\alpha_{2}$. Prove: 
        \begin{align*}
            &\alpha_1 < \alpha_2 \text{ leads to a contradiction} \\
            &\alpha_1 > \alpha_2 \text{ leads to a contradiction}
        \end{align*}
        \item Existence:
        Let $A=\{x\in\mathbb{R} : x > 0, x^2 <2\}.$ Show that $A\not \in \emptyset$ and $A$ is bounded above.
        Let $\alpha = \sup A$.
        Prove
        \begin{align*}
            &\alpha^2>2 \text{ leads to a contradiction}\\
            &\alpha^2<2 \text{ leads to a contradiction}
        \end{align*}
    \end{enumerate}
\end{info}

\begin{note}
    A similar argument can be used to prove that if $x>0$ and $m\in\mathbb{N},$ then there exists a unique positive real number $\alpha$ such that $\alpha^m=x.$ We write $\alpha=\sqrt[n]{x}$ of $\alpha = x^{1/m}$.
\end{note}

\begin{definition}[Function] \leavevmode \\
    Version 1: Let $A$ and $B$ be two sets. A function from $A$ to $B$, denoted by $f:A\rightarrow B,$ is a rule that assigns to each element $x\in A$ a unique element $f(x) \in B$. \\

    Version 2: Let $A$ and $B$ be two sets. A function from $A$ to $B$ is a triple $(f, A, B)$ where $f$ is a relation from $A$ to $B$ satisfying \begin{enumerate}[$(i)$]
        \item $\text{Dom} (f) = A$
        \item If $(x,y)\in f$ and $(x, z) \in f$, then $y=z$
    \end{enumerate}
    A is called the domain of $f$, $B$ is called the codomain of $f$.
\end{definition}

\begin{example}
    Let $A=\emptyset, B$ be any set. Clearly, $\emptyset = \emptyset \times B.$ So the only function from $A=\emptyset$ to $B$ is the empty function $$(f, \emptyset, B)$$
    The empty function is one-to-one. The empty function is onto only when $B=\emptyset$.
\end{example}

\begin{definition}[Image and Range, onto] \leavevmode \\
    Consider a function $f:A\to B$. Let $E \subseteq A$.
    \begin{align*}
        &f(E)=\{f(x) : x \in E\}=\{y\in B : y=f(x) \text{ for some } x\in E\} = \text{ the image of E under F} \\
        &f(A) = \text{ the collection of all the outputs of f = the Range of f.}
    \end{align*}
    If $f(A) = B,$ then we say $f$ is onto. 
\end{definition}

\begin{definition}[Pre-image] \leavevmode \\
    Consider a function $f:A\rightarrow B$. Let $D\subseteq B.$
    $$
        f^{-1}(D) = \text{ the pre-image of D under f = } \{ x\in A : f(x)\in D\}.
    $$
\end{definition}

\begin{definition}[One-to-one] \leavevmode \\
    Consider a function $f:A\rightarrow B. f$ is said to be one-to-one if any of the following equivalent statements hold:
    \begin{enumerate}[$(i)$]
        \item $\forall x_1, x_2 \in A,$ if $x_1 \not = x_2$ then $f(x_1) \not = f(x_2)$
        \item $\forall x_1, x_2 \in A,$ if $f(x_1) = f(x_2)$ then $x_1 = x_2$
        \item $\forall y \in B,$ the set $f^{-1}(\{y\})$ contains at most one element of $A$
    \end{enumerate}
\end{definition}

\begin{definition}[Bijection] \leavevmode \\
    Let $A$ and $B$ be two sets. We say that $A$ and $B$ have the same cardinal number, and we write $A\sim B$, if there is a function $f:A\rightarrow B$ that is both one-to-one and onto. We call this $f$ a bijection.
\end{definition}

\begin{example} \leavevmode
    $\{1,2,3\} \sim \{a, b, c\}$ \\
    Indeed the function $f:\{1,2,3\} \rightarrow \{a,b,c\}$ defined by
    \begin{align*}
        f=
        \begin{cases*}
            &f(1)=a \\
            &f(2)=b \\
            &f(3)=c
        \end{cases*}
    \end{align*}
    is a bijection.
\end{example}

\begin{example} \leavevmode
    $\mathbb{N} \sim \{2, 4, 6,...\}$ \\
    Indeed the function of $f:\mathbb{N} \to \{2, 4, 6,...\}$ defined by $f(n)=2n$ is a bijection.
    \begin{align*}
        &1 \mapsto 2 \\
        &2 \mapsto 4 \\
        &3 \mapsto 6\\
        &\vdots
    \end{align*}
\end{example}

\begin{example} \leavevmode
    $\mathbb{N} \sim \mathbb{Z}$ \\
    Indeed, $f: \mathbb{N} \rightarrow \mathbb{Z}$ defined by $f(n)= \begin{cases} \frac{n}{2}&\text{if n is even} \\ -\frac{n-1}{2}&\text{if n is odd} \end{cases}$  is a bijection.
    \begin{align*}
        &1 \mapsto 0 &&2 \mapsto 1 \\
        &3 \mapsto -1 &&4 \mapsto 2 \\
        &5 \mapsto -2 &&6 \mapsto 3 \\
        &\vdots
    \end{align*}
\end{example}

\begin{example} \leavevmode
    $(-\infty, \infty) \sim (0, \infty)$ \\
    Indeed, $f(x)=e^x$ is a bijection.
\end{example}

\begin{example} \leavevmode
    $(0, \infty) \sim (0, 1)$ \\
    Indeed the function $f:(0, \infty) \rightarrow (0, 1)$ defined by $f(x) = \frac{x}{x+1}$ is a bijection.
\end{example}

\begin{example} \leavevmode
    $[0,1) \sim (0, 1)$ (comprehensive question!) \\
    Indeed, $f:[0,1) \rightarrow (0,1)$ defined by $f(x)=
    \begin{cases}
        \frac{1}{2}&x=0 \\
        \frac{1}{n+1}&x=\frac{1}{n} \text{ for } n\geq 2 \\
        x &\text{otherwise}
    \end{cases}$ is a bijection.
\end{example}

\begin{remark}
    Note that
    \begin{enumerate}[$(i)$]
        \item $A \sim A$ ($\sim$ is reflexive)
        \item If $A\sim B$, then $B\sim A$ ($\sim$ is symmetric)
        \item If $A \sim B$ and $B \sim C,$ then $A\sim C$ ($\sim$ is transitive)
    \end{enumerate}
    $$\sim \text{ is an equivalence relation}$$
\end{remark}

\begin{notation}
    \begin{align*}
        &\mathbb{N}=\{1, 2, 3,...\} \\
        &\mathbb{N}_n=\{1, 2, ..., n\}
    \end{align*}
\end{notation}

\begin{definition}[Finite, Infinite]
    Let $A$ be any set.
    \begin{enumerate}[$(i)$]
        \item We say that $A$ is finite if $A=\emptyset$ or $A\sim \mathbb{N}_n$ for some natural number $n$.
        \begin{enumerate}
            \item When $A\sim \mathbb{N}_n$ we say $A$ has $n$ elements and we write $card(A)=n$.
            \item Also, we set $card\emptyset = 0.$
        \end{enumerate}
        \item The set $A$ is said to be infinite if it is not finite.
        \item The set $A$ is said to be countable if $A\sim \mathbb{N}$
        \item The set $A$ is said to be uncountable if it is neither countable nor finite.
        \item The set $A$ is said to be at most countable if it is either finite or countable.
    \end{enumerate}
\end{definition}