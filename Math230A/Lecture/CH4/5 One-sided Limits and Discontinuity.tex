\begin{definition}[Right-handed limit, Left-handed limit] \leavevmode \\
    Let $f: (a,b) \to (Y, \overset{\sim}{d})$.
    \begin{enumerate}[$(i)$]
        \item Let $a \leq c < b.$ We write 
        $$\lim \limits_{x \to c^+} f(x) = L$$
        if any of the following equivalent conditions hold:
        \begin{enumerate}
            \item For all sequences $(x_n)$ in $(c,b)$ satisfying $x_n \to c$ we have $f(x_n) \to L.$
            \item $\forall \epsilon > 0 ~\exists \delta > 0 \st \text{ if $c < x < c + \delta$ then $\overset{\sim}{d}(f(x), L) < \epsilon$.}$
        \end{enumerate}
        \item Let $a < c \leq b.$ We write 
        $$\lim \limits_{x \to c^-} f(x) = L$$
        if any of the following equivalent conditions hold:
        \begin{enumerate}
            \item For all sequences $(x_n)$ in $(a,c)$ satisfying $x_n \to c$ we have $f(x_n) \to L.$
            \item $\forall \epsilon > 0 ~\exists \delta > 0 \st \text{ if $c - \delta < x < c$ then $\overset{\sim}{d}(f(x), L) < \epsilon$.}$
        \end{enumerate}
    \end{enumerate}
\end{definition}

\begin{definition}[Classifications of Discontinuities] \leavevmode \\
    Let $f:(a,b) \to (Y, \overset{\sim}{d}).$ Let $c\in (a,b)$. Suppose $f$ is discontinuous at $c$.
    \begin{enumerate}[$(i)$]
        \item $f$ is said to be discontinuity of the first kind, or a simple discontinuity, at $c$ if $\lim \limits_{x\to c^-}f(x)$ and $\lim \limits_{x\to c^+}f(x)$ exist, but: 
        \begin{enumerate}
            \item $\lim \limits_{x \to c^-}f(x) = \lim \limits_{x \to c^+}f(x) \not = f(c)$ (removable discontinuity)
            \item $\lim \limits_{x \to c^-}f(x) \not = \lim \limits_{x \to c^+}f(x)$ (jump discontinuity)
        \end{enumerate}
        \item $f$ is said to be discontinuous of the second kind at $c$ if at least one of $\lim \limits_{x \to c^-}f(x)$ or $\lim \limits_{x\to c^+}f(x)$ does not exist.
    \end{enumerate}
\end{definition}

\begin{theorem} \leavevmode \\
    \label{thm4.29}
    Let $f:(a,b) \to \R$ be an increasing function. Then at every point $c \in (a,b),$ the one-sided limits exist, and 
    \begin{enumerate} [$(i)$]
        \item $\lim \limits_{x \to c^-}f(x) = \sup \limits{a < x < c}f(x) \leq f(c)$
        \item $\lim \limits_{x \to c^+}f(x) = \inf \limits{c < x < b}f(x) \geq f(c)$
        \item If $a < c < d < b$, then 
        $$\lim \limits_{x \to c^+}f(x) \leq \lim \limits_{x \to d^-}f(x)$$
    \end{enumerate}
    A similar statement holds for decreasing functions.
\end{theorem}

\begin{proof}
    \begin{enumerate}[$(i)$]
        \item First, note that since $f$ is increasing, $f(c)$ is an upper bound for $\{f(x) : a < x < c\}$. So, $A=\sup \{f(x) : a < x < c\}$ exists in $\R$. Moreover, $A \leq f(c).$ It remains to show that $A = \lim \limits_{x \to c^-} f(x).$ To this end, it is enough to show that
        $$
    \forall \epsilon > 0 ~\exists \delta > 0 \st \text{if $c-\delta < x < c$ then $|f(x) - A| < \epsilon $.}
        $$
        Let $\epsilon > 0$ be given. Our goal is to find $\delta > 0$ such that 
        \begin{equation*}
            \text{if $c - \delta < x < c$ then $|f(x) - A| < \epsilon $} \tag{$*$}
        \end{equation*}
        Since $A = \sup \{f(x) : a < x < c\}$, for this given $\epsilon$, $A - \epsilon$ is not an upper bound for $\{f(x): a < x <c\}$. Therefore,
        $$
        \exists r \in (a, c) \st A - \epsilon < f(r) \leq A
        $$
        We claim that we can use $c - r$ as the $\delta$ that satisfies $(*)$: if $c - \delta < x < c,$ then
        \begin{enumerate}[$(1)$]
            \item $c - \delta < x \implies r < x \implies f(r) \leq f(x) \implies A - \epsilon < f(x)$
            \item $a < x < c \implies f(x) \leq \sup \{f(t) : a < t < c\} = A < A + \epsilon.$
        \end{enumerate}
        
        \item Analogous to $(i)$.
        
        \item Since $f:(c,b) \to \R$ is increasing, by $(i)$ we have
        \begin{equation}
            \lim \limits_{x \to d^-} f(x) = \sup \{f(x) : c < x < d\}
            \tag{$I$}
        \end{equation}
        Since $f: (a, d) \to \R$ is increasing, by $(ii)$ we have
        \begin{equation}
            \lim \limits_{x \to c^+} f(x) = \inf \{f(x) : c < x < d\}
            \tag{$II$}
        \end{equation}
        $$
        (I), (II) \implies \lim \limits_{x \to c^+} f(x) \leq \lim \limits_{x \to d^-} f(x)
        $$
    \end{enumerate}
    \qed
\end{proof}

\begin{corollary}
    Let $f: (a,b) \to \R$ be a monotone function. Then $f$ will have no discontinuities of the second kind. As a matter of fact, $f$ can only have jump discontinuities.
\end{corollary}

\begin{theorem}[Monotone Functions have at most Countable Discontinuities] \leavevmode\\
    \label{Thm4.30}
    Let $f: (a,b) \to \R$ be a monotone function. Then the set of points at which $f$ is discontinuous is at most countable.
\end{theorem}

\begin{proof}
    We prove the case where $f$ is increasing. The proof for the case where $f$ is decreasing is completely analogous. Let $E$ be the set set of points where $f$ is discontinuous. In what follows, we will construct a one-to-one function $G: E \to \Q$. For each $c\in E,$ define $G(c)$ as follows:
    $$
    \lim \limits_{x \to c^-} f(x) < \lim \limits_{x \to c^+} f(x) \implies \exists r \in \Q \st \lim \limits_{x \to c^-} < f(r) < \lim \limits_{x \to c^+} f(x)
    $$
    Let $G(c) = r.$ Then $G$ is one-to-one. Indeed, if $c_1, c_2 \in E$ and $c_1 < c_2$, then 
    $$
    \lim \limits_{x \to c_1^-}f(x) < G(c_1) < \lim \limits_{x \to c_1^+} \leq \lim \limits_{x \to c_2^-} f(x) < G(c_2) < \lim \limits_{x \to c_2^+}
    $$
    so $G(c_1) \not = G(c_2)$. \qed
\end{proof}