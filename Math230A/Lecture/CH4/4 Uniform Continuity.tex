Uniform continuity will allow us to extend the domain of our function to the entire metric space.

\begin{theorem}[A Special Case of the Tietze Extension Theorem] \leavevmode \\
    Let $(X,d)$ be a metric space and let $A$ be a nonempty closed set in $X$. If $f:A\to \R$ is continuous, then $f$ has a continuous extension to all of $X$.
\end{theorem}

\begin{theorem} \leavevmode \\
    Let $(X,d)$ be a metric space and let $A$ be a nonempty set in $X$. If $f:A\to \R$ is uniformly continuous on $A$, then $f$ can be extended to a continuous function $\overline{f} : \overline{A} \to \R$.
\end{theorem}

\begin{definition} [Uniformly Continuous] \leavevmode \\
    Let $f: A \subseteq (X,d) \to (Y, \overset{\sim}{d})$ be a function. $f$ is said to be uniformly continuous on $A$ if 
    $$\forall \epsilon > 0 ~\exists \delta_\epsilon \st \forall x, c \in A \text{ if $d(x,c) < \delta_\epsilon$ then $\overset{\sim}{d}(f(x), f(c)) < \epsilon.$}$$
\end{definition}

\begin{note}
    What does it mean to say $f$ is not uniformly continuous on $A$?
    $$\exists \epsilon > 0 \st \forall \delta > 0 ~\exists x,c \in A \text{ satisfying $d(x,c) < \delta$ but $\overset{\sim}{d}(f(x),f(c)) \geq \epsilon$}$$
\end{note}

\begin{example}
    Prove that $f: \R \to \R$ defined by $f(x) = 2x+1$ is uniformly continuous on $\R$.
\end{example}

\begin{proof}
    Our goal is to show that 
    $$\forall \epsilon > 0 ~\exists \delta > 0 \st \forall x,c \in \R \text{ if $|x-c| < \delta$ then $|f(x) - f(c)| < \epsilon$.}$$
    Let $\epsilon > 0$ be given. Our goal is to find $\delta > 0$ such that 
    $$\forall x,c \in \R \text{ if $|x-c| < \delta$ then $|(2x+1) - (2c+1)| < \epsilon.$}$$
    Clearly, we can take $\delta = \epsilon / 2$ (or any positive number less than $\epsilon / 2$). \qed
\end{proof}

\begin{remark}
    It is a direct consequence of our definition of uniform continuity that if $f$ is uniformly continuous on a set $A$ and $\emptyset \not = B \subseteq A$, then $f$ is uniformly continuous on $B$.
\end{remark}

\begin{theorem} \leavevmode \\
    Let $f:A \subseteq (X,d) \to (Y, \overset{\sim}{d})$. If we can find a number $\epsilon_0 > 0$ and two sequences $(x_n)$ and $(c_n)$ in $A$ such that 
    \begin{equation*}
        d(x_n, c_n) \to 0 \text{ and } \forall n ~~\overset{\sim}{d}(f(x_n), f(c_n)) \geq \epsilon_0
        \tag{$*$}
    \end{equation*}
    then $f$ is not uniformly continuous on $A$.
\end{theorem}

\begin{proof}
    Recall that $f$ is not uniformly continuous if and only if 
    $$\exists \epsilon > 0 \st \forall \delta > 0 ~\exists x, c \in A \text{ satisfying $d(x,c) < \delta$ but $\overset{\sim}{d}(f(x), f(c)) \geq \epsilon$.}$$
If $(*)$ holds, then the above statement will hold with $\epsilon = \epsilon_0$. Indeed, given $\delta > 0$, $~~\exists N \st d(x_N, c_N) < \delta$ but $\overset{\sim}{d}(f(x_N), f(c_N)) \geq \epsilon.$ \qed
\end{proof}

\begin{example}
    Prove that $f(x) = x^2$ is not uniformly continuous on $\R$.
\end{example}

\begin{proof}
    Let $x_n = n, ~c_n = n + 1/n.$ We have 
    $$\lim \limits_{n\to \infty} |x_n - c_n| = \lim \limits_{n \to \infty}|-1/n| = 0.$$
    Also, for all $n$,
    \begin{align*}
        |f(x_n) - f(c_n)| &= |n^2 - (n + 1/n)^2| \\
        &= |n^2 - n^2 - 2 - 1 / n^2| \\
        &= |-(x + 1/n^2)| \\
        &= 2 + 1/n^2 \geq 2.
    \end{align*}
    Thus, $f(x) = x^2$ is not uniformly continuous on $\R$. \qed
\end{proof}

\begin{example}
    Prove that $f(x) = \sin \frac{1}{x}$ is not uniformly continuous on $(0,1)$.
\end{example}

\begin{proof}
    Use $x_n = \frac{1}{2n\pi}$ and $c_n = \frac{1}{2n\pi + pi/2}.$
    $$\begin{rcases*}
        \lim x_n = 0 \\
        \lim c_n = 0
    \end{rcases*}
    \implies \lim (x_n - c_n) = 0 \implies \lim |x_n - c_n| = 0.$$
    But for all $n$
    $$|f(x_n) - f(c_n)| = |\sin (2n\pi) - \sin (2n\pi + \pi / 2)| = |0-1| = 1$$
    so $f$ is not uniformly continuous. \qed
\end{proof}

\begin{theorem} \leavevmode \\
    \label{thm4.19}
    Let $f: A \subseteq (X,d) \to (Y, \overset{\sim}{d})$ be continuous and suppose $A$ is compact. Then $f$ is uniformly continuous.
\end{theorem}

\begin{proof}
    For the sake of contradiction, suppose $f$ is not uniformly continuous. Then 
    $$\exists \epsilon > 0 \st \forall \delta > 0 ~\exists x,c \in A \text{ satisfying $d(x,c) < \delta$ but $\overset{\sim}{d}(f(x), f(c)) \geq \epsilon$.}$$
    In particular,
    \begin{align*}
        &\delta = 1 &&\exists x_1, c_1 \in A \text{ satisfying $d(x_1, c_1) < 1$ but $\overset{\sim}{d}(f(x_1), f(c_1)) \geq \epsilon$} \\
        &\delta = \frac{1}{2} &&\exists x_2, c_2 \in A \text{ satisfying $d(x_2, c_2) < \frac{1}{2}$ but $\overset{\sim}{d}(f(x_2), f(c_2)) \geq \epsilon$} \\
        &\delta = \frac{1}{3} &&\exists x_3, c_3 \in A \text{ satisfying $d(x_3, c_3) < \frac{1}{3}$ but $\overset{\sim}{d}(f(x_3), f(c_3)) \geq \epsilon$} \\
        \vdots
    \end{align*}
    In this way, we will obtain two sequences $(x_n)$ and $(c_n)$ in $A$ such that 
    \begin{enumerate}[$(i)$]
        \item $0 \leq d(x_n, c_n) < \frac{1}{n} ~~\forall n$
        \item $\overset{\sim}{d}(f(x_n), f(c_n)) \geq \epsilon ~~ \forall n$
    \end{enumerate}
    We have
    $$\begin{rcases*}
        A \text{ is compact } \implies A \text{ is sequentially compact} \\
        (x_n) \text{ is a sequence in } A
    \end{rcases*}
    \implies (x_n) \text{ has a subsequence $(x_{n_k})$ that converges to a point in $A$}$$
    Let $x = \lim \limits_{k \to \infty}x_{n_k}.$ Let $(c_{n_k})$ be the corresponding subsequence of $(c_n)$. We have 
    $$0 \leq d(c_{n_k}, x) \leq d(c_{n_k}, x_{n_k}) + d(x_{n_k}, x)$$
    So, $\lim \limits_{k\to \infty} c_{n_k} = x.$ Therefore, $(x_{n_k})$ and $(c_{n_k})$ are two sequences in $A$ that converge to $x\in A$.
    \begin{align*}
        x_{n_k} \to x &\overset{\text{$f$ is cont.}}{\implies} f(x_{n_k}) \to f(x) \\
        c_{n_k} \to x &\implies f(c_{n_k}) \to f(x)
    \end{align*}
    So, $\exists N_0$ such that $\forall k > N_0$
    \begin{align*}
        \overset{\sim}{d}(f(x_{n_k}), f(x)) < \epsilon / 4 \\
        \overset{\sim}{d}(f(c_{n_k}), f(x)) < \epsilon / 4
    \end{align*}
    As a result, $\forall k > N_0$ we have 
    \begin{align*}
        \overset{\sim}{d}(f(x_{n_k}), f(c_{n_k})) &\leq \overset{\sim}{d}(f(x_{n_k}), f(x)) + \overset{\sim}{d}(f(x), f(c_{n_k})) \\
        &< \epsilon / 4 + \epsilon / 4 \\
        &< \epsilon.
    \end{align*}
    This contradicts $(ii)$. \qed
\end{proof}