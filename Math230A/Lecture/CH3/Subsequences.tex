\begin{definition} (Subsequences)
    \routineSeq Let $n_1 < n_2 < n_3 < ...$ be a strictly increasing sequence of natural numbers. Then $(x_{n_1}, x_{n_2}, x_{n_3},...)$ is called a subsequence of $(x_1, x_2, x_3,...)$, and is denoted by $(x_{n_k})$, where $k \in \N$ indexes the subsequence.
\end{definition}

\begin{example}
    Let $(x_n) = (1, \frac{1}{2}, \frac{1}{3}, \frac{1}{4},...)$.
    \begin{enumerate}[(i)]
        \item $(1, \frac{1}{3}, \frac{1}{5}, \frac{1}{7},...)$ is a subsequence.
        \item $(\frac{1}{100}, \frac{1}{1000}, \frac{1}{10000},...)$ is a subsequence.
        \item $(1, \frac{1}{5}, \frac{1}{3}, \frac{1}{7}, \frac{1}{2},...)$ is not a subsequence (we do not have $n_1 < n_2 < n_3 <...$).
    \end{enumerate}
\end{example}
\begin{remark}
    Suppose $(x_{n_1}, x_{n_2}, x_{n_3},...)$ is a subsequence of $(x_1, x_2, x_3, ...)$. Notice that $n_i \in \N$ and $n_1 < n_2 < n_3 < ...$ so
    \begin{enumerate}[(i)]
        \item $n_1 \geq 1$
        \item For each $k \geq 2,$ there are at least $k-1$ natural numbers, namely $n_1,...,n_{k-1}$, strictly less than $n_k$, so $n_k \geq k$.
    \end{enumerate}
\end{remark}

\begin{theorem}
    \routineSeq If $\lim \limits_{n \to \infty} x_n = x$, then every subsequence of $(x_n)$ converges to $x$.
\end{theorem}
\begin{proof}
    Let \subseq{x} be a subsequence of \seq{x}. Our goal is to show that $\lim \limits_{k \to \infty} x_{n_k} = x.$ That is, we want to show
    $$\forall \epsilon > 0 ~\exists N \in \N \st \forall k > N ~d(x_{n_k}, x) < \epsilon.$$
    Let $\epsilon > 0$ be given. Our goal is to find $N$ \st
    \begin{equation}
        \text{if $k > N$, then $d(x_{n_k}, x) < \epsilon$}
        \tag{$I$}
    \end{equation}
    Since $x_n \to x$, we have
    \begin{equation}
        \exists \hat{N} \st \forall n > \hat{N} ~d(x_n, x) < \epsilon
        \tag{$II$}
    \end{equation}
    We claim that this $\hat{N}$ can be used as the $N$ we are looking for. Indeed, if we let $N=\hat{N}$, then if $k > N$ we can conclude that $n_k \geq k > N$ and so, by $(II)$
    $$d(x_{n_k},x) < \epsilon$$ \qed
\end{proof}

\begin{corollary} (contrapositive)
    \begin{enumerate}[(i)]
        \item If a subsequence of $(x_n)$ does not converge to $x$, then $(x_n)$ does not converge to $x$.
        \item If $(x_n)$ has a pair of subsequences converging to different limits, then $(x_n)$ does not converge.
    \end{enumerate}
\end{corollary}

\begin{example}
    Let $x_n = (-1)^n$ in $\R$.
    \begin{enumerate}
        \item The subsequence $(x_1, x_3, x_5,...) = (-1, -1, -1,...)$ converges to $-1.$
        \item The subsequence$(x_2, x_4, x_6,...) = (1, 1, 1, ...)$ converges to $1$.
    \end{enumerate}
    By (i) and (ii), $(x_n)$ does not converge.
\end{example}

\begin{theorem}
    \routineSeq The subsequential limits of $(x_n)$ form a closed set in $X$.
\end{theorem}
\begin{proof}
    Let $E=\{b \in X : b \text{ is a limit of a subsequence of } x_n\}.$ Our goal is to show that $E' \subseteq E$. To this end, we pick an arbitrary element $a \in E'$ and we will prove that $a \in E$. That is, we will show that there is a subsequence of $(x_n)$ that converges to $a$. We may consider two cases:
    \begin{description}
        \item[Case 1: $\forall n \in \N ~x_n = a$.]
        In this case, $(x_n)$ and any subsequence of $(x_n)$ converges to $a$. So $a \in E.$
        \item[Case 2: $\exists n_1 \in \N \st x_{n_1} \not = a.$] Let $\delta = d(a, x_{n_1}) > 0.$ Since $a \in E', \nbhd{\frac{\delta}{2^2}}{a}\cap(E \backslash \{a\}) \not = \emptyset.$ So,
        $$\exists b \in E \backslash \{a\} \st d(b, a) < \frac{\delta}{2^2}.$$
        Since $b\in E$, $b$ is a limit of a subsequence of $(x_n)$, so
        $$\exists n_2 > n_1 \st d(x_{n_2}, b) < \frac{\delta}{2^2}.$$
        Now note that
        $$d(x_{n_2}, a) \leq d(x_{n_2}, b) + d(b,a) < \frac{\delta}{2^2} + \frac{\delta}{2^2} = \frac{\delta}{2}.$$
        Since $a \in E'$, $\nbhd{\frac{\delta}{2^3}}{a} \cap (E \backslash \{a\}) \not = \emptyset.$ So,
        $$\exists b \in E \backslash \{a\} \st d(b,a) < \frac{\delta}{2^3}.$$
        Since $b \in E$, $b$ is a limit of a subsequence of $(x_n)$, so
        $$\exists n_3 > n_2 \st ~d(x_{n_3}, b) < \frac{\delta}{2^3}.$$
        Now note that
        $$d(x_{n_3}, a) \leq d(x_{n_3},b) + d(b, a) < \frac{\delta}{2^3} + \frac{\delta}{2^3} = \frac{\delta}{2^2}.$$
        In this way, we obtain a subsequence $x_{n_1}, x_{n_2}, x_{n_3},...$ of $(x_n)$ \st
        $$\forall k \geq 2 ~~d(x_{n_k}, a) < \frac{\delta}{2^{k-1}}$$
        so, clearly, $x_{n_k} \to a.$ Hence, $a \in E.$ \qed
    \end{description}
\end{proof}

\begin{theorem} (Compactness $\implies$ Sequential Compactness)
    \label{thm3.6a}
    Let $(X,d)$ be a compact metric space. Then every sequence in $X$ has a convergent subsequence.
\end{theorem}

\begin{proof}
    Let $(x_n)$ be a sequence in the compact metric space $X$. Let $E=\{x_1, x_2,...\}.$ If $E$ is infinite, then there exists $x\in X$ and $n_1< n_2 < n_3<...$ \st
    $$x_{n_1} = x_{n_2} = x_{n_3} = ... = x.$$
    Clearly, the subsequence $(x_{n_1}, x_{n_2},...)$ converges to $x$. If $E$ is infinite, then since $X$ is compact, by Theorem 2.37, $E$ has a limit point $x\in X$. Since $x\in E'$,
    $$\forall \epsilon > 0 ~\nbhd{\epsilon}{x} \cap (E \backslash \{x\}) \text{ is infinite.}$$
    In particular,
    \begin{align*}
        &\text{for $\epsilon = 1, ~\exists n_1 \in \N \st d(x_{n_1}, x) < 1$} \\
        &\text{for $\epsilon = 2, ~\exists n_2 \in \N \st d(x_{n_2}, x) < \frac{1}{2}$} \\
        &\text{for $\epsilon = 3, ~\exists n_3 \in \N \st d(x_{n_3}, x) < \frac{1}{3}$} \\
        &\vdots \\
        &\text{for $\epsilon = m, ~\exists n_m \in \N \st d(x_{n_m}, x) < \frac{1}{m}$} \\
        &\vdots
    \end{align*}
    In this way, we obtain a subsequence $x_{n_1}, x_{n_2}, x_{n_3},...$ of $(x_n)$ that converges to $x$. \qed
\end{proof}

\begin{corollary} (Bolzano-Weierstrass)
    Every bounded sequence in $\R ^k$ has a convergent subsequence.
\end{corollary}
\begin{proof}
    Let $(x_n)$ be a bounded sequence in $\R ^k$.

    $$\implies \exists q \in \R ^k \text{ and } r > 0 \st \{x_1, x_2, x_3,...\} \subseteq \nbhd{r}{q}.$$

    Note that $\nbhd{r}{q}$ is bounded and so $\overline{\nbhd{r}{q}}$ is closed and bounded. So, $\overline{\nbhd{q}{r}}$ is a compact subset of $\R ^k$. So,
    $\closure{\nbhd{q}{r}}$ is a compact metric space and $(x_n)$ is a sequence in $\closure{\nbhd{q}{r}}$. By Theorem \ref{thm3.6a}, there exists a subsequence $(x_{n_k})$ of $(x_n)$ that converges in the metric space $\closure{\nbhd{r}{q}}$. Since the distance function in $\closure{\nbhd{r}{q}}$ is the same as the distance function in $\R ^k$, we can conclude that $(x_{n_k})$ converges in $\R^k$ as well. \qed
\end{proof}

Recall:
$$x_n \to x \iff \forall \epsilon > 0 ~\exists N \st \forall n > N ~d(x_n, x) < \epsilon.$$
This is useful \textit{IF} we know that a sequence converges. How do we first determine that a sequence converges? Perhaps, given a sequence $(x_n)$, we can determine convergence by comparing two consecutive terms:
$$\text{If }\forall \epsilon > 0 ~\exists N \st d(x_{n+1}, x_n) < \epsilon, \text{ then the sequence converges.}$$
Unfortunately, this will not do. Consider $\R: x_n = \sqrt{n}$ diverges (it is unbounded) yet 
$$x_{n+1} - x_n = \sqrt{n+1} - \sqrt{n} = \frac{n+1 - n}{\sqrt{n+1}+\sqrt{n}} = \frac{1}{\sqrt{n+1}+\sqrt{n}} \to 0.$$
Cauchy proposed that instead of comparing the distance between two consecutive terms, we compare the distance between \textit{any} two terms after a certain index:
$$\text{If }\forall \epsilon > 0 ~\exists N \st \forall n,m > N ~d(x_m, d_n) < \epsilon, \text{ then the sequence converges.}$$

\begin{definition} (Cauchy Sequence)
    \routineMS A sequence $(x_n)$ in $X$ is said to be a Cauchy Sequence if
    $$\forall \epsilon > 0 ~\exists N \st ~\forall n,m > N ~d(x_m,x_n) < \epsilon.$$
\end{definition}

\begin{theorem}(Convergent $\implies$ Cauchy)
    \label{thm3.11a}
    \routineSeq Then
    $$(x_n) \text{ converges} \implies (x_n) \text{ is a Cauchy sequence}$$
\end{theorem}

\begin{proof}
    Assume there exists $x\in X$ \st $x_n \to x$. Our goal is to show that
    \begin{equation}
        \forall \epsilon > 0 ~\exists N \st \forall n, m > N ~d(x_n, x_m) < \epsilon \tag{$I$}
    \end{equation}
    \begin{info}
    We want to make $d(x_n, x_m)$ less than $\epsilon$ using the fact that $d(x_n, x)$ and $d(x_m, x)$ can be made as small as we like for large enough $m$ and $n$. It would be great if we could bound $d(x_n, x_m)$ with a combination of $d(x_n, x)$ and $d(x_m, x)$. Note that
    $$d(x_n,x_m) \leq d(x_n, x) + d(x, x_m)$$
    so it is enough to make each piece on the RHS less than $\epsilon / 2$
    \end{info}
    We have
    $$x_n \to x \implies \exists \hat{N} \st \forall n > \hat{N} ~d(x_n, x) < \epsilon / 2.$$
    We claim that this $\hat{N}$ can be used as the $N$ that we were looking for. Indeed, if we let $N=\hat{N}$, $(I)$ will hold because $\forall n,m > \hat{N}$,
    \begin{align*}
        d(x_m,x_n) &\leq d(x_m, x) + d(x, x_n) \\
        &< \epsilon / 2 + \epsilon / 2 \\
        &= \epsilon,
    \end{align*}
    as desired. \qed
\end{proof}

\begin{remark}
    The converse in general is not true. Eg, consider $\Q$ as a subspace of $\R$. In $\Q$, it is not true that every Cauchy sequence is convergent. For example, let $(q_n)$ be a sequence in $\Q$ \st $q_n \to \sqrt{2}.$
    \begin{align*}
        q_n \to \sqrt{2} \text{ in } \R &\implies (q_n) \text{ is convergent in $\R$} \\
        &\implies (q_n) \text{ is Cauchy in $\R$} \\
        &\implies (q_n) \text{ is Cauchy in $\Q$}
    \end{align*}
    but $(q_n)$ does not converge in $Q$.
\end{remark}
It is desirable to define a metric space in which Cauchy sequences imply convergence.

\begin{definition} (Complete Metric Space)
    A metric space in which every Cauchy sequence is convergent is called a complete metric space.
\end{definition}