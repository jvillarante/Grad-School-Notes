\begin{theorem}[Cauchy Condensation Test]
    Assume $a_n \geq 0$ for all $n$, and $(a_n)$ is a decrasing sequence. Then
    $$\sum_{n=1}^{\infty}a_n \text{ converges } \iff \sum_{n=0}^{\infty}2^n a_{2^n} = a_1 + 2a_2 + 4a_4 + 8a_8 + 16a_{16} +... \text{ converges}$$
\end{theorem}
\begin{proof}\leavevmode
    Let $s_m = a_1 + ... + a_m$, $t_m = a_1 + 2a_2 + 4a_4 + ... + 2^{m-1}a_{2^{m-1}}.$ Note that
    \begin{align*}
        s_{2^k} &= a_1 + a_2 + (a_3 + a_4) + (a_5 + a_6 + a_7 + a_8) +... + (a_{2^{k-1}+1}+...+a_{2^k}) \\
        &\geq a_1 + a_2 + (a_4 + a_4) + (a_8 + a_8 + a_8 + a_8) + ... + (a_{2^k} + ... + a_{2^k}) \\
        &= a_1 + a_2 + 2a_4 + 4a_8 + ... + 2^{k-1}a_{2^k} \\
        &= a_1 + \frac{1}{2}[2a_2 + 4a_4 + 8a_8 + ... + 2^k a_{2^k}] \\
        &= a_1 + \frac{1}{2}[t_{k+1} - a_1] \\
        &=\frac{1}{2}a_1 + \frac{1}{2}t_{k+1} \\
        &\geq \frac{1}{2}t_{k+1}.
    \end{align*}
    So,
    $$s_{2^k} \geq \frac{1}{2}t_{k+1}.$$
    Similarly,
    \begin{align*}
        s_{2^{k+1}} &= a_1 + (a_2 + a_3) +(a_4 + a_5 + a_6 + a_7) +... + (a_{2^{k-1}}+...+a_{2^k -1}) \\
        &\leq a_1 + (a_2 + a_2) + (a_4 + a_4 + a_4 + a_4) + ... + (a_{2^{k-1}} + ... + a_{2^{k-1}}) \\
        &=a_1 + 2a_2 + 4a_4 + ... + 2^{k-1}a_{2^{k-1}} \\
        &=t_k.
    \end{align*}
    So,
    $$s_{2^k -1} \leq t_k.$$
    \\
    $(\impliedby)$ Suppose $\sum_{n=0}^{\infty}2^n a_{2^n}$ converges ($(t_m)$ converges). We want to show $\sum_{n=1}^{\infty}a_n$ converges ($(s_m)$ converges). Note that since $a_n \geq 0$, both $(s_m)$ and $(t_m)$ are increasing sequences. It follows from the MCT that in order to prove $(s_m)$ converges, it is enough to show that $(s_m)$ is bounded.
    \begin{align*}
        (t_m) \text{ converges } &\implies (t_m) \text{ is bounded} \\
        &\implies \exists R > 0 \st \forall m ~~t_m \leq R.
    \end{align*}
    In what follows we will show that $R$ is an upper bound for $(s_m)$ as well. Indeed, let $m\in \N$ be given. Choose $k$ large enough so that $m < 2^k -1.$ Then
    $$s_m \leq s_{2^k - 1} \leq t_k \leq R.$$
    So for all $m$, $0 \leq s_m \leq R.$ Hence $(s_m)$ is bounded.
    \\ \\
    $(\implies)$ Suppose $\sum_{n=1}^{\infty}a_n$ converges ($(s_m)$ converges). We want to show that $\sum_{n=0}^{\infty} 2^n a_{2^n}$ converges ($(t_m)$ converges). We will prove the contrapositive: we will show that if $(t_m)$ diverges, then $(s_m)$ diverges. Suppose $(t_m)$ diverges. Let $R > 0$ be given. We will show that there is a term in the nonnegative sequence $(s_m)$ that is larger than $R$.
    $$
    \begin{rcases*}
        (t_m) \text{ diverges } \\
        (t_m) \text{ is increasing }
    \end{rcases*}
    \overset{MCT}{\implies} (t_m) \text{ is not bounded above} \implies \exists k \st t_{k+1} > 2R
    $$
    Now we have
    $$s_{2^k} \geq \frac{1}{2} t_{k+1} > \frac{1}{2}(2R) = R.$$
    So, $(s_m)$ is unbounded. \qed
\end{proof}

\begin{example} \leavevmode P-Series \\
    Let $p > 0$. Then $\left(a_n = \frac{1}{n^p}\right)_{n \geq 1}$ is decreasing and nonnegative.
    $$\sum_{n=1}^{\infty}\frac{1}{n^p} \text{ converges } \iff p > 1$$
\end{example}
\begin{proof}
    \begin{align*}
        \sum_{n=1}^{\infty} \frac{1}{n^p} &\iff \sum_{n=0}^{\infty} 2^n \frac{1}{(2^n)^p} \text{ converges } \\
        &\iff \sum_{n=0}^{\infty} \frac{1}{2^{np-n}} \text{ converges } \\
        &\iff \sum_{n=0}^{\infty} \left(\frac{1}{2^{p-1}}\right)^n \text{ converges } \\
        &\iff \left|\frac{1}{2^{p-1}}\right| < 1 \\
        &\iff 1 < 2^{p-1} \\
        &\iff 0 < p-1 \\
        &\iff 1 < p
    \end{align*} \qed
\end{proof}

\begin{example}
    Let $p > 0$. $\left(a_n = \frac{1}{n(\ln n)^p}\right)_{n \geq 2}$ is a decreasing nonnegative sequence.
    $$\sum_{n=2}^{\infty} \frac{1}{n (\ln n)^p} \text{ converges } \iff p > 1.$$
\end{example}
\begin{proof}
    \begin{align*}
        \sum_{n=2}^{\infty} \frac{1}{n(\ln n)^p} \text{ converges } &\iff \sum_{n=1}^{\infty} \cancel{2^n} \frac{1}{\cancel{2^n} (\ln 2^n)^p} \text{ converges } \\
        &\iff \sum_{n=1}^{\infty} \frac{1}{(n \ln 2)^p} \text{ converges } \\
        &\iff \frac{1}{(\ln 2)^p} \sum_{n=1}^{\infty} \frac{1}{n^p} \text{ converges } \\
        &\iff p > 1.
    \end{align*}
    \qed
\end{proof}

\begin{theorem}[Comparison Test]
    \label{thm3.25}
    Assume there exists an integer $n_0$ \st $0 \leq a_n \leq b_n$ for all $n \geq n_0$:
    \begin{enumerate}[$(i)$]
        \item If $\sum_{n=1}^{\infty}b_n$ converges, then $\sum_{n=1}^{\infty}a_n$ converges
        \item If $\sum_{n=1}^{\infty}a_n$ diverges, then $\sum_{n=1}^{\infty}b_n$ diverges.
    \end{enumerate}
\end{theorem}
\begin{proof}
    $(ii)$ is the contrapositive of $(i)$; we only need to prove $(i)$. By the Cauchy Criterion for Convergence of Series, it is enough to show that
    $$\forall \epsilon > 0 ~\exists N \st \forall n > m > N ~~~\left|\sum_{k=m+1}^{\infty}a_k\right| < \epsilon.
    $$
    Let $\epsilon > 0$ be given. Our goal is to find $N$ \st
    $$\text{if $n > m > N,$ then $\left|\sum_{k=m+1}^{\infty}a_k\right| < \epsilon$}$$
    Since $\sum_{n=1}^{\infty}b_n$ converges, it follows from the Cauchy criterion for series that 
    $$\exists \hat{N} \st \forall n > m > \hat{N} ~~\left|\sum_{k=m+1}^{\infty}b_k\right| < \epsilon.$$
    Let $N = \max \{n_0 , \hat{N}\}$. For $n > m > N$ we have
    $$\left|\sum_{k=m+1}^{\infty}a_k\right| = \sum_{k=m+1}^{\infty}a_k \leq \sum_{k=m+1}^{\infty}b_k = \left|\sum_{k=m+1}^{\infty}b_k\right| < \epsilon.$$
    \qed
\end{proof}

\begin{example} \leavevmode
    Does $\sum_{n=1}^{\infty}\frac{1}{n+5^n}$ converge?\\
    $\forall n\in \N,$
    $$\begin{rcases*}
        \frac{1}{n+5^n} \leq \frac{1}{5^n} \\
        \sum_{n=1}^{\infty} \frac{1}{5^n} \text{ converges (geometric series)}
    \end{rcases*}
    \implies \sum_{n=1}^{\infty} \frac{1}{n+5^n} \text{ converges}$$
\end{example}

\begin{example}
    Suppose $a_n \geq 0$ and $\sum_{n=1}^{\infty}a_n$ converges. Prove that $\sum_{n=1}^{\infty}a_n ^2$ converges.
\end{example}
\begin{proof}
    $$\sum_{n=1}^{\infty}a_n \text{ converges } \implies \lim a_n = 0 \implies \exists n_0 \forall n \geq n_0 ~~ 0 \leq a_n < 1 \implies \forall n \geq n_0 ~~ 0 \leq a_n ^2 \leq a_n$$
    It follows from the comparison test that $\sum_{n=1}^{\infty}a_n$ converges. \qed
\end{proof}

\begin{theorem}[Useful Theorem 1] \leavevmode \\
    \label{thm3.17}
    Let $(a_n)$ be a sequence of real numbers.
    \begin{enumerate}[$(i)$]
        \item Suppose $\beta \in \R$ is \st $\limsup a_n < \beta.$ Then
        $$\exists N \st \forall n > N ~~a_n < \beta$$
        \item Suppose $\alpha \in \R$ is \st $\liminf a_n > \alpha.$ Then
        $$\exists N \st \forall n > N ~~ a_n > \alpha$$
    \end{enumerate}
\end{theorem}
\begin{proof}
    Here we will prove $(i)$.
    Since $\limsup a_n < \beta$, clearly, $\limsup a_n \not = \infty.$ We may consider two cases:
    \begin{description}
        \item[Case 1: $\limsup a_n = - \infty$] \leavevmode \\
        Since $\liminf a_n \leq \limsup a_n$, we conclude that $\liminf an = -\infty$. Therefore, $\lim a_n = -\infty$. The claim follows directly from the definition of $a_n \to -\infty$.
        \item[Case 2: $\limsup a_n \in \R$] \leavevmode \\
        Let $A = \limsup a_n$ and let $r = \frac{\beta - A}{2}.$ Since $\lim_{n\to \infty} \sup \{a_k : k \geq n\} = A,$ there exists $N$ \st
        $$\forall n > N ~\sup \{a_k : k \geq n\} < A + r$$
        In particular,
        $$\forall n > N ~\sup \{a_k : k \geq n\} < \beta$$
        Therefore, $$\forall n > N ~~ a_n < \beta$$
    \end{description}
    \qed
\end{proof}

\begin{theorem}[Useful Theorem 2] \leavevmode \\
    Let $(a_n)$ be a sequence of real numbers.
    \begin{enumerate}[$(i)$]
        \item Suppose $\limsup a_n > \beta$. Then, for infinitely many $k$, $a_k > \beta$. That is,
        $$\forall n \in \N ~\exists k \geq n \st a_k > \beta.$$
        \item Suppose $\liminf a_n < \alpha$. Then, for infinitely many $k$, $a_k < \alpha$. That is,
        $$\forall n \in \N ~\exists k \geq n \st a_k < \alpha.$$
    \end{enumerate}
\end{theorem}

\begin{proof}
    Here we will prove $(i)$. Assume for contradiction that only for finitely many $k$, $a_k > \beta$. Then
    $$\exists N ~\forall k> N ~~a_k \leq \beta$$.
    Therefore
    $$\limsup a_k \leq \limsup \beta = \lim \beta = \beta$$
    which contradicts the assumption that $\limsup a_k > \beta.$\qed
\end{proof}

\begin{theorem} [Root Test] \leavevmode \\
    \label{thm3.33}
    Let $(a_n)$ be a sequence of real numbers. Let $\alpha = \limsup \sqrt[n]{|a_n|}$.
    \begin{enumerate}[$(i)$]
        \item If $\alpha < 1,$ then $\sum_{n=1}^{\infty}a_n$ is absolutely convergent.
        \item If $\alpha > 1,$ then $\sum_{n=1}^{\infty}a_n$ diverges.
    \end{enumerate}    
\end{theorem}
\begin{proof} \leavevmode
    $(i)$ Choose a number $\beta \st \alpha < \beta < 1.$ We have
    $$\limsup \sqrt[n]{|a_n|} < \beta \overset{\text{Useful Theorem 1}}{\implies} \exists N \st \forall n > N ~\sqrt[n]{|a_n|} < \beta$$
    Hence,
    $$
    \begin{rcases*}
        \forall n > N ~~0 \leq |a_n| < \beta ^n \\
        \sum_{n=1}^{\infty}\beta ^n \text{ converges (geometric series)}
    \end{rcases*}
    \overset{\text{comparison test}}{\implies} \sum_{n=1}^{\infty}\sqrt[n]{|a_n|} \text{ converges.}$$
    \\
    $(ii)$ Choose a number $\beta \st 1 < \beta < \alpha$. We have $\beta < \limsup \sqrt[n]{|a_n|}.$ By the Useful Theorem 2, we have $\beta < \limsup \sqrt[n]{|a_n|}.$ By the Useful Theorem 2
    \begin{align*}
        &\forall n \in \N ~\exists k \geq n \st \sqrt[k]{|a_k|} > \beta \\
        &\implies \forall n \in \N ~\exists k \geq n \st |a_k| > \beta ^k \\
        &\implies \forall n \in \N ~\exists k \geq n \st \sup \{|a_m| : m \geq n\} > \beta ^ k \\
        &\implies \forall n \in \N ~\sup \{|a_m| : m \geq n\} > \beta ^n.
    \end{align*}
    Since $\lim_{n \to \infty}\beta ^n = \infty$ ($\beta > 1$), it follows from the OLT in $\Rbar$ that $\lim_{n\to \infty} \sup\{|a_m| : m \geq n\} = \infty$. So, $\limsup|a_n| = \infty.$ This tells us that $\lim a_n \not = 0$. So, $\sum a_n $ diverges by the Divergence Test. \qed
\end{proof}

\begin{theorem} [Ratio Test] \leavevmode \\
    \label{thm3.34}
    Let $(a_n)$ be a sequence of real numbers.
    \begin{enumerate}[$(i)$]
        \item If $\limsup \left|\frac{a_{n+1}}{a_n}\right| < 1,$ then $\sum_{n=1}^{\infty}a_n$ converges absolutely.
        \item If $\left|\frac{a_{n+1}}{a_n}\right| \geq 1$ for all $n \geq n_0$, then $\sum_{n=1}^{\infty}a_n$ diverges.
        \item If $\liminf \left|\frac{a_{n+1}}{a_n}\right| > 1,$ then $\sum_{n=1}^{\infty} a_n$ diverges.
    \end{enumerate}
\end{theorem}
\begin{proof} \leavevmode
    Let $\lim_{n \to \infty} \left|\frac{a_{n+1}}{a_n}\right| = \rho.$ \\ 
    $(i)$ Choose a number $\beta \st \rho < \beta < 1$. We have
    $$\lim_{n \to \infty} \left|\frac{a_{n+1}}{a_n}\right| = \rho \implies \exists N \st \forall n \geq N ~~\left|\frac{a_{n+1}}{a_n}\right| < \beta$$
    So,
    \begin{align*}
        \left|a_{N+1}\right| &< \beta |a_N| \\
        \left|a_{N+2}\right| &< \beta |a_{N+1}| < \beta ^ 2 |a_N| \\
        \left|a_{N+3}\right| &< \beta |a_{N+2}| < \beta ^3 |a_N| \\
        &\vdots \\
    \end{align*}
    So $\forall n \in N$, $|a_{N+n}| < \beta ^n |a_N|.$ Now, notice that $\sum_{n=1}^{\infty} \beta ^n |a_N| = |a_N| \sum_{n=1}^{\infty} \beta ^n$ converges (geometric series). It follows from the comparison test that $\sum_{n=1}^{\infty}|a_{N+n}|$ converges. This immediately implies that $\sum_{n=1}^{\infty}|a_n|$ converges. \\
    $(ii)$ Choose a number $\beta \st 1 < \beta < \rho$.
    $$\lim_{n\to \infty}\left|\frac{a_{n+1}}{a_n}\right| = \rho \implies \exists N \st \forall n \geq N ~\left|\frac{a_{n+1}}{a_n}\right| > \beta.$$
    So,
    \begin{align*}
        |a_{n+1}| &> \beta |a_N| \\
        |a_{n+2}| &> \beta |a_{N+1}| > \beta^2 |a_N| \\
        |a_{n+3}| &> \beta |a_{N+2}| > \beta^3 |a_N| \\
        &\vdots
    \end{align*}
    Thus, $\forall n \in \N ~~|a_{N+n}| > \beta ^n |a_N|.$ Since $\beta > 1,$
    $$\lim_{n \to \infty} \beta ^n |a_N| = \infty.$$
    So, $\lim_{n\to \infty}|a_{N+n}| = \infty.$ Therefore, $\lim_{n\to \infty} a_n \not = 0.$ Thus $\lim_{n \to \infty}a_n \not = 0.$ So, $\sum_{n=1}^{\infty}a_n$ diverges by the divergence test.\qed
\end{proof}

\begin{example}
    Let $R \not = 0$ be a fixed number. Prove that the series $\sum_{n=1}^{\infty} \frac{R^n}{n!}$ converges.
    \begin{align*}
        \rho = \lim_{n\to \infty} \left|\frac{a_{n+1}}{a_n}\right| = \lim_{n \to \infty} \left|\frac{\frac{R^{n+1}}{(n+1)!}}{\frac{R^n}{n!}}\right| &= \lim_{n\to \infty} \left|\frac{R^{n+1}n!}{R^n (n+1)!}\right| \\
        &=\lim_{n \to \infty} \left|\frac{R}{n+1}\right| \\
        &= |R| \lim_{n\to \infty} \frac{1}{n+1} \\
        &= 0.
    \end{align*}
    $\rho = 0 < 1 \implies \sum_{n=1}^{\infty} \frac{R^n}{n!}$ is absolutely convergent.
\end{example}

\begin{theorem}[Dirichlet's Test] \leavevmode \\
    Consider Sequences $(a_n)$ and $(b_n)$ \st
    \begin{enumerate}[$(i)$]
        \item Partial sums of $\sum_{n=1}^{\infty}a_n$ are bounded
        \item $(b_n)$ is a decreasing sequence of nonnegative numbers: $b_1 \geq b_2 \geq b_3 \geq ... \geq 0$
        \item $\lim_{n \to \infty}b_n = 0$
    \end{enumerate}
    Then $\sum_{n=1}^{\infty}a_n b_n$ converges.
\end{theorem}

\begin{example}
    Consider the infinite sum
    $$1-1+\frac{1}{2} - \frac{1}{2} +\frac{1}{3} - \frac{1}{3} + \frac{1}{4} - \frac{1}{4} +...$$
    $(i)$ What is $(s_n)$?
    \begin{align*}
        s_1 &= 1 \\
        s_2 &= 1 - 1 = 0 \\
        s_3 &= 1 - 1 + \frac{1}{2} = \frac{1}{2} \\
        s_4 &= 1 - 1 + \frac{1}{2} - \frac{1}{2} = 0 \\
        s_5 &= 1 - 1 + \frac{1}{2} - \frac{1}{2} + \frac{1}{3} = \frac{1}{3} \\
        &\vdots \\
        s_{2k} = 0,& ~~s_{2k-1} = \frac{1}{k}
    \end{align*}

    $(ii)$ What is $\lim_{n \to \infty} s_n$?
    \begin{align*}
        &\lim_{k \to \infty} s_{2k} = 0 = \lim_{k \to \infty} s_{2k-1} \\
        \implies &\lim_{n \to \infty} s_n = 0.
    \end{align*}
\end{example}

\begin{remark}
    Consider the following rearrangement:
    $$1+\frac{1}{2} -1 + \frac{1}{3} + \frac{1}{4} - \frac{1}{2} + \frac{1}{5} + \frac{1}{6} - \frac{1}{3} +...$$
    Here is the corresponding sequence of partial sums:
    \begin{align*}
        s_1 &= 1 \\
        s_2 &= \frac{3}{2} \\
        s_3 &= \frac{1}{2} \\
        &\vdots \\
        s_{3\times 10^2 + 2} &\approx 0.6939 \\
        s_{3\times 10^4 + 2} &\approx 0.6932 \\
        s_{3\times 10^6 + 2} &\approx 0.6931 \\
        &\vdots
    \end{align*}
    It can be shown that $s_n \to \ln 2 \approx 0.6931$.
\end{remark}

\begin{theorem}
    If a series converges absolutely, then any rearrangement of the series converges to the same limit.
\end{theorem}

\begin{theorem}[Riemann Rearrangement Theorem]
    If a series $\sum_{n=1}^{a_n}$ converges conditionally, then for any $L \in \R$ there exists some rearrangement of $\sum_{n=1}^{\infty}a_n$ which converges to $L$.
\end{theorem}