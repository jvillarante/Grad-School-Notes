The main objective of Math 230 is to rigorously explore and generalize many of the concepts learned in calculus such as limits, continuity, sequence convergence, differentiation and integration. 

\begin{remark}
    We aim to uncover the fundamental structures that make calculus possible. For instance, we'll examine what kind of structure is required to discuss the convergence of a sequence within a given set.
    \begin{enumerate}
        \item $\R: ~~~1, \frac{1}{2}, \frac{1}{3}, \frac{1}{4}, ..., \to 0$
        \item $\{\text{blue, dog, pen}\}: ~~~ blue, dog, blue, dog, ... \to ?$
    \end{enumerate}
\end{remark}
\leavevmode \\
In particular, in Math 230A we will be studying:
\begin{enumerate}[1.]
    \item The real numbers
    \item Some aspects of set theory (finite sets, countable sets, ...)
    \item Metric spaces and their topological properties
    \item Sequences and series
    \item Continuous functions between metric spaces
\end{enumerate}
\leavevmode \\
What is a square? There are two ways to answer this question:
\begin{enumerate}[1.]
    \item Show how to construct a square
    \item Identify properties that distinguish a square from any other object in the universe. Defining properties: four sides of equal lengths, rights angles, ...
\end{enumerate}
For this class, we will use 2. to identify the real numbers.

\begin{remark}
    It is possible to 
    \begin{enumerate}
        \item Define the natural numbers $\N$ (equipped with addition and multiplication) using set theory.
        $$1:= \{\emptyset\}, 2:= \{\emptyset, \{\emptyset\}\}, 3:= \{\emptyset, \{\emptyset\}, \{\emptyset, \{\emptyset\}\}\}, ...$$
        \item Then proceed to define the integers $\Z$ as certain equivalence classes with respect to an equivalence relation on $\N \times \N$.
        \item Then define the field of rational numbers $\Q$ as the set of equivalence classes with respect to an equivalence relation on $\Z \times (\Z \backslash \{0\})$.
        \item Then ultimately the field of real numbers.
        \begin{align*}
            &\text{Approach 1: Equivalence classes of Cauchy sequences of rational numbers} \\
            &\text{Approach 2: Dedekind cuts}
        \end{align*}
    \end{enumerate}
\end{remark}

\begin{observation}
    The set $\R$ is not just a boring collection of elements; $\R$ is a set with equipped with extra structures:
    \begin{enumerate}
        \item In $\R$ we have addition and multiplication; addition and multiplication have certain properties.
        $$\text{$\R$ is a field (first defining property)}$$
        \item In $\R$ we have a notion of order.
        $$\text{$\R$ is an ordered field (second defining property)}$$
        \item In fact, $\R$ is the uniqued ordered field that has the least-upper-bound property (third defining property).
        \item In $\R$, there is a notion of length and a corresponding notion of distance.
        \begin{align*}
            &\text{$\R$ is a normed space} \\
            &\text{$\R$ is a metric space}
        \end{align*}
    \end{enumerate}
\end{observation}

\begin{description}
    \item[(I) Elaborating on the first defining property:] 
\end{description}

\begin{definition}[Field] \leavevmode \\
    A field is a set $\mathbb{F}$ with two operations called addition and multiplication, which satisfy the following field-axioms:
    \begin{align*}
        &(A) \text{Addition Axioms} \\
        &(A1) \forall x,y \in \F, ~~x+y \in \F \\
        &(A2) \forall x,y \in \F, ~~x+y = y+x \\
        &(A3) \forall x,y,z \in \F, ~~(x+y)+z = x+(y+z) \\
        &(A4) \F \text{ contains an element $0$ such that $\forall x \in \F, ~0+x = x$} \\
        &(A5) \text{If $x \in \F,$ then there exists an element $-x \in \F \st x + (-x) = 0$}
        \\ \\
        &(M) \text{Multiplication Axioms} \\
        &(M1) \forall x,y \in \F, ~~xy \in \F \\
        &(M2) \forall x,y \in \F, ~~xy = yx \\
        &(M3) \forall x,y,z \in \F ~~(xy)z = x(yz) \\
        &(M4) \F \text{ contains an element $1 \not = 0 \st \forall x \in \F ~x \cdot 1 = x$} \\
        &(M5) \text{If $x \in \F$ and $x \not = 0$, then there exists an element $\frac{1}{x} \in \F \st x \cdot \frac{1}{x} = 1$}
        \\ \\
        &(D) \text{Distributive Law} \\
        &~~~\forall x,y,z \in \F, ~x \cdot (y + z) = x \cdot y + x \cdot z
    \end{align*}
\end{definition}

\begin{description}
    \item[(II) Elaborating on the second defining property:] 
\end{description}
\begin{definition}[Ordered Field] \leavevmode \\
    An ordered field is a field $\F$ equipped with a relation, $<$, with the following properties:
    \begin{enumerate}[$(i)$]
        \item If $x \in \F$ and $y \in \F$, then one and only one of the following statements is true:
        $$x < y, ~y < x, ~x = y$$
        \item If $x,y,z\in \F$ and $x < y$ and $y < z$, then $x < z$.
        \item If $x,y,z \in \F$ and $y < z,$ then $x+y < x+z$.
        \item If $x,y \in \F$, and $0 < x$ and $0 < y$, then $0 < xy$.
    \end{enumerate}
\end{definition}

\begin{remark}
    $\Q$ (the set of rational numbers) is another example of an ordered field (and so, the properties we have discussed is not enough to uniquely identify $\R$).
\end{remark}

\begin{description}
    \item[(III) Elaborating on the third defining property:] 
\end{description}

\begin{definition}[Upper Bounds] \leavevmode \\
    Suppose $\F$ is an ordered field and let $A \subseteq \F.$ If there exists $\beta \in \F \st$
    $$\forall x \in A, ~x \leq \beta$$
    then $\beta$ is called an upper bound of $A$.

    \begin{enumerate}[$*)$]
        \item The collection of all upper bounds of $A$ is denoted by $UP(A)$
        \item If $UP(A) \not = \emptyset$, then we say $A$ is bounded above
    \end{enumerate}
\end{definition}

\begin{definition}[Lower Bounds] \leavevmode \\
    Suppose $\F$ is an ordered field and let $A \subseteq \F.$ If there exists $\alpha \in \F \st$
    $$\forall x \in A, ~\alpha \leq x$$
    then $\alpha$ is called a lower bound of $A$.
    
    \begin{enumerate}[$*)$]
        \item the collection of all the lower bounds of $A$ is denoted by $LO(A)$
        \item If $LO(A) \not = \emptyset,$ then we say $A$ is bounded below
    \end{enumerate}
\end{definition}

\begin{example}
    Let $A = [0,1).$
    $$UP(A) = [1, \infty), ~~LO(A) = (-\infty, 0]$$
\end{example}