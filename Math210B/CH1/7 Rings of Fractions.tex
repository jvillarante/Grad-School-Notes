We take a brief look at fractions. Fractions are equivalence cleases, as given any two fractions $a/b$ and $c/d$ we have 
$$\frac{a}{b}=\frac{c}{d} \iff da = cb$$
If we look at $R\times R$ where $R$ is a ring we can define fractions as ordered pairs under the relation
$$(a,b) = (c,d) \iff da = cb$$
What this allows us to do when $R$ is an integral domain is to create a field which contains an isomorphic copy of $R$. We will not spend much time here, so we end this chapter with the following remark.

\begin{remark}
    Every integral domain is contained in a field and the smalles field which contains an integral domain is called the field of fractions of the integral domain.
\end{remark}

In the next chapter, we will discuss different types of integral domains. More specifically, we will discuss the following:

\begin{enumerate}
    \item Euclidean domain (ED)
    \item Principal ideal domain (PID)
    \item Unique factorization domains (UFD)
\end{enumerate}