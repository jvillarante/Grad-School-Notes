\begin{proposition}
    Let $I$ and $J$ be ideals of a ring $R$. Define 
    $$I+J = \set{a+b : a \in I, b \in J}$$
    Then $I+J$ is the smalles ideal of $R$ that contains both $I$ and $J$.
\end{proposition}

\begin{proof}
    We know that $I$ and $J$ are normal additive subgroups of $R$, so $I+J$ is a subgroup of $R$.
    \begin{align*}
        a_1+b_1, a_2+b_2 \in I+J &\implies (a_1 + b_1)(a_2 + b_2) \\
        &= a_1a_2 + a_1b_2 b_1a_2 + b_1b_2
    \end{align*}
    Notice
    \begin{center}
        \begin{multicols}{2}
            \begin{align*}
                a_1a_2 \in I \\
                a_1 \in I \implies a_1b_2 \in I
            \end{align*} \\
            \begin{align*}
                b_1b_2 \in J \\
                b_1 \in J \implies b_1a_2 \in J
            \end{align*}
        \end{multicols}
    \end{center}
    since $I$ and $J$ are ideals. Hence, $(a_1+b_1)(a_2+b_2) \in I + J$ and so $I + J$ is an ideal of $R$. \\
    Futhermore, $I+J$ is the smalles ideal of $R$ which contains $I$ and $J$. We can verify that $I$ and $J$ are contained in $I+J$. Letting $b = 0$, we see any $a\in I$ is in $I+J$. Similarly, letting $a = 0$ we see any $b \in J$ is in $I+J$. Hence both $I$ and $J$ are contained in $I+J$. Let $K$ be an ideal of $R$ which contains both $I$ and $J$. Let $a+b \in I+J$. Then 
    \begin{align*}
        a\in I \text{ and } b \in J &\implies a \in K \text{ and } b \in K \\
        &\implies a+b \in K \\
        &\implies I+J \subseteq K
    \end{align*}
    \qed
\end{proof}

\begin{definition}[Finitely Generated Ideal] \leavevmode \\
    Let $R$ be a ring with identity $1$. Let $A$ be a subset of $R$. Let $(A)$ denote the smallest ideal of $R$ containing $A$. We can define $(A)$ more concretely by 
    $$(A) = \bigcap_{I \subseteq A}I, ~\text{$I$ is an ideal of $R$}$$
    An ideal genearted by a finite set is called a finitely generated ideal. If $A = \set{a_1, a_2, ..., a_n}$, then $(A)$ will be written as $(a_1, a_2, ..., a_n)$. \\
    $(I,J) = I\cup J$ (where $I$ and $J$ are subsets of $R$)
\end{definition}

What is the simplest group generated by a set? Cyclic groups that are generated with a single element. Given a ring $R$ with unity $1 \neq 0$ and some $a \in R$, consider $(a)$, the ideal generated by $a$. $(a)$ needs to contain $ra$ and $ar$ for all $r \in R$. What's more is $(a)$ must be closed with respect to addition so we need all finite sums of $ra$'s and $ar$'s 
$$(r_1a + r_2a + ar_3 + ar_4 + ...)$$
If we suppose that $R$ is commutative, then $(a)$ is very simple.

\begin{definition}[Principal Ideal] \leavevmode \\
    Let $R$ be a ring and $a \in R$. When $R$ is commutative with $1 \neq 0$,
    $$(a) = \set{ar : r \in R}$$
    and $(a)$ is called the principal ideal of $R$ generated by $a$.
\end{definition}

\begin{example}
    \begin{enumerate}[(i)]
        \item In any ring, $0 = (0)$ and when $R$ has unity $1$, $(1) = R$.
        \item Let $R$ be a commutatie ring with unity $1 \neq 0$.
        $$(a) + (b) = (a+b) \text{ (from what we proved about $I+J$)}$$
        \item All ideals are principal.
        \item All additive subgroups of $\Z$ are of the form $n\Z, n\geq 0$ and $n\Z$ is an ideal for all $n \geq 0$. Furthermore, $n\Z = \set{nx : x \in \Z} = (n)$.
    \end{enumerate}
\end{example}

\begin{proposition}
    Let $I$ be an ideal of $R$.
    \begin{enumerate}[(i)]
        \item $I = R$ if and only if $I$ contains a unit.
        \item Assume $R$ is commutative. $R$ is a field if and only if its only ideals are $0$ and $R$.
    \end{enumerate}
\end{proposition}

\begin{proof}
    \begin{enumerate}[(i)]
        \item If $I = R$, then $1 \in I$. Conversely, if $u$ is a unit in $I$ with inverse $v \in R$, then for any $r \in R$
        $$r = r\cdot 1 = r(vu) = (rv)u \in I$$
        Thus $R = I$.
        \item The ring $R$ is a field if and only if every nonzero element is a unit. Hence, any nonzero ideal contains a unit. Thus the only nonzero ideal of $R$ is $R$. Conversely, if $0$ and $R$ are the only ideals of $R$, we let $u$ be any nonzero element of $R$. By hypothesis, $(u) = R$ so $1 \in (u)$. That is, $1 = uv$ for some $v \in R$. That is, $u$ is a unit. Thus every nonzero element of $R$ is a unit so $R$ is a field.
    \end{enumerate}
    \qed
\end{proof}

Let's say $I$ is a nonzero ideal in $\Q$. Then there exists $a \in I \st a \neq 0$. Since ideals have the absorbative property, we know that $\frac{1}{a}\in \Q$ and $\frac{1}{a}\cdot a \in I$ which implies $1 \in I$. Then for any $b \in \Q$ we have $b = b\cdot 1 \in I$ which implies $\Q \subseteq I$. Thus $I = \Q$.