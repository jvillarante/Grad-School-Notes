Consider $f: I \to \R$ given by $f(x) = (x-x_0)^k.$ What do the $n$ derivatives look like? If $n =0$, then by the mean value theorem we have
$$
\begin{rcases*}
    f:I \to \R \text{ is differentiable} \\
    f(x_0) = 0
\end{rcases*} \implies f(x) = f'(c)(x-x_0)
$$

\begin{observation}
    Let $k$ be a natural number. Let $x_0$ be a fixed number.
    \begin{enumerate}[$*)$]
        \item $\ddx{(x-x_0)^k}{}=k(x-x_0)^{k-1}$
        \item $\ddx{(x-x_0)^k}{2} = \ddx{k(x-x_0)^{k-1}}{} = k(k-1)(x-x_0)^{k-2}$
        \item $\ddx{(x-x_0)^k}{3} = \ddx{k(k-1)(x-x_0)^{k-2}}{} = k(k-1)(k-2)(x-x_0)^{k-3}$
    \end{enumerate}
    \vdots
    \begin{enumerate}[$*)$]
        \item $\ddx{(x-x_0)^k}{k} = k(k-1)\dots(2)(1)(x-x_0)^{k-k}=k!$
    \end{enumerate}
    If $j < k$
    \begin{enumerate}[$*)$]
        \item $\ddx{(x-x_0)^k}{j}=k(k-1)\dots \left(k-(j-1)\right)(x-x_0)^{k-j}$
    \end{enumerate}
    Thus we have
    \begin{align*}
        \ddx{(x-x_0)^k}{j} &=
        \begin{cases*}
            k(k-1)\dots(k-j+1)(x-x_0)^{k-j} &if $j < k$ \\
            k! &if $j=k$ \\
            0 &if $j>k$
        \end{cases*} \\ 
        \ddx{(x-x_0)^k}{j}\Big|_{x=x_0} &=
        \begin{cases*}
            0 &if $j < k$ \\
            k! &if $j=k$ \\
            0 &if $j > k$
        \end{cases*}
    \end{align*}
\end{observation}

\begin{theorem}[Corollary of the General Mean Value Theorem] \leavevmode\\
    \label{Corollary of GMVT}
    Let $I\subseteq \R$ be an open interval, $x_0 \in I$, and $n \in \N \cup \{0\}.$ Let $f: I \to \R$ have $n+1$ derivatives. Suppose $f^{(k)}(x_0) = 0 ~~\forall 0 \leq k \leq n.$ Then for each point $x \not = x_0$ in the interval $I$, there exists a point $c_{x, x_0}$ strictly between $x$ and $x_0 \st$
    $$
    f(x) = \frac{f^{(n+1)}(c_{x,x_0})}{(n+1)!}(x-x_0)^{n+1}
    $$
\end{theorem}

\begin{proof}
    Here we will prove the claim for the case where $x > x_0$. The proof for $x < x_0$ is completely analogous. Let $g: I \to \R$ be defined by $g(t) = (t-x_0)^{n+1}$. Note that
    \begin{align*}
        g^{(k)}(x_0) = 0 ~~\forall 0 \leq k \leq n \\
        g^{(n+1)}(t)=(n+1)! ~~\forall t \in I
    \end{align*}
    Now, we apply the general mean value theorem to $f$ and $g$ on the interval $[x_0, x]$:

    \begin{align*}
        &\exists x_1 \in (x_0, x) \st \frac{f'(x_1)}{g'(x_1)} = \slope{f(x)}{f(x_0)}{g(x)}{g(x_0)} \\ &\implies
        \frac{f'(x_1)}{g'(x_1)} = \frac{f(x)}{g(x)}
        \tag{$I$}
    \end{align*}

    Next, we apply the general mean value theorem to $f'$ and $g'$ on the interval $[x_0, x_1]:$
    \begin{align*}
        &\exists x_2 \in (x_0, x_1) \st \frac{f''(x_2)}{g''(x_2)} = \slope{f'(x_1)}{f'(x_0)}{g'(x_1)}{g'(x_0)} \\ &\implies
        \frac{f''(x_2)}{g''(x_2)} = \frac{f'(x_1)}{g'(x_1)} \\
        &\overset{(I)}{\implies} \frac{f''(x_2)}{g''(x_2)} = \frac{f(x)}{g(x)}
    \end{align*}
    Continuing in this way, we will obtain $x_{n+1} \in (x_0, x) \st$
    $$
    \frac{f^{(n+1)}(x_{n+1})}{g^{(n+1)}(x_{n+1})} = \frac{f(x)}{g(x)}
    $$
    So,
    $$
    \frac{f^{(n+1)}(x_{n+1})}{(n+1)!} = \frac{f(x)}{(x-x_0)^{n+1}}
    $$
    Thus
    $$
    f(x) = \frac{f^{(n+1)}(x_{n+1})}{(n+1)!}(x-x_0)^{n+1}
    $$
    (We can use $x_{n+1}$ as the $c$ we were looking for)
    \qed
\end{proof}

\begin{description}
    \item[Question: ] What are the nicest functions that we know? Which functions are the easiest to work with?
    \item[Answer: ] Polynomials
    \item[General Question: ] Given a function $f$, is it possible to find a "good" approximation for $f$ among polynomials?   
    \item[Setup: ] \leavevmode\\
    \begin{enumerate}[$*)$]
        \item Let $I$ be a nonempty open interval in $\R$
        \item Let $n$ be a nonnegative integer
        \item Suppose $f:I \to \R$ has $n$ derivatives and $x_0 \in I$
        \item Suppose that we want to use the values
        $$
        f(x_0), f'(x_0), ..., f^{(n)}(x_0)
        $$
        to construct a polynomial approximation for $f$
    \end{enumerate}
\end{description}

What is the best we could hope for? Find a polynomial such that
\begin{align*}
    p(x_0) &= f(x_0) \\
    p'(x_0) &= f'(x_0) \\
    &\vdots \\
    p^{(n)}(x_0) &= f^{(n)}(x_0)
\end{align*}

\begin{observation}
    Let $x_0$ be a fixed real number. A general polynomial of degree at most $n$ can be expressed in powers of $(x-x_0)$ in the form
    $$
    p(x) = c_0 + c_1(x-x_0) + ... + c_n(x-x_0)^n
    $$
\end{observation}

\begin{example}
    Consider $p(x) = x^2-3x-1.$ Let $x_0 = 1$. We can express $p(x)$ in powers of $x-1$:
    \begin{align*}
        p(x) = x^2-3x-1 &= \left[(x-1)+1\right]^2-3\left[(x-1)+1\right]-1 \\
        &= (x-1)^2+2(x-1) + 1 -3(x-1) -3 -1 \\
        &= (x-1)^2 -(x-1) - 3
    \end{align*}
\end{example}

\begin{theorem}[Uniqueness of the Approximating Polynomial] \leavevmode\\
    Let $I \subseteq \R$ be an open interval and $n\in \N$. Suppose $f:I \to \R$ has $n$ derivatives ad $x_0 \in I.$ Then there exists a unique polynomial $p(x)$ of degree at most $n$ such that 
    $$
    \forall 0 \leq l \leq n ~~p^{(l)}(x_0) = f^{(l)}(x_0), \text{ with } \sum_{k=0}^{n} \frac{f^{(k)}(x_0)}{k!}(x-x_0)^k
    $$
\end{theorem}

\begin{proof}
    Let $p(x)$ be a general polynomial of degree at most $n$:
    $$
    p(x) = c_0 + c_1(x-x_0) + ... + c_n(x-x_0)^n
    $$
    Our goal is to show that
    $$
    \text{If $\forall 0 \leq l \leq n ~~p^{(l)}(x_0)=f^{(l)}(x_0)$ then $p(x) = \sum_{k=0}^{n}\frac{f^{(k)}(x_0)}{k!}(x-x_0)^k$}
    $$
    Note that $p(x_0)=c_0.$ Also, for $1 \leq l \leq n$ we have
    \begin{align*}
        p^{(l)}(x) &= \ddx{c_0 + \sum_{k=1}^{n}c_k (x-x_0)^k}{l} \\ 
        &= \ddx{\sum_{k=1}^{n}c_k(x-x)^k}{l} \\
        &= \sum_{k=1}^{n}c_k \ddx{(x-x_0)^k}{l}
    \end{align*}
    Hence,
    $$
    p^{(l)}(x_0) = \sum_{k=1}^{n}c_k \ddx{(x-x_0)^k}{l} \Big|_{x=x_0} = c_l \cdot l!
    $$
    Therefore,
    $$  
    \forall 1 \leq l \leq n ~~p^{(l)}(x_0) = c_l \cdot l!
    $$
    We conclude that
    \begin{align*}
        \text{$p$ agrees with $f$ to order $n$ at $x_0$} &\iff \begin{cases*}
            p(x_0) = f(x_0) \\
            p^{(l)}(x_0) = f^{(l)}(x_0) ~\forall 1 \leq l \leq n
        \end{cases*} \\
        &\iff \begin{cases*}
            c_0 = f(x_0) \\
            l! c_l = f^{(l)}(x_0) ~\forall 1 \leq l \leq n
        \end{cases*} \\ 
        &\iff \begin{cases*}
            c_0 = f(x_0) \\
            c_l = \frac{f^{(l)}(x_0)}{l!} ~\forall 1 \leq l \leq n
        \end{cases*} \\ 
        &\iff p(x) = \sum_{k=0}^{n}c_k(x-x_0)^k \\ 
        &= c_0 + \sum_{k=1}^{n} \frac{f^{(k)}(x_0)}{k!}(x-x_0)^k \\
        &= f(x_0) + \sum_{k=1}^{n}\frac{f^{(k)}(x_0)}{k!}(x-x_0)^k \\
        &= \sum_{k=0}^{n}\frac{f^{(k)}(x_0)}{k!}(x-x_0)^k
    \end{align*}
    \qed
\end{proof}

\begin{note}
$n^{\text{th}}$ Taylor Polynomial centered at $0$ is called $n^{\text{th}}$ Maclaurin Polynomial
\end{note}

\begin{description}
    \item[Big Lesson: ] There is exactly one polynomial of degree at most $n$ that satisfies
    \begin{align*}
        p(x_0) &= f(x_0) \\
        p'(x_0) &= f'(x_0) \\
        &\vdots \\
        p^{(n)}(x_0) &= f^{(n)}(x_0)
    \end{align*}
    This polynomial is called the $n^{\text{th}}$ Taylor polynomial for $f$ centered at $x_0$, and is given by
    $$
    T_{n, x_0}(x) = f(x_0) + f'(x_0)(x-x_0) + \frac{f''(x_0)}{2!}(x-x_0)^2 +...+ \frac{f^{(n)}(x_0)}{n!}(x-x_0)^n
    $$
\end{description}

\begin{theorem}[Taylor's Theorem with Lagarange Remainder] \leavevmode\\
    \label{Taylor's Thm}
    Let $I \subseteq \R$ be an open interval, $x_0\in I$, and $n \in \N\cup \{0\}$. Let $f: I \to \R$ have $n+1$ derivatives. Then  for each point $x \not = x_0$ in $I$, there is a point $c$ strictly between $x$ and $x_0$ such that
    $$
    f(x) = \sum_{k=0}^{n}\frac{f^{(k)}(x_0)}{n!}(x-x_0)^k + \frac{f^{(n+1)}(c)}{(n+1)!}(x-x_0)^{n+1}
    $$
    \begin{remark}
        Note that clearly the above equality holds at $x=x_0$ too (for any value of $c$). Recall that for any fixed number $R$, $\lims{n}{\infty}\frac{R^{n+1}}{(n+1)!} = 0$, however $f^{(n+1)}(c)$ may become very large.
    \end{remark}
\end{theorem}

\begin{proof}
    Let $F_{n, x_0}=f(x)-T_{n,x_0}(x).$ Our goal is to show that
    $$
    R_{n,x_0} = \frac{f^{(n+1)}(c)}{(n+1)!}(x-x_0)^{n+1}
    $$
    for some $c$ between $x$ and $x_0$. Note that
    \begin{enumerate}[$(i)$]
        \item $\begin{rcases*}
            \text{$f$ has $n+1$ derivatives}\\
            \text{$T_{n,x+0}$ is a polynomial of degree $n$, so it has $n+1$ derivatives} \\
            R_{n, x_0} = f - T
        \end{rcases*} \implies \text{$R_{n, x_0}$ has $n+1$ derivatives}$
        \item $\forall 0 \leq k \leq n ~~R_{n,x_0}^{(k)}(x_0) = f^{(k)}(x_0) - T_{n,x_0}^{(k)}(x_0) = 0$
    \end{enumerate}

    $(i), (ii), \text{Theorem $\ref{Corollary of GMVT}$} \implies \text{For each point $x \not = x_0$ in $I$, we have }$
    \begin{align*}
        R_{n,x_0}(x) = \frac{R_{n,x_0}^{(n+1)}(c)}{(n+1)!}(x-x_0)^{n+1} \text{ for some $c$ strictly between $x$ and $x_0$}
        \tag{$I$}
    \end{align*}

    Now note that
    \begin{equation*}
        R_{n,x_0}^{(n+1)}(c) = f^{(n+1)}(c) - T_{n, x_0}^{(n+1)}(c) = f^{(n+1)}(c)
        \tag{$II$}
    \end{equation*}
    $$
    (I), (II) \implies R_{n,x_0} = \frac{f^{(n+1)}(c)}{(n+1)!}(x-x_0)^{(n+1)}
    $$
    \qed
\end{proof}