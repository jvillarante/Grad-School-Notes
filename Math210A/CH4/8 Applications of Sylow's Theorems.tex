\begin{fact}
    Let $P \in \syl{p}{G}$The following are equivalent:
    \begin{enumerate}
        \item $n_p = 1$
        \item $P \nsubgroup G$
        \item $P$ is characteristic in $G$
        \item All subgroups generated by elements of $p$-power order are $p$-groups
    \end{enumerate}
\end{fact}

\begin{example}(Comprehensive Exam Spring 2024, problem 4) \leavevmode \\
    Let $|G| = 1225 = 5^2 \cdot 7^2.$ Prove that $G$ is abelian.
\end{example}

\begin{proof}
    By Sylow's theorems, we have
    \begin{align*}
        \begin{rcases*}
            n_7 \equiv 1 \mod 7 \\
            n_7 \mid 5^2
        \end{rcases*} &\implies n_7 \in \set{1, 5, 5^2} \implies n_7 = 1 \text{ (since neither $5$ nor $5^2$ are congruent to $1$ mod $7$)} \\
        \begin{rcases*}
            n_5 \equiv 1 \mod 5 \\
            n_5 \mid 7^2
        \end{rcases*} &\implies n_5 \in \set{1, 7, 7^2} \implies n_5 = 1 \text{ (since neither $7$ nor $7^2$ are congruent to $1$ mod $5$)}
    \end{align*}
    Thus there is only one Sylow $7$-subgroup, $P_7$, and one Sylow $5$-subgroup, $P_5$. Since $n_7 = 1$ and $n_5 = 1$, we have $P_7 \nsubgroup G$ and $P_5 \nsubgroup G$. Since $\gcd(|P_5|,|P_7|) = \gcd(5,7) = 1$, we have $P_5 \cap P_7 = 1$. So $P_5P_7 \leq G$ with order 
    $$|P_5P_7| = 5^2 \cdot 7^2 = |G| \implies P_5P_7 = G$$
    Since both subgroups are normal, we have
    $$G = P_5P_7 \cong P_5 \times P_7$$
    Since $P_5$ is of order $5^2$,
    $$P_5 \cong \Z_5 \times \Z_5 \text{ or } \Z_{25}$$
    Similarly,
    $$P_7 \cong \Z_7 \times \Z_7 \text{ or } \Z_{49}$$
    In each case, $P_5$ and $P_7$ are abelian. Hence, $G$ is abelian.
    \qed
\end{proof}

\begin{example} Let $|G| = 7 \cdot 11 \cdot 17$. Prove that $G$ is cyclic.
\end{example}

\begin{proof}
    By Sylow's theorems, we have
    \begin{align*}
        \begin{rcases*}
            n_7 \equiv 1 \mod 7 \\
            n_7 \mid 11 \cdot 17
        \end{rcases*} &\implies n_7 \in \set{1, 11, 17, 11 \cdot 17} \implies n_7 = 1 \text{ (since 1 is the only option that's congruent to $1$ mod $7$)} \\
        \begin{rcases*}
            n_{11} \equiv 1 \mod 11 \\
            n_{11} \mid 7 \cdot 17
        \end{rcases*} &\implies n_{11} \in \set{1, 7, 17, 7 \cdot 17} \implies n_{11} = 1 \text{ (since 1 is the only option that's congruent to $1$ mod $11$)} \\
        \begin{rcases*}
            n_{17} \equiv 1 \mod 17 \\
            n_{17} \mid 7 \cdot 11
        \end{rcases*} &\implies n_{17} \in \set{1, 7, 11, 7 \cdot 11} \implies n_{17} = 1 \text{ (since 1 is the only option that's congruent to $1$ mod $17$)}
    \end{align*}
    Thus 
    \begin{align*}
        n_7 = 1 &\implies \exists P_7 \in \syl{7}{G} \st P_7 \nsubgroup G \\
        n_{11} = 1 &\implies \exists P_{11} \in \syl{11}{G} \st P_{11} \nsubgroup G \\
        n_{17} = 1 &\implies \exists P_{17} \in \syl{17}{G} \st P_{17} \nsubgroup G
    \end{align*}
    Note that $\gcd(7, 11, 17) = 1$, so $P_7\cap P_{11} = 1, P_7\cap P_{17} = 1,$ and $P_{11}\cap P_{17} = 1$. Thus 
    $$P_7P_{11}P_{17} \leq G$$
    and
    $$|P_7P_{11}P_{17}| = 7\cdot 11 \cdot 17 = |G| \implies P_7P_{11}P_{17} = G$$
    Since $P_7, P_{11}, P_{17} \nsubgroup G$, we know 
    $$P_7P_{11}P_{17} \cong P_7 \times P_{11} \times P_{17} \cong \Z_7 \times \Z_{11} \times \Z_{17}$$
    Thus 
    $$G \cong \Z_7 \times \Z_{11} \times \Z_{17}$$
    so $G$ is cyclic.
    \qed
\end{proof}

\begin{example}
    Let $|G| = 3 \cdot 5 \cdot 7$. Prove $G$ is not simple.
\end{example}

\begin{proof}
    For the sake of contradiction, suppose $G$ is simple. By Sylow's theorems, we have
    $$n_3 \in \set{1, 7}, ~n_5 \in \set{1, 21}, ~n_7 \in \set{1, 15}$$
    If any one of these $n$'s were 1, then we'd get a normal subgroup, contradicting the assumption that $G$ is simple. Thus we have $n_3 = 7, n_5 = 21,$ and $n_7 = 15$. For each $p\in \set{3, 5, 7}$, the Sylow $p$-subgroup has prime order, and is therefore cyclic and each distinct $p$-subgroup is disjoint. Hence the number of elements of orders 3,5, and 7 are 
    $$(3-1) \cdot 7 = 14, ~(5-1) \cdot 21 = 84, ~(7-1)\cdot 15 = 90$$
    It follows that $|G| = 1 + 14 + 84 + 90 = 189$, a contradiction to $|G| = 105$. Therefore at least one of the Sylow subgroups is normal. Hence, $G$ is not simple.
    \qed
\end{proof}
\begin{example} (Comprehensive Exam Spring 2022, problem 4) \leavevmode \\
    Let $p$ be prime. Let $G$ be a finite group of order $m\cdot p^2$ with $m$ relatively prime to $p$. Suppose that $P$ and $Q$ are distinct Sylow $p$-subgroups of $G$ such that $|P\cap Q| \not = 1.$
    \begin{enumerate}[(a)]
        \item Prove that $N_G(P\cap Q)$ contains both $P$ and $Q$.
        \item Prove that $|N_G(P\cap Q)| = m'p^2$ for some $m' \geq p+1.$
    \end{enumerate}
\end{example}

\begin{proof}
    (a) Given $|P| = p^2 = |Q|,$ we have 
    \begin{align*}
        \begin{rcases*}
            |P\cap Q| \in \set{1, p, p^2} \\
            |P\cap Q| \not = 1 \\
            \text{$P$ and $Q$ are distinct}
        \end{rcases*} \implies |P\cap Q| = p
    \end{align*}
    Notice $P$ and $Q$ are abelian since $|P| = p^2 = |Q|$:
    $$N_G(P\cap Q) = \set{g \in G : g P\cap Q g^{-1} = P \cap Q}$$
    Our goal is to show $P \subseteq N_G(P\cap Q).$ That is, we want to show that 
    $$\forall x \in P ~~x P\cap Q x^{-1} = P$$
    Let $x \in P, h \in P\cap Q$. Then, since $P$ is abelian,
    $$xhx^{-1} = xx^{-1} = h \in H$$
    Thus 
    \begin{align*}
        &\forall x \in P ~~x P \cap Q x^{-1} = P\cap Q \\
        \implies &\forall x \in P ~~x \in N_G(P\cap Q) \\
        \implies &P \subseteq N_G(P\cap Q)
    \end{align*}
    The proof for $Q \subseteq N_G(P\cap Q)$ is analogous. Hence, $N_G(P\cap Q)$ contains both $P$ and $Q$. \\
    (b) By part (a), $N_G(P\cap Q)$ contains $P$ and $Q$. Thus 
    $$|P|=|Q| = p^2 \mid |N_G(P\cap Q)| \implies |N_G(P\cap Q)| = m'p^2 \text{ for some } \gcd(m', p) = 1$$
    It remains to show $m' \geq p+1.$ Let $N = N_G(P\cap Q)$. What is $n_p(N)?$
    \begin{align*}
        \begin{rcases*}
            n_p \equiv 1 \mod p \\
            n_p \mid m'
        \end{rcases*} \implies n_p \in \set{1, 1+p, 1+2p, 1+3p, ...}
    \end{align*}
    $P$ and $Q$ are Sylow $p$-subgroups, so $n_p \geq 2$. Thus 
    $$n_p \in \set{1+p, 1+2p, 1+3p, ...}$$
    Thus $n_p \geq 1+p$, and since $n_p \mid m'$, we have $m' \geq n_p \geq 1+p$ as desired.
    \qed
\end{proof}

\begin{example}Groups of order $pq$: \leavevmode \\
    \label{Sylowpq}
    Let $p, q$ be primes with $p < q$ and $p \nmid q -1$. Suppose $|G| = pq.$ Let $P \in \syl{p}{G}$ and $Q \in \syl{q}{G}$. Then
    \begin{align*}
        \begin{rcases*}
            n_p \equiv 1 \mod p \\
            n_p \mid q
        \end{rcases*} \implies n_p \in \set{1, q}
    \end{align*}
    If $n_p = q$, then $q \equiv 1 \mod p \implies p | q-1$, a contradiction to our assumptions. Thus $n_p = 1$ and $P \nsubgroup G.$
    \begin{align*}
        \begin{rcases*}
            n_q \equiv 1 \mod q \\
            n_q | p
        \end{rcases*} \implies n_q \in \set{1, p}
    \end{align*}

    Since $p < q$, we have have $p \not \equiv 1 \mod q \implies n_q \not = p.$ Thus $n_q = 1$ and $Q \nsubgroup G$. Since $P\cap Q = 1$ and $P, Q \nsubgroup g$,
    $$G = PQ \cong P \times Q \cong \Z_p \times \Z_q$$
    and since $\gcd(p, q) = 1$,
    $$G \cong Z_{pq}$$
    Thus groups of order $pq$ with $p < q$ and $p \nmid q-1$ are cyclic.
\end{example}

\begin{example}(Comprehensive Exam Fall 2020, problem 4) \leavevmode \\
    Let $G$ be a group of order $70$.
    \begin{enumerate}[(a)]
        \item Prove $G$ has a normal subgroup of order 35.
    \end{enumerate}
\end{example}

\begin{proof}
    We have
    \begin{align*}
        \begin{rcases*}
            n_5 \equiv 1 \mod 5 \\
            n_5 \mid 7\cdot 2
        \end{rcases*} &\implies n_5 = 1 \implies \exists P_5 \in \syl{5}{G} \st P_5 \nsubgroup G \\
        \begin{rcases*}
            n_7 \equiv 1 \mod 7 \\
            n_7 \mid 5 \cdot 2
        \end{rcases*} &\implies n_7 = 1 \implies \exists P_7 \in \syl{7}{G} \st P_7 \nsubgroup G
    \end{align*}
    Since $P_5, P_7 \nsubgroup G$, we have $P_5P_7 \leq G$. Also,
    $$\indx{G}{PQ} = 2 \implies PQ \nsubgroup G$$
    \qed
\end{proof}

\begin{example}
    Show that a group of order $315 = 3^2 \cdot 5 \cdot 7$ with a normal Sylow 3-subgroup has a Sylow 3-subgroup contained in the center of $G$.
\end{example}

\begin{proof}
    Since $G$ has a normal Sylow 3-subgroup, we know that $n_3 = 1.$ Applying Sylow's theorems, we have
    \begin{align*}
        n_5 \in \set{1, 21} \\
        n_7 \in \set{1, 15}
    \end{align*}
    If none of these $n$'s are 1, then adding up the elements of orders 3, 5, and 7 will give no useful information. \\
    Since $n_3 = 1$, there exists $P_3 \in \syl{3}{G} \st P_3 \nsubgroup G.$ We know that 
    $$\dfrac{N_G(P_3)}{C_G(P_3)} \cong K \leq \aut{P_3}$$
    Since $P_3 \nsubgroup G$, $N_G(P_3) = G$. This implies 
    $$\dfrac{G}{C_G(P_3)} \cong K \leq \aut{P_3}$$
    Then 
    \begin{align*}
        \left|\dfrac{G}{C_G(P_3)}\right| \mid \left|\aut{P_3}\right| &\implies \dfrac{|G|}{|C_G(P_3)|} \mid \left|\aut{P_3}\right| \\
        &\implies \dfrac{3^2 \cdot 5 \cdot 7}{|C_G(P_3)} \mid \left|\aut{P_3}\right|
    \end{align*}
    Recall that for a cyclic group of order $n$, the automorphism group has order $\phi(n)$. Thus if $P_3 \cong \Z_9$,
    $$\left|\aut{P_3}\right| = \phi(9) = 3(3-1) = 6$$
    In order for $3^2 \cdot 5 \cdot 7 / |C_G(P_3)|$ to divide 6, we must have that $3\cdot 5 \cdot 7 \mid |C_G(P_3)|$. We also know that $P_3 \leq C_G(P_3).$ It follows that $3^2$ divides $|C_G(P_3)|$. Hence $|C_G(P_3)| = 3^2 \cdot 5 \cdot 7 \implies C_G(P_3) = G \implies P_3 \leq Z(G)$. \\
    If $P_3 \cong \Z_3 \times \Z_3$, we need to use the following fact:
    \begin{fact}
        $|\aut{(\Z_p \times \Z_p)}| = p(p-1)^2(p+1) = (GL_2(\F_p))$
    \end{fact}
    We have been told that
    $$|\aut{(\Z_3 \times \Z_3)}| = 3 \cdot 2^2 \cdot 4 = 48$$
    Again, we have $9 \mid |C_G(P_3)|$ and $\frac{3^2 \cdot 5 \cdot 7}{|C_G(P_3)|} \mid 3 \cdot 2^4$ so 
    $$3\cdot 5 \cdot 7 \mid |C_G(P_3)| \implies |C_G(P_3)| = |G|$$
    \qed
\end{proof}

\begin{example}
    Let $|G| = 3 \cdot 5 \cdot 7 = 105$. Show that $G$ has a cylcic subgroup of order 35.
\end{example}

\begin{proof}
    By Sylow's theorems,
    $$n_5 \in \set{1, 21}, ~n_7 \in \set{1, 15}$$
    If $n_5 \not = 1$ and $n_7 \not = 1$, then there are 
    \begin{align*}
        \text{$4 \cdot 21 = 84$ elements of order $5$} \\
        \text{$6 \cdot 15 = 90$ elements of order $7$}
    \end{align*}
    So either $n_5 = 1$ or $n_7 = 1$. Thus 
    $$P_5P_7 < G$$
    as one (or both) of $P_5$ and $P_7$ are normal. Also, $\gcd(5, 7) = 1 \implies P_5 \cap P_7 = 1$. Thus 
    $$|P_5P_7| = \dfrac{|P_5|\cdot |P_7|}{|P_5\cap P_7|} = 35$$
    It remains to show that $P_5P_7$ is cyclic. \\
    Notice
    $$|P_5P_7| = 35 = 5 \cdot 7, ~5 < 7, ~5 \nmid 7 -1 = 6$$
    Thus by example \ref{Sylowpq}, we have that $P_5P_7$ is cyclic, completing the proof.
    \qed
\end{proof}