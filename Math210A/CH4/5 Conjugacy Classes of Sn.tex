Each conjugacy class in $S_n$ consists of all elements in $S_n$ with the same cycle type.
\newline Let's look at $S_5$. What are the conjugacy classes?
\[
\begin{array}{|c|c|c|}
    \hline
    \text{Elements with cycle type: } & \text{Representative: } & \text{Number of Elements in Each Class: }\\
    \hline
    1,1,1,1,1 & 1_{S_5} & 1 \\[6pt]
    1,1,1,2 & (1~2) & \dfrac{5 \cdot 4}{2}=10 \\ [8pt]
    1, 1, 3 & (1~2~3) & \dfrac{5 \cdot 4 \cdot 3}{3} = 20 \\ [8pt]
    1, 4 & (1~2~3~4) & \dfrac{5 \cdot 4 \cdot 3 \cdot 2}{4} = 30 \\ [8pt]
    5 & (1~2~3~4~5) & \dfrac{5 \cdot 4 \cdot 3 \cdot 2 \cdot 1}{5} = 24 \\ [8pt]
    1, 2, 2 & (1~2)(3~4) & \dfrac{\frac{5\cdot 4}{2} \cdot \frac{3 \cdot 2}{2}}{2} = 15 \\ [8pt]
    2, 3 & (1~2)(3~4~5) & \dfrac{5 \cdot 4}{2} \cdot \dfrac{3 \cdot 2 \cdot 1}{3} = 20 \\ [8pt]
    \hline
\end{array}
\]
We can use the number of elements in a conjugacy class to find the centralizer of an element in $S_5$.
\newline Let $\sigma = (1~2~3)$. We know the conjugacy class of $\sigma$, $K_\sigma$, has $\dfrac{5\cdot 4\cdot 3}{3} = 20$ elements. So, the orbit-stabilizer theorem says
\begin{align*}
    |K_\sigma| &= 20 \\
    &= |S_5 : C_{S_5}(\sigma)| \\
    &= \dfrac{|S_5|}{|C_{S_5}(\sigma)|} \\
    &= \dfrac{5!}{|C_{S_5}(\sigma)|} \\
    \implies |C_{S_5}(\sigma)| &= \dfrac{5!}{20} = 3! = 3 \cdot 2
 \end{align*}
 We know $\cyc{\sigma} = \set{1, \sigma, \sigma^2} \leq C_{S_5}(\sigma)$. We also know that disjoint cycles commute so $(4~5)$ commutes with $\sigma$. Thus 
 $$C_{S_5}(\sigma) = \set{1, \sigma, \sigma^2, (4~5), (4~5)\sigma, (4~5)\sigma^2}$$

 \begin{example}
    \begin{enumerate}
        \item Find $C_{S_5}\left((1~2~3~4~5)\right)$.
        \begin{align*}
            |K_\tau| = \dfrac{5!}{5} = 24 = \dfrac{|S_5|}{|C_{S_5}(\tau)|} \\
            \implies |C_{S_5}(\tau)| = \dfrac{5!}{4!} = 5 \\
            C_{S_5}(\tau) = \set{1, \tau \tau^2, \tau^3, \tau^4} = \cyc{\tau}
        \end{align*}
        \item Find $|C_{S_5}\left((1~2)(3~4)\right)|$.
        \begin{align*}
            |K_\alpha| = \dfrac{\frac{5 \cdot 4}{2} \cdot \frac{3 \cdot 2}{2}}{2} = 15 = \dfrac{|S_5|}{|C_{S_5}(\alpha)|} \\
            \implies |C_{S_5}(\alpha)| = \dfrac{5!}{5 \cdot 3} = 4\cdot 2 = 8
        \end{align*}
        \item Find $|C_{S_n}(\sigma)|$ where $\sigma$ is an $m$-cycle in $S_n$.
        \begin{align*}
            |K_\sigma| &= \dfrac{n(n-1)...(n-(m-1))}{m} \\ 
            &= \dfrac{n!}{m(n-m)!}
        \end{align*}
        So,
        \begin{align*}
            \dfrac{|S_n|}{|C_{S_n}(\sigma)|} &= \dfrac{n!}{m(n-m)!} \\
            \implies \dfrac{n!}{|C_{S_n}(\sigma)|} &= \dfrac{n!}{m(n-m)!} \\
            \implies |C_{S_n}(\sigma)| &= m(n-m)!
        \end{align*}
    \end{enumerate}
    Note that there are $m$ elements in $\cyc{\sigma}$ and there are $(n-m)!$ many elements of $S_n$ which fix $1, 2, 3, ..., n$, so $C_{S_n}(\sigma)$ consists of $\sigma^j\tau$ where $j = 0, ..., m-1$ and $\tau$ is a permutation that is disjoint from $\sigma$. Since there are $m(n-m)!$ elements of this form, they make up all elements of $C_{S_n}(\sigma).$
 \end{example}

 We will prove $A_5$ is simple, but we need a general fact.

 \begin{fact}
    \label{fact4.1}
    For any group $G$ and $H \nsubgroup G$ we have
    $$H = \bigcup_{h\in H}K_h$$
    That is, $H$ is the union of a collection of conjugacy classes.
 \end{fact}

 \begin{proof}
    Let $h \in H$. Then $h$ is in exactly one conjugacy class, namely $K_h.$ Hence 
    $$h \in \bigcup_{h \in H}K_h \implies H \subseteq \bigcup_{h \in H}K_h$$ Now, let $k \in K_h$ for some $h \in H$. Then $k = ghg^{-1}$ for some $g \in G.$ Since $H \nsubgroup G$, $k \in H.$ Thus 
    $$H = \bigcup_{h \in H}K_h$$
    \qed
 \end{proof}
 
 As a consequence of the above fact, if $|H| = n < \infty$ then 
 $$|H| = |K_1|+|K_2|+...+|K_j|$$
 where $\set{K_i}_{i=1}^{i=j}$ is the collection of conjugacy classes of $G$.

 \begin{proposition}
    $A_5$ is simple.
 \end{proposition}

 \begin{proof}
    We will find the order of all conjugacy classes and show they cannot add up to any divisor of $|A_5| = \dfrac{5!}{2} = 60$ except for $|A_5|$ and $1$. \\
    Consider the following representatives of conjugacy classes of $A_5$:
    \begin{align*}
        &1 \\
        &(1~2~3) = (1~3)(1~2) \\ 
        &(1~2~3~4~5) = (1~2)(1~3)(1~4)(1~5) \\
        &(1~2)(3~4)
    \end{align*}
    First, we find the size of each conjugacy class in $A_5$:
    \begin{align*}
        K_1 &= \set{1} \\
        |K_{(1~2~3)}| &= \dfrac{|A_5|}{|C_{A_5}(1~2~3)} = \dfrac{\frac{5!}{2}}{3} = 20 \\ 
        |K_{(1~2~3~4~5)}| &= \dfrac{\frac{5!}{2}}{5} = 12
    \end{align*}
    Notice there are $24$ $5$-cycles in $A_5$. Thus there must be a $5$-cycle, $\sigma$, such that $\sigma$ is not conjugate to $(1~2~3~4~5)$. The same exact same calculation gives 
    $$|K_\sigma| = 12$$
    Lastly, we know that 
    $$|K_{(1~2)(3~4)}| = \dfrac{|A_5|}{|C_{A_5}\left((1~2)(3~4)\right)|}$$
    Notice that $C_{A_5}\left((1~2)(3~4)\right)$ is a group and $C_{A_5}\left((1~2)(3~4)\right) \subseteq C_{S_5}\left((1~2)(3~4)\right)$, which implies its order divides $8$. But $8 \not | ~ |A_5| = 5 \cdot 4 \cdot 3,$ so 
    $$|C_{A_5}\left((1~2)(3~4)\right)| \in \set{1, 2, 4}$$
    Since $|(1~2)(3~4)| = 2$ and $(1~3)(2~4)$ commutes with $(1~2)(3~4)$ we have 
    $$|C_{A_5}\left((1~2)(3~4)\right)| = 4$$
    which implies all cycle types $1,2,2$ in $S_5$ are conjugate as $|K_{(1~2)(3~4)}| = 15.$ So, by fact \ref{fact4.1}, a normal subgroup of $A_5$ must have order of the sum of a collection of $1, 20, 12, 12, 5$, but non of these sums divide $|A_5|$ except for $1$ and $60$. Therefore, $A_5$ must be simple.
    \qed
 \end{proof}