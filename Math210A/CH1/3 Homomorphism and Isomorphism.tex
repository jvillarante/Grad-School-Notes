In general, we can tell how similar groups are by the mappings we make between them where the mappings preserve the group structure of the domain.

\begin{definition}[Homomorphism] \leavevmode \\
    Let $(G, \star)$ and $(H, \diamond)$ be groups. A map $\Phi: G \to H$ is a homomorphism if for all $g_1, g_2 \in G$,
    $$\Phi(g_1 \star g_2) = \Phi(g_1) \diamond \Phi(g_2)$$
    We usually write
    $$\Phi(xy) = \Phi(x) \Phi(y)$$
    and we know that $xy$ happens in $G$ and $\Phi(x)\Phi(y)$ happens in $H$.
\end{definition}

\begin{example}
    $\pi: \R^2 \to \R$ by $\pi(x,y) = x$ $\forall (x,y) \in \R^2$ is a homomorphism. Letting $(x_1, y_1), (x_2, y_2) \in \R^2$, we have
    \begin{align*}
        \pi((x_1, y_1) + (x_2, y_2)) &= \pi(x_1 + x_2, y_1 + y_2) \\
        &= x_1 + x_2 \\
        &= \pi(x_1, y_1) + \pi(x_2, y_2)
    \end{align*}
    Showing that $\pi$ is indeed a homomorphism.

    What elements are in the set $\set{p \in \R^2 : \pi(p) = 0} = K$?
    $$K = \set{(x,y) : x=0}$$
    This is the kernel of $\pi$.
\end{example}

\begin{definition}[Kernel] \leavevmode \\
    Let $G$ and $H$ be groups and let $\Phi: G \to H$ be a group homomorphism. The kernel of $\Phi$ is
    $$\ker(\Phi) = \set{g \in G : \Phi(g) = e_H} = \Phi^{-1}(e_H)$$
    where $e_H$ is the identity element in $H$.
\end{definition}

\begin{definition}[Isomorphism] \leavevmode \\
    Let $G$ and $H$ be groups. A map $\Psi: G \to H$ is an isomorphism if
    \begin{enumerate}
        \item $\Psi$ is a homomorphism
        \item $\Psi$ is bijective
    \end{enumerate}
    If there exists an isomorphism $\Psi: G \to H$, we say that $G$ and $H$ are isomorphic, denoted $G \cong H$.
    
    $\cong$ is an equivalence relation on any collection of groups.
\end{definition}

\begin{example}
    Let $k \in \Q^\ast = \Q \setminus \set{0}$. Define $\phi_k: \Q^\ast \to \Q^\ast$ by $\phi_k(q) = kq$. We claim that $\phi$ is an isomorphism.
    Show that $\Phi_k$ is a homomorphism and a bijection:
    \begin{enumerate}
        \item Homomorphism:
        \begin{align*}
            \phi_k(q_1 + q_2) &= k(q_1 +q_2) \\
            &= k (q_1+ q_2) \\
            &= kq_1 + kq_2 \\
            &= \phi_k(q_1) + \phi_k(q_2)
        \end{align*}
        \item Bijections:
        \begin{itemize}
            \item Injective: Suppose $\phi_k(q_1) = \phi_k(q_2)$. Then \begin{align*}
                \phi_k(q_1) &= \phi_k(q_2) \\ \iff
                kq_1 &= kq_2 \\ \iff
                q_1 &= q_2 &&(k \neq 0)
            \end{align*}
            \item Surjective: We want to show $\phi_k(\Q) = \Q$. Let $q \in \Q.$ Since $k \neq 0$, $\frac{q}{k} \in \Q$. Then
            $$\phi_k\left(\frac{q}{k}\right) = k \cdot \frac{q}{k} = q$$
            Thus $\phi_k$ is surjective.
        \end{itemize}
    \end{enumerate}
    ker$\phi_k=\set{0}$ since $\phi_k(q) = 0 \iff kq = 0 \iff q=0$.
\end{example}

\begin{fact}
    Suppose $G \cong H,$ that is there exists $\phi: G \to H$ which is a homomorphic bijection. Then
    \begin{enumerate}
        \item $|G|=|H|$
        \item $G$ is abelian if and only if $|H|$ is abelian
        \item $\forall x \in G ~~|x| = |\phi(x)|$ (Corresponding elements have the same order)
    \end{enumerate}
\end{fact}