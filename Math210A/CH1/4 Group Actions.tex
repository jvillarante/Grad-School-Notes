There are many examples of groups acting on sets. For instance, consider an element in $S_5$, call it $\sigma.$ $\sigma$ is a permutation of $\set{1,2,3,4,5}$ and it is also an element of a group
\begin{align*}
    \sigma &= (1~2~3~4~5) \\
    &\sigma(5) = 4
\end{align*}
We say that $\sigma$ is acting on the set $\set{1,2,3,4,5}$.

Consider the set of all $2\times 2$ matrices with elements in $\R$. Let $A = \begin{bmatrix}
1 & 2 \\
3 & 4
\end{bmatrix}$ and let $k \in \R.$ Then $kA = \begin{bmatrix}
k & 2k \\
3k & 4k
\end{bmatrix}.$ We say that $\R$ is acting on the set of all $2\times 2$ matrices with elements in $\R$.

\begin{definition}[Group Action] \leavevmode \\
    Let $G$ be a group and $A$ be a set. A group action of $G$ on $A$ is a map from $G\times A$ to $A$ (written $g.a ~~\forall g \in G, a \in A$) such that
    \begin{enumerate}
        \item $g_1.(g_2.a) = (g_1g_2).a ~~\forall g_1, g_2 \in G$ (Compatability)
        \item $1.a = a$ (or $e.a = a$) $ ~~\forall a \in A$ (Identity)
    \end{enumerate}
\end{definition}

\begin{example}
    Let $G = S_n$. Let's verify that $S_n$ acts on the set $\set{1,2,...,n}.$ Define the group action
    \begin{align*} \tag{$*$}
        \sigma.a = \sigma(a) ~~\forall \sigma \in S_n, a \in \set{1,2,...,n}
    \end{align*}
    Then let $\sigma_1, \sigma_2 \in S_n$ and $a \in \set{1,2,...,n}.$ We have
    \begin{align*}
        \sigma_1.(\sigma_2.a) &= \sigma_1.(\sigma_2(a)) \\
        &= \sigma_1(\sigma_2(a)) \\
        &= (\sigma_1\circ\sigma_2)(a) \\
        &= (\sigma_1\circ\sigma_2).a \tag{I}
    \end{align*}
    To verify the identity property, recall that the identity map, denoted $I$, is the identity of $S_n$ and $$I(a) = a ~~\forall a \in \set{1,2,...,n}$$
    That is,
    \begin{align*}
        I.a = I(a) = a ~~\forall a \in \set{1,2,...,n} \tag{II}
    \end{align*}
    By $(I)$ and $(II)$, $S_n$ acts on the set $\set{1,2,...,n}$ by the group action defined in $(*)$.
\end{example}

\begin{example}
    A vector space over a field $F$ is a set $V$ with two binary operations vector addition and scalar multiplication, and other poperties including 
    \begin{itemize}
        \item $a(bv) = (ab)v ~~\forall a, b \in F, v \in V$ (Compatability)
        \item $1v = v ~~\forall v \in V$ where $1$ is the multiplicative identity in $F$ (Identity)
    \end{itemize}
    Since $F$ is not a group with respect to multiplication, we must say that $F^\ast = F \setminus \set{0}$ acts on $V$.
\end{example}