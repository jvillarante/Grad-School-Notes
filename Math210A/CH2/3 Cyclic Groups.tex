\begin{definition}[Cyclic Group] \leavevmode\\
    A group $H$ is cyclic if $H$ is generated by a single element. That is,
    $$\exists x \in H \st H = \set{x^n : n \in \Z}$$
    $$\left(\exists x \in H \st H = \set{nx : n \in \Z} \text{ using additive notation}\right)$$
    We write $<x> = H$ ($x$ generates $H$).
\end{definition}

\begin{example}
    \begin{enumerate}
        \item $\Z=<1>=<-1>$
        \item The rotations in $D_{2n}$ are generated by $r$ ($360/n$ clockwise rotation)
        \item $U_4 = {1, -1, i, -i} = <i>$
    \end{enumerate}
\end{example}

\begin{note}
    If $H=<x> = \set{x^n : n \in \Z}$, we define 
    \begin{align*}
        x^0 &= 1 \\
        x^{-n} &= (x^n)^{-1} = (x^{-1})^n \text{ for } n > 0
    \end{align*}
\end{note}

\begin{proposition}
    If $H=<x>$, then $|H| = |x|$. If one side of this equality is infinity, then so is the other. More specifically,
    \begin{enumerate}
        \item If $|x| = n < \infty$, then $x^n = 1$ and $1, x, x^2, ..., x^{n-1}$ are all the distinct elements of $H$.
        \item If $|x| = \infty$, then $x^n \not = 1$ when $n \not = 0$ and $x^a \not = x^b$ for all $a \not = b \in \N$.
    \end{enumerate}
\end{proposition}

\begin{proof}
    Let $|x| = n$.
    \begin{enumerate}
        \item Consider the case where $n < \infty$. Consider the elements $1, x, x^2, ..., x^{n-1}$ and suppose $x^a = x^b$ where $0 \leq a < b < n$. Then 
        \begin{align*}
            x^a = x^b &\implies 1 = x^bx^{-a} \\
            &\implies 1 = x^{b-a}
        \end{align*}
        Since $b-a > 0$, this contradicts $n$ being the order of $x$. Thus, all the $1, x, x^2, ..., x^{n-1}$ are distinct. Also, $x^n = 1$ as $n = |x|$. Thus $H$ contains at least $n$ elements. It remains to show we have all of them. \\
        Let $t \in Z \st x^t \in H$. By the division algorithm, there exitst $q, r \in \Z$ such that 
        $$t = qn + r \text{ where } 0 \leq r < n$$
        Then
        \begin{align*}
            x^t = x^{qn+r} &= x^{qn}x^r \\
            &= (x^n)^qx^r \\
            &= 1^q x^r \\
            &=x^r \in \set{1, x, x^2, ..., x^{n-1}} \text{ since } 0 \leq r < n
        \end{align*}
        Hence, $H = \set{1, x, x^2, ..., x^{n-1}}$.

        \item Next, suppose $|x| =\infty$ (no positive powers of $x$ is the identity). For the sake of contradiction, if $x^a=x^b$ with $a < b$ then $x^{a-b}=1$, a contradiction. So distinct powers of $x$ give distinct elements of $H$. It follows that $|H| =\infty$.
    \end{enumerate}
    \qed
\end{proof}

\begin{proposition}
    Let $G$ be a group and let $x \in G$. Let $m,n \in \Z.$ If $x^n = 1$ and $x^m = 1,$ then $x^d = 1$ where $d = \gcd (m,n).$ In particular, if $x^m = 1$ for some $m \in \Z$ then $|x| | m$.
\end{proposition}

\begin{proof}
    Let $m, n, d$ be defined as above. Then by the Euclidean algorithm
    $$\exists x_0, y_0 \in \Z \st d = mx_0 +ny_0$$
    Then 
    \begin{align*}
        x^d &= x^{mx_0 + ny_0} \\
        &= (x^m)^{x_0}(x^n)^{y_0} \\
        &=1^{x_0}1^{y_0} \\
        &=1
    \end{align*}
    To prove the second assertion, let $x^m = 1$ and $n = |x|$. Then $x^n = 1$ by definition of order.
    \begin{description}
        \item[Case 1: ] If $m = 0$ then certainly $n | m$.
        \item[Case 2: ] Let $m \not = 0$. We know $n < \infty$ since $x^m = 1$. Let $d = \gcd(m,n)$ and hence by the first assertion $x^d = 1$. Since $0 < d \leq n$ and $n$ is the smallest positive integer such that $x^n = 1$, we have that $n = d.$ By definition,
        $$d | m \implies n | m \text{ as desired.}$$
    \end{description}
    \qed
\end{proof}

\begin{theorem}[Cyclic Groups Isomorphisms] \leavevmode\\
    \begin{enumerate}
        \item Any infinite cyclic group $<x>$ is isomorphic to $\Z$ (with the mapping $\phi : \Z \to <x>$, $k \mapsto x^k$).
        \item If $<x>$ and $<y>$ are cyclic groups both with order $n < \infty$, then
        \begin{align*}
            \phi: <x> &\to <y> \\
            x^k &\mapsto y^k
        \end{align*}
        is a well-defined isomorphism.
    \end{enumerate}
\end{theorem}

We will use multiplicative notation when describing an arbitrary cyclic group of order $n \in \N$, and denote this group $\Z_n.$ NOT to be confused with the additive group $\Z/n\Z$, which is cyclic of order $n$. Most times we will refer to an infinite cyclic group as $\Z$.

\begin{proposition}[The Order of $x^a$ in a Cyclic Group] \leavevmode\\
    \label{prop5}
    Let $G$ be a group and let $x \i9n G$. Let $a \in \Z-\{0\}.$
    \begin{enumerate}
        \item If $|x| = \infty,$ then $|x^a| = \infty$.
        \item If $|x| = n < \infty,$ then $|x^a| = \frac{n}{\gcd(n,a)}$.
    \end{enumerate}
    In particular, $|x^a| = \frac{n}{a}$ when $a|n$ ($a \in \N$).
\end{proposition}

\begin{proof}
    We start with the following claim: Let $a,n, \in \Z$ not both zero.
    $$\text{If $\gcd(a,n) = d$ then $\gcd(\frac{a}{d}, \frac{n}{d})=1$}$$

    \begin{proof}
        Let $a,n$ and $d$ be as defined. Then there exists $x_0, y_0 \in Z$ such that 
        $$d = ax_0 + ny_0$$
        It follows that
        $$1 = \frac{a}{d}x_0 + \frac{n}{d}y_0$$
        Since $\gcd(\frac{a}{d}, \frac{n}{d})$ divides $\frac{a}{d}$ and $\frac{n}{d}$, $\gcd(\frac{a}{d}, \frac{n}{d})$ divides the right-hand side, so $\gcd(\frac{a}{d}, \frac{n}{d}) | 1$. Thus, $\gcd(\frac{a}{d}, \frac{n}{d}) = 1$.
        \qed
    \end{proof}
    \begin{enumerate}
        \item Suppose by way of contradiction that 
        $$|x| = \infty \text{ and } |x^a| = m < \infty$$
        By definition of order
        $$(x^a)^m = 1 \iff x^{am} = 1$$
        It follows that
        $$(x^{am})^{-1} = 1^{-1} \iff x^{-am} = 1$$
        Since $a \not = 0$ by assumption and $m \not = 0$ by definition of order, then $am \not = 0$ and one of $-am$ or $am$ is positive, so some positive power of $x$ is the identity, contradicting $|x| = \infty.$ So, $|x^a| = \infty$.

        \item Let $|x| = n < \infty$ and let $y = x^a$, $\gcd(a,n) = d$. We also write $n = db$ and $a = dc$ for some integers $c, b$ (not thate $ b > 0$). From our claim,
        $$\gcd(c,b) = \gcd(\frac{a}{d}, \frac{n}{d}) = 1$$
        We want to show that $|y| = b$. To this end, cotice that
        \begin{align*}
            y^b = (x^a)^b &= x^{ab} \\
            &= x^{(dc)b} \\ 
            &= x^{(dc)(\frac{n}{d})} \\
            &= (x^n)^c \\ 
            &= 1^c \\
            &= 1
        \end{align*}
        Thus, $|y|$ divides $b$. Let $k = |y|$. Then 
        $$y^k = 1 = x^{ak}$$
        Hence, $|x| \mid ak$. That is, 
        \begin{align*}
            n \mid ak &\iff db \mid dck \\
            &\iff b \mid ck \\
            &\iff \frac{n}{d} \mid \frac{a}{d}k
        \end{align*}
        Since $\frac{n}{d}$ and $\frac{a}{d}$ are relatively prime, this gives $\frac{n}{d} \mid k$, that is $b \mid k$. Since $b \mid k$ and $k \mid b$, $k = b$ as both $k,b \in \N$.
        \qed
    \end{enumerate}
\end{proof}

\begin{proposition}
    Let $H = <x>$.
    \begin{enumerate}
        \item Assume $|x| = \infty.$ then $H = <x^a>$ if and only if $a = \pm 1$.
        \item Assume $|x| = n \infty$. Then $H = <x^a>$ if and only if $\gcd(a,n) = 1$. In particular, the number of generators of $H$ is $\phi(n)$, where $\phi$ is Euler's Phi funciton.
    \end{enumerate}
\end{proposition}

\begin{proof}
    2. If $|x| = n < \infty$, we know that $|x^a| = |<x^a>|.$ This subgroup equals all of $H \iff |x^a| = n \iff \frac{n}{\gcd(a,n)}=n \iff \gcd(a,n) = 1.$ Since $\phi(n)$ is the number of $a \in \set{1, 2, 3,..., n}$, which are relatively prime to $n$, $\phi(n)$ gives the number of generators of $H$.
    \qed
\end{proof}

What are the generators of $<x> = \Z_{10}$? $\phi(1) = \phi(2)\phi(5) = 4$
$$x^1, x^3, x^7, x^9$$
What are the generators of $\Z/15\Z = <\overline{1}> = \set{k\dot 1 : k \in \Z}$?
$$\overline{1}, \overline{2}, \overline{4}, \overline{7}, \overline{8},\overline{11},\overline{13},\overline{14}$$

\begin{theorem}[Subgroups of Cyclic Groups] \leavevmode\\
    Let $H=<x>$ be a cyclic group.
    \begin{enumerate}
        \item Every subgroup of $H$ is cyclic. More precisely, if $K \leq H$ then either
        $$K = \set{1} \text{ or } K = <x^d>$$
        where $d$ is the smallest positive integer such that $x^d \in K$.
        \item If $|H| = \infty$, then for any distinct nonnegative integers $a$ and $b$
        $$<x^a> \not = <x^b>$$
        and $\forall m \in \Z$
        $$<x^m> = <x^{|m|}>$$
        where $|m|$ denotes the absolute value of $m$. So, the nontrivial subgroups of $H$ correspond bijectively with the integers $1, 2, 3, ...$
        \item If $|H| = n < \infty$, then for every $a \in \N$ which divides $n$, there is a unique subgroup $H$ with order $a$. This subgroup is the cyclic group $<x^d>$ where $d = \frac{n}{a}$. Furthermore, for every $m \in \Z$, $<x^m> = <X^{\gcd(n,m)}>$ so the subgroups of $H$ correspond bijectively with the positive divisors of $n$.
    \end{enumerate}
\end{theorem}

\begin{proof}
    \begin{enumerate}
        \item Let $K \leq H$. If $K = \set{1}$, then we are done. Suppose $K \not = \set{1}$. Thus, there exists some $a \not = 0 \st x^a \in K$. Since $K$ is a group, $(x^a)^{-1} \in K.$ That is, $x^{-a} \in K$, and since either $a$ or $-a$ must be positive the set of all positive powers of $x \st x$ to that positive power is an element of $K$ is nonempty. That is,
        $$P = \set{n \in \N : x^n \in K} \not = \emptyset$$
        Thus, by the well-ordering principle, the set $P$ contains a minimal element, call it $d$. By definition, $x^d \in K.$ and since $K$ is a group $<x^d> \leq K.$ Let $k \in K.$ Then, $k = x^b$ for some $b \in \Z.$ By the division algorithm, we have integers $q,r$, such that
        $$b = qd +r \text{ where }0 \leq r < d$$
        Hence,
        \begin{align*}
            &x^b = x^{qd+r} \\
            \implies &x^b = (x^{qd})x^r = (x^d)^qx^r \\
            \implies (x^d)^{-q}&x^b = x^r
        \end{align*}
        Since $x^d, x^b \in K$ and $K$ is a group,
        $$(x^d)^{-q} \in K \text{ and } (x^d)^{-q}x^b \in K$$
        so $x^r \in K$. However, since $d$ is the minimal positive power of $x$ such that $x^d \in K$, $r$ must not be a positive power. Therefore, $r = 0$ and it follows that
        $$k = x^b = (x^d)^q \in <x^d>$$
        Therefore, $K \leq <x^d>.$ This gives $<x^d> = K$.

        \item Suppose $|H| = n < \infty$ and $a \mid n$ where $a \in \Z$. Let $d = \frac{n}{a}$. Hence
        $$|<x^d>| = \frac{n}{n/a} = a$$

        \begin{description}
            \item[Uniqueness: ] To show uniqueness, suppose $K$ is any subgroup of $H$ of order $a$. Then by part 1, $K = <x^b>$ where $b$ is the smallest positive integer such that $x^b \in K$. We know
            $$\frac{n}{d} = a = |K| = |x^b| = \frac{d}{\gcd(n,b)}$$
            It follows that 
            $$d = \gcd(n,b)$$
            Hence, $d \mid b$ by definition and $x^b \in <x^d>$. It follows that 
            $$K = <x^b> \leq <x^d>$$
            and so $K = <x^d>$ as they have the same order. The final assertion follows from the fact that 
            $$<x^m> \leq <x^{\gcd(m,n)}>$$
            and \ref{prop5} (2) says
            $$\left|<x^m>\right| = \frac{n}{\gcd(n,m)}$$
            and
            $$\left|x^{\gcd(m,n)}\right|=\frac{n}{\gcd(n, \gcd(m,n))}$$
            and we know $\gcd(n, \gcd(m,n)) = \gcd(n,m)$. Since $\gcd(,m,n) \mid n$ this shows that every subgroup of $H$ arises from a divisor of $n$.
            \qed
        \end{description}
    \end{enumerate}
\end{proof}