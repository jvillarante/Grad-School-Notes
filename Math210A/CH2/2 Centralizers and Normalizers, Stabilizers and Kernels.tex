\begin{definition} [Centralizers] \leavevmode \\
    Let $A$ be a nonempty subset of a group $G$. Define the centralizer of $A$ in $G$ to be the set
    \begin{align*}
        C_G(A) &= \{ g \in G : gag^{-1} = g ~~\forall a \in A \} \\
        &= \set{g \in G : ga = ag ~~\forall a \in A}
    \end{align*}
\end{definition}

The centralizer of $A$ in $G$ is the set of all elements in $G$ which commute with every element in $A$.

\begin{theorem}
    $C_G(A) \leq G$.
\end{theorem}

\begin{proof}
    Let $a \in A.$ Then 
    \begin{align*}
        1a1^{-1} &= (1a)1^{-1} \\
        &= a1^{-1} \\
        &= a1 \\
        &= a
    \end{align*}
    Thus, $1 \in C_G(A)$.

    Let $x,y \in C_G(A)$. Then $xax^{-1} = a$ and $yay^{-1}=a.$ Note that
    \begin{equation*}
        yay^{-1} = a \iff a = y^{-1}
        \tag{$*$}
    \end{equation*}
    Now
    \begin{align*}
        (xy^{-1})a(xy^{-1})^{-1} &= xy^{-1}a(y^{-1})^{-1}x^{-1} \\
        &= x(y^{-1}ay)x^{-1} \\
        &\overset{(*)}{=} xax^{-1} \\
        &= a
    \end{align*}
    Hence, $xy^{-1} \in C_G(A)$. Furthermore, $C_G(A) \leq G.$
    \qed
\end{proof}

\begin{notation}
    If $A = \set{a}$, we write $C_G(a)$ instead of $C_G(\set{a})$.
\end{notation}

Why was this unnecessary? From the homework, we know that $G$ acts on the subset $A$ by conjugation. That is, we have a mapping $(.): G\times A \to A$ defined by $g.a = gag^{-1} ~~\forall g \in G, a \in A$ which satisfies both axioms of a group action.

Recall that the kernel of a group action is the kernel of the permutation representation of the group action (PRGA). The PRGA is the Homomorphism induced by the group action
\begin{align*}
    \Psi : G \to S_A \\
    g \mapsto \sigma_g
\end{align*}

\begin{example}
    Find the kernel of $G$ acting on $A \subset G$ by conjugation.
    \begin{align*}
        \set{g \in G : g.a = a ~~ \forall a \in A} &= \set{g \in G : gag^{-1} = a ~~\forall a \in A} \\
        &= C_G(A)
    \end{align*}
\end{example}

Suppose that $A = G$. What is $C_G(G)$?
$$\set{g \in G : gag^{-1}=a ~~\forall a \in G}$$
This set is called the center of $G$ denoted $Z(G)$. Since $Z(G)$ is a special case of $C_G(A)$, we know $Z(G) \leq G$.

\begin{definition}[Normalizer] \leavevmode \\
    Define $gAg^{-1} = \set{gag^{-1} : a \in A}$. We will define the normalizer of $A$ in $G$ to be the set 
    $$N_G(A) = \set{g \in G : gAg^{-1} = A}$$
\end{definition}

We will prove $N_G(A) \leq G$, but not yet.
Notice if $gag^{-1} = a ~~ \forall a \in A$ then $gAg^{-1} = \set{gag^{-1} : a \in A} = \set{a : a \in A} = A$. Hence
$$C_G(A) \subseteq N_G(A)$$

\begin{fact} \leavevmode \\
    \begin{enumerate}
        \item If $G$ is abelian, then $Z(G) = G$ since every element commutes with every other element. That is,
        \begin{align*}
        \forall a,b \in G ~~ab = ba &\iff a = bab^{-1} ~~ \forall a,b \in G \\ &\implies b \in Z(G) ~~\forall b \in G
        \end{align*}
        Similarly, $C_G(A) = N_G(A) = G.$
        \item Consider $A=\set{1, (1 ~2)} \subseteq S_3$. Find $C_{S_3}(A)$. Notice that $1$ commutes with everything in $S_3$, specifically $1$ and $(1~2)$. Also,
        $$(1~2)(1~2)(1~2)^{-1} = (1~2)$$
        so $(1~2)\in C_{S_3}(A)$. Hence, $A \leq C_{S_3}(A)$.
        \begin{theorem}[Lagrange's Theorem] \leavevmode \\
            Let $G$ be a finite group $\left(|G| \in \N\right)$ and let $H \leq G$. Then
            $$|H| \text{ divides } |G|$$
        \end{theorem}
        Since $|A| = 2$ and $A \leq C_{S_3}(A)$, we know $2 \big| |C_{S_3}(A)|$ since $C_{S_3}(A) \leq S_3$.
        $$
        \begin{rcases*}
            |C_{S_3}(A)| \big| |S_3| = 3! = 6 \\
            |A| \big| |C_{S_3}(A)|
        \end{rcases*} \implies |C_{S_3}(A)| \in \set{2, 6}$$.
        Thus, $C_{S_3}= A$ or $C_{S_3}(A) = S_3$. Well,
        \begin{align*}
            (1~2)(1~2~3) = (2~3) \\
            (1~2~3)(1~2) = (1~3)
        \end{align*}
        so $(1~2~3) \not \in C_{S_3}(A)$. It follows that $|C_{S_3}(A)| = 2 \implies C_{S_3}(A) = A.$
    \end{enumerate}
\end{fact}

Let $G$ be a group acting on a set $S$. That is, there is a mapping
$$(., .): G\times S \to S$$
denoted by $g.a ~~\forall a \in S$ with $g_1.(g_2.a) = (g_1g_2).a$ and $1.a = a ~~\forall a \in S, g_1,g_2 \in G$.

\begin{definition} [Stabilizers] \leavevmode \\
    If $G$ is a group acting on a set $S$ and $s \in S$, then we define the stabilizers of $s$ in $G$ to be the set 
    $$G_s = \set{g \in G : g.s = s}$$
\end{definition}

\begin{theorem}
    $G_s \leq G$.
\end{theorem}

\begin{proof}
    Since $G$ acts on $S$ we know that $1.s = s$. Hence $1 \in G_s \implies G_s \not = \emptyset$. Let $x,y \in G_s$. Then 
    \begin{align*}
        s = 1.s &= (y^{-1}y).s \\ 
        &= y^{-1}.(y.s) \\ 
        &= y^{-1}.s ~~~~\text{ (since $y \in G_s$)}
    \end{align*}
    Hence $y^{-1} \in G_s$. Furthermore, 
    \begin{align*}
        (xy).s &= x.(y.s) \\
        &= x.s \\ 
        &= s
    \end{align*}
    Hence $xy \in G_s$. Thus, $G_s \leq G$.
    \qed
\end{proof}

Now to show $N_G(A)$ where $A \subseteq G$ is a subgroup of $G$. To that end, let $S = \mathcal{P}(G)$, the power set of $G$, and define a map
$$G\times S \to S \text{ by } g.B = gBg^{-1} = \set{gbg^{-1}:\forall g \in G, B \in \mathcal{P}(G)}$$
Let's prove this defines a group action.
Let $g_1, g_2 \in G$ and $B \in \powerset(G)$. Well,
$$1.B = \set{1b1^{-1} : b \in B} = \set{b : b \in B} = B$$
so the identity axiom holds. Furthermore,
\begin{align*}
    (g_1g_2).B &= (g_1g_2)B(g_1g_2)^{-1} \\
    &= \set{(g_1g_2)b(g_1g_2)^{-1} : b \in B} \\
    &= \set{(g_1g_2)b(g_2^{-1}g_1^{-1}) : b \in B} \\
    &= \set{g_1(g_2b g_2^{-1})g_1^{-1} : b \in B} \\
    &= \set{g_1b'g_1^{-1} : b' \in g_2Bg_2^{-1}} \\
    &= g_1 \left(g_2Bg_2^{-1}\right)g_1^{-1} \\
    &= g_1\left(g_2.B\right)g_1^{-1} \\
    &= g_1.(g_2.B)
\end{align*}
Hence, we have defined a group action. Now, back to showing that $N_G(A) \leq G$ ($A \subseteq G$).

Recall, $G_s = \set{g \in G : g.s=s}$. Given our new group action $G$ acting on $\powerset(G)$ by conjugation, we have 
\begin{align*}
    G_a &= \set{g \in G : g.A = A} \\
    &= \set{g \in G : gAg^{-1} = A} \\
    &= N_G(A)
\end{align*}
We can then deduce that $N_G(A) \leq G$ as $G_A \leq G$.